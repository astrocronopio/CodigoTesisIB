
%%%%%%%%%%%%%%%%%%%%%%%%%%%%%%%%%%%%%%%%%%%%%%%%%%%%%%%%%%%%%%%%%%%%%%%%
%Para las ecuaciones siempre es Ec.(n).
%Para las figuras siempre es Fig.n, incluso en el caption de la figura. Tambien las Tablas
%Para las referencias es [n]
%%%%%%%%%%%%%%%%%%%%%%%%%%%%%%%%%%%%%%%%%%%%%%%%%%%%%%%%%%%%%%%%%%%%%%%%

\documentclass[
reprint,
%notitlepage,
%superscriptaddress,
%groupedaddress,
%unsortedaddress,
%runinaddress,
%frontmatterverbose, 
%preprint,
%showpacs,preprintnumbers,
%nofootinbib,
%nobibnotes,
%bibnotes,
%11 pt,
amsmath,
amssymb,
aps,
%pra,
%prb,
%rmp,
%tightenlines %esto hizo el milagro de sacar los espacios en blancos estocásticos (?)
 %prstab,
%prstper,
%floatfix,\textbf{}
]{revtex4-1} %Instalar primero para usarlo. Paquete malo.

%\documentclass[onecolumn, aps, amsmath,amssymb ]{article}
\usepackage{lipsum}  
\usepackage{graphicx}% Include figure files
\usepackage{subfig}
\usepackage{braket}
\usepackage{comment} %comment large chunks of text
\usepackage{dcolumn}% Align table columns on decimal point
\usepackage{bm}% bold math
%\usepackage{hyperref}% add hypertext capabilities
\usepackage[mathlines]{lineno}% Enable numbering of text and display math
%\linenumbers\relax % Commence numbering lines
\usepackage{mathtools} %% Para el supraíndice

\usepackage[nice]{nicefrac}

%%%%%%%El Señor Español%%%%%%%%%%%%%%%%%%%%%%%%%%%
\usepackage[utf8]{inputenc} %acento
\usepackage[
spanish, %El lenguaje.
es-tabla, %La tabla y no cuadro.
activeacute, %El acento.
es-nodecimaldot %Punto y no coma con separador de números
]{babel}
\usepackage{microtype} %para hacerlo más bonito :33 como vos (?) 
%%%%%%%%%%%%%%%%%%%%%%%%%%%%%%%%%%%%%%%%%%%%%%%%%%%
%%%%%%%%% Para que las imágenes se queden dónde las quiero (?
\usepackage{float}
%%%%%%%%%%

%%%%%%%%Cambia a Fig de Figure%%%%%%%%%%
\makeatletter
\renewcommand{\fnum@figure}{Fig. \thefigure} 
\makeatother
%%%%%%%%%%%%%%%%%%%%%%%%%%%%%%%%%%%%%%%%
\raggedbottom


\begin{document}
%%%%%%%%%%%%%%%%%%%%%%%%%%%%%%%%%%Título%%%%%%%%%%%%%%%%%%%%%%%%%%%%%%%%%%%%%%
%%%%%%%%%%%%%%%%%%%%%%%%%%%%%%%%%%%%%%%%%%%%%%%%%%%%%%%%%%%%%%%%%%%%%%%%%%%%%%

\title{Anisotropías para todos los disparos: sin y con pesos de los hexágonos}
\author{Evelyn~G.~Coronel}

\affiliation{
Tesis de Maestría en Ciencias Físicas\\ Instituto Balseiro\\}

\date[]{\lowercase{\today}} %%lw para lw, [] sin date

%\begin{abstract}

%\end{abstract} 
\maketitle
%%%%%%%%%%%%%%%%%%%%%%%%%%%%%%%%%%%%%%%%%%%%%%%%%%%%%%%%%%%%%%%%%%%%%%%%%%%%%%%%%%%
% Podemos usar cualquiera de los dos comandos: \input o \include para incluir el texto

\subsubsection{Nomenclatura}

			\begin{table}[H]
			\centering
				\begin{tabular}{c|c|c|c}
			Archivo AllTriggers  & \text{Eventos} & UTC inicial &  UTC final  \\ \hline
				2020			 & 13 739 351	  &  1372680068	&  1577879983 \\ %2019
				2019			 & 	8 463 063	  &	 1372680068 &  1496318388 \\ %Herald
				2017			 &	8 592 302	  &  1372680068 &  1498521517 \\ %Oscar
					\end{tabular}
			\end{table}



\section{Anisotropías sin considerar el peso de los hexágonos}


\section{Anisotropías  considerando el peso de los hexágonos}

\subsection{Variación de los pesos en función de la ascensión recta}

\subsubsection{Energía entre 1\,EeV y 2\,EeV}

Para este caso utilizamos el archivo con todos los disparos
%fig


\end{document}
