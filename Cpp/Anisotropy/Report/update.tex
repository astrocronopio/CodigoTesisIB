\input{Preamble.sty}

\begin{document}
%%%%%%%%%%%%%%%%%%%%%%%%%%%%%%%%%%Título%%%%%%%%%%%%%%%%%%%%%%%%%%%%%%%%%%%%%%
%%%%%%%%%%%%%%%%%%%%%%%%%%%%%%%%%%%%%%%%%%%%%%%%%%%%%%%%%%%%%%%%%%%%%%%%%%%%%%

\title{Anisotropías para todos los disparos: sin y con pesos de los hexágonos}
\author{Evelyn~G.~Coronel}

\affiliation{
Tesis de Maestría en Ciencias Físicas\\ Instituto Balseiro\\}

\date[]{\lowercase{\today}} %%lw para lw, [] sin date

%\begin{abstract}

%\end{abstract} 
\maketitle
%%%%%%%%%%%%%%%%%%%%%%%%%%%%%%%%%%%%%%%%%%%%%%%%%%%%%%%%%%%%%%%%%%%%%%%%%%%%%%%%%%%
% Podemos usar cualquiera de los dos comandos: \input o \include para incluir el texto

\subsubsection{Nomenclatura}

			\begin{table}[H]
			\centering
				\begin{tabular}{c|c|c|c}
			Archivo AllTriggers  & \text{Eventos} & UTC inicial &  UTC final  \\ \hline
				2020			 & 13 739 351	  &  1372680068	&  1577879983 \\ %2019
				2019			 & 	8 463 063	  &	 1372680068 &  1496318388 \\ %Herald
				2017			 &	8 592 302	  &  1372680068 &  1498521517 \\ %Oscar
					\end{tabular}
			\end{table}



\section{Anisotropías sin considerar el peso de los hexágonos}


\section{Anisotropías  considerando el peso de los hexágonos}

\subsection{Variación de los pesos en función de la ascensión recta}

\subsubsection{Energía entre 1\,EeV y 2\,EeV}

Para este caso utilizamos el archivo con todos los disparos
%fig


\end{document}
