
\chapter{Resultados del método EW}


\section{Tabla cantidad de eventos para distintos rangos de energía}

Los eventos son clasificados en los distintos rangos con la energía reportada el archivo del Herald de todos los disparos  entre el 2014 y 2019 y para el disparo estándar entre el 2004 y 2018.


\begin{table}[H]
    \begin{small}
        \begin{center}
            \begin{tabular}{|l|l|l|l|l|}
                \hline
                \multicolumn{2}{|l|}{Rango}                                                          & 0.25 EeV- 0.5 EeV & 0.5 EeV - 1 EeV & 1 EeV - 2 EeV \\ \hline
                \multirow{2}{*}{Eventos}                                                  & Todos    & $3\,967\,368$     & $3\,638\,226$   & $1\,081\,846$ \\ \cline{2-5} 
                                                                                          & Estandar & $770\,323$        & $2\,388\,468$   & $1\,243\,098$ \\ \hline
                \multirow{2}{*}{\begin{tabular}[c]{@{}l@{}}Energía \\ Media\end{tabular}} & Todos    & $0.375$           & $0.687$         & $1.315$       \\ \cline{2-5} 
                                                                                          & Estandar & $0.42$            & $0.71$          & $1.34$.       \\ \hline
                \end{tabular}
            \caption{Tabla de eventos por rango de energía }
            \label{tab:}
        \end{center}
    \end{small}
\end{table}


\section{Resultados en distintos rangos de energía}
\subsection{Resultados en el rango 0.25 EeV - 0.5 EeV}

En la Fig. \ref{fig:primer} se comparan las direcciones en las que apuntan la fase en frecuencia sidérea obtenida en este trabajo con la obtenida en \cite{Aab_2020}. 
Las fases tiene un margen donde se solapan en la incertidumbre pero no son comparables, la línea punteada marca la dirección del centro galáctico.
\begin{table}[H]
    \begin{small}
        \begin{center}
            \begin{tabular}[c]{l|c||c|c}
                Frecuencia:                 & 365.25	  & 366.25	                    & 366.25 \cite{Aab_2020}   \\ 
                \hline
                Amplitud r [\%]:            & 0.17	      & $0.12^{+0.24}_{-0.03}$ 	    & $0.5^{+0.4}_{-0.2}$ \cite{codigo}      \\
                $r_{99}$ [\%]:              & 0.73	      & 0.6                         & 1.5\cite{codigo}                 \\
                $r^{UL}$ [\%]:              & - 	      & -                           & -\cite{codigo}                 \\
                \hline
                Amplitud $d_\perp$[\%]:     & -	          & $0.16^{+0.31}_{-0.04}$ 	    & $0.6^{+0.6}_{-0.3}$       \\
                $d_{99}$ [\%]:              & - 	      & -                           & -                 \\
                $d_{\perp,UL}$[\%]:         & -           & 0.8       & 1.8      \\
                \hline
                $\sigma_{x,y}$[\%]:         & -	          & 0.24	   & 0.48       \\
                Probabilidad      :         & 0.66        & 0.81	   & 0.45       \\
                Fase[$^o$]:                 & 221$\pm$63  & 280$\pm$88 & 225$\pm$64\cite{discrepancia} \\
            \end{tabular}
        \end{center}
    \end{small}
    \caption{Características para las frecuencias solar y sidérea con el método East-West en el primer armónico en rango de energía 0.25 EeV - 0.5 EeV}
    \label{tab:solar}
\end{table}


Realizando el barrido de frecuencias con la variable de la Ec.\ref{ra_arb}, se obtiene que en este rango de energía las amplitudes se  distribuyen en frecuencia como se muestra en la Fig.\ref{fig:primer_barrido}. La línea horizontal indica el valor de $r_{99}$ para cada frecuencia, además se observa que ninguna frecuencia supera dicho umbral.

\begin{figure}[H]
    \begin{small}
        \begin{center}
            \includegraphics[width=0.75\textwidth]{phase_primer_bin.pdf}
        \end{center}
        \caption{Valores de las fases obtenidos en este trabajo y en la referencia con sus respectivas incertidumbres para la frecuencia sidérea en el  rango 0.25 EeV - 0.5 EeV .}
        \label{fig:primer}
    \end{small}
\end{figure}

\begin{figure}[H]
    \begin{small}
        \begin{center}
            \includegraphics[width=0.75\textwidth]{plot_bin_1_barrido_v3_EW.pdf}
        \end{center}
        \caption{Barrido de frecuencias en el  rango 0.25 EeV - 0.50 EeV .}
        \label{fig:primer_barrido}
    \end{small}
\end{figure}

\subsection{Resultados en el rango 0.5 EeV - 1 EeV}
En este rango de energía se observa una diferencia entre las probabilidades de este trabajo y \cite{Aab_2020}  ne la frecuencia sidérea. Este valor dice cuando probable es que las amplitudes sean debido al ruido. Este trabajo obtiene que la amplitud en sidérea es significativa por un  $6\%$.  

En la Fig. \ref{fig:segundo} se comparan las direcciones en las que apuntan la fase en frecuencia sidérea obtenida en este trabajo con la obtenida en \cite{Aab_2020}. En esta figura se observa que las fases son comparables entre sí y apuntan a una dirección cercana al centro galáctico (línea punteada).

El barrido de frecuencias con la variable de la Ec.\ref{ra_arb} para este rango de energía se observa en la Fig.\ref{fig:segundo_barrido}. La línea horizontal indica el valor de $r_{99}$ para cada frecuencia, además se observa que ninguna frecuencia supera dicho umbral. Otro aspecto es que la zona de la frecuencia anti-sidérea no tiene picos pronunciados, como en la frecuencia solar o sidérea.

\begin{figure}[H]
    \begin{small}
        \begin{center}
            \includegraphics[width=0.75\textwidth]{phase_segundo_bin.pdf}
        \end{center}
        \caption{Valores de las fases obtenidos en este trabajo y en la referencia con sus respectivas incertidumbres para la frecuencia sidérea en el  rango 0.5 EeV - 1.0 EeV .}
        \label{fig:segundo}
    \end{small}
\end{figure}

\begin{table}[H]
        \begin{small}
            \begin{center}
                \begin{tabular}[c]{l|c||c|c}
                    Frecuencia:             & 365.25	    & 366.25		& 366.25\cite{Aab_2020}\\
                    \hline
                    Amplitud r [\%]:        & 0.42          & $0.4^{+0.2}_{-0.1}$ 	        & $0.40^{+0.2}_{-0.1}$\cite{codigo}  \\
                    $r_{99}$[\%]:           & 0.70	        & 0.6          & 0.8\cite{codigo}  \\
                    $d_{\perp,UL}[\%]$      & -             & 1.1          & 1.1\\                    
                    \hline
                    Amplitud $d_\perp$[\%]: & -             & $0.6^{+0.3}_{-0.2}$           & $0.50^{+0.3}_{-0.2}$\\
                    $r_{99}$[\%]:           & 0.70	        & 0.6          & 0.8\cite{codigo}  \\
                    $d_{\perp,UL}[\%]$      & -             & 1.1          & 1.1\\

                    $\sigma_{x,y}$[\%]:     & -	            & 0.23	        & 0.27       \\
                    Probabilidad:           & 0.06          & 0.06	        & 0.20\\
                    Fase[$^o$]:             & 205$\pm$25	& 258$\pm$24	& 261$\pm$43\cite{discrepancia}   \\
                \end{tabular}
            \end{center}
        \end{small}
        \caption{Características para las frecuencias solar y sidérea con el método East-West en el primer armónico en rango de energía 0.5 EeV - 1 EeV}
        \label{tab:solar}
    \end{table}


    \begin{figure}[H]
        \begin{small}
            \begin{center}
                \includegraphics[width=0.75\textwidth]{plot_bin_2_barrido_v3_EW.pdf}
            \end{center}
            \caption{Barrido de frecuencias en el  rango 0.5 EeV - 1.0 EeV .}
            \label{fig:segundo_barrido}
        \end{small}
    \end{figure}    


\subsection{Resultados en el rango 1 EeV - 2 EeV}

 
En las Tablas \ref{tab:solar_3} y \ref{tab:siderea_3} se comparan los resultados de este trabajo y los obtenidos en \cite{Aab_2020} para la frecuencia solar y sidérea respectivamente. En el Fig.\ref{fig:tercer} se observan en un gráfico polar las fases de la referencia y este trabajo para la frecuencia sidérea. Los resultados son comparables entre sí.
    
    \begin{table}[H]
        \begin{small}
            \begin{center}
                \begin{tabular}[c]{l|c|c}
                                    & Rayleigh      & EW            \\\hline
                    Frecuencia:     & 365.25	    & 365.25        \\
                    Amplitud $r$  [\%]:  & 0.39     & 0.28     \\
                    Probabilidad:   & 0.02          & 0.64          \\
                    Fase:           & 288$\pm$20    & 279$\pm$61    \\
                    $r_{99}$ [\%]:  & 0.41263       & 1.2       \\
                \end{tabular}
            \end{center}
        \end{small}
        \caption{Características para la frecuencia solar con los métodos de Rayleigh  e East-West en el primer armónico.}
        \label{tab:solar_3}
    \end{table}

    \begin{table}[H]
        \begin{small}
            \begin{center}
                \begin{tabular}[c]{l|c||c|c}
                                            & Rayleigh    & EW                          & EW\cite{Aab_2020}      \\\hline
                    Frecuencia:             & 366.25	  & 366.25                      & 366.25        \\
                    Amplitud $r$ [\%]:      & 0.40	      & $0.5^{+0.3}_{-0.2}$         & $0.14^{+0.37}_{-0.02}$\cite{codigo}       \\
                    Amplitud $d_\perp$ [\%]:& 0.51        & $0.6^{+0.4}_{-0.3}$         & $0.18^{+0.47}_{-0.02}$       \\ 
                    $\sigma_{x,y}$[\%]:     & -	          & 0.38	                    & 0.35          \\
                    Probabilidad:           & 0.012	      & 0.26                        & 0.87          \\
                    Fase[$^o$]:             & 330$\pm$20  & 320$\pm$30                  & 291$\pm$100 \cite{discrepancia}      \\
                    $r_{99}$[\%]:           & 0.41	      & 0.9                         & 0.84\cite{codigo}        \\
                    $d_{\perp,UL}[\%]$      & 0.53        & 1.6                         & 1.1        \\
                \end{tabular}
            \end{center}
        \end{small}
        \caption{Características para la frecuencia sidérea con los métodos de Rayleigh  e East-West en el primer armónico.}
        \label{tab:siderea_3}
    \end{table}
   
    \begin{figure}[H]
        \begin{small}
            \begin{center}
                \includegraphics[width=0.75\textwidth]{phase_tercer_bin.pdf}
            \end{center}
        \caption{Valores de las fases obtenidos en este trabajo y en la referencia con sus respectivas incertidumbres para la frecuencia sidérea en el  rango 1.0 EeV - 2.0 EeV .}
        \label{fig:tercer}
        \end{small}
    \end{figure}


    El barrido de frecuencias con la variable de la Ec.\ref{ra_arb} para este rango de energía se observa en la Fig.\ref{fig:tercer_barrido}. La línea horizontal indica el valor de $r_{99}$ para cada frecuencia y se observa que ninguna frecuencia supera dicho umbral. En la frecuencia solar no se observa ningún pico, esto se debe a que el método EW es robusto con respecto a las modulación del clima. Se observa un pico en sidérea pero el mismo no es significativo con respecto al $r_{99}$.


    \begin{figure}[H]
        \begin{small}
            \begin{center}
                \includegraphics[width=0.75\textwidth]{plot_bin_3_barrido_v3_EW.pdf}
            \end{center}
            \caption{Barrido de frecuencias en el rango 1 EeV - 2 EeV .}
            \label{fig:tercer_barrido}
        \end{small}
    \end{figure}    

    \section{Gráficos}

    Para poder comparar los resultados de $d_\perp$ entre sí, podríamos graficar los valores de la proyección y de la límite del $99\%$ como se muestra en la Fig.\ref{fig:no_normalizado}. El inconveniente es la cantidad de datos en cada rango de energía entre los conjuntos de datos, todos los disparos y disparo estándar, son distintos.



    \begin{figure}[H]
        \begin{small}
            \begin{center}
                \includegraphics[width=0.75\textwidth]{d_perp_no_normalizado_v2.pdf}
            \end{center}
            \caption{Sin normalizar}
            \label{fig:no_normalizado}
        \end{small}
    \end{figure}
    
    Para compararlos mejor con respecto a $d_{\perp,UL}$, usamos el valor de cada rango y de cada conjunto de datos, para normalizar la amplitud de $d_{\perp,UL}$. Como se muestra en la Fig.\ref{fig:normalizado}, ahora $d_{\perp,UL}=1$ y los otros valores se pueden comparar. 

    \begin{figure}[H]
        \begin{small}
            \begin{center}
                \includegraphics[width=0.75\textwidth]{d_perp_normalizado.pdf}
            \end{center}
            \caption{Valores normalizados con $d_{\perp,UL}$}
            \label{fig:normalizado}
        \end{small}
    \end{figure}

    También podemos comparar cuan apartados están con respecto al valor $\sigma_{x,y}$ y normalizar los valores en cada rango de energía, así se obtiene la Fig.\ref{fig:normalizado_sigma}.

    \begin{figure}[H]
        \begin{small}
            \begin{center}
                \includegraphics[width=0.75\textwidth]{d_perp_normalizado_sigmas_v4.pdf}
            \end{center}
            \caption{Valores normalizados con $d_{\perp,UL}$}
            \label{fig:normalizado_sigma}
        \end{small}
    \end{figure}

Por lo que ahora podemos decir que en los rangos entre 0.5 EeV - 1.0 EeV y 1.0 EeV - 2.0 EeV, la amplitud obtenida en este trabajo está por encima que la referencia. 

Para comparar los resultados en el  rango 0.25 EeV - 0.5 EeV, tenemos que tener en cuenta que el disparo estándar tiene una sensibilidad menor que el todos los disparos. Esto se ve claramente en la Tabla \ref{tab:}, donde el primer tiene 7 veces menos eventos para analizar. Por lo tanto, la discrepancia entre la referencia y los trabajos puede deberse a la  diferencia de eventos a estudiar causada por la sensibilidad del disparo.


Considerando los valores de $\sigma$ y $d_\perp$ para cada rango de energía, puedo comparar las direcciones, valores e incertidumbres en un sola figura como en la Fig.\ref{fig:incertidumbre}. Las líneas punteadas están centradas en los valores de referencia en cada rango de energía y con radio igual a sus incertidumbres. 

\begin{figure}[H]
    \begin{small}
        \begin{center}
            \includegraphics[width=0.75\textwidth]{comparando_sigmas_v2.pdf}
        \end{center}
        \caption{Amplitudes con incertidumbre, apuntando en la dirección  de la fase. Los círculos punteados los valores de referencia del trabajo \cite{Aab_2020} con sus respectivas incertidumbres y la línea punteada en negro marca la dirección del centro galáctico.}
        \label{fig:incertidumbre}
    \end{small}
\end{figure}


\section{Comparando resultados entre métodos para barridos de frecuencias}

\section{Verificación del código escrito durante la maestría}

Para ver que todo cierre, obtuve los resultados del paper \cite{Aab_2020} con el código del Rayleigh para distintos bines. En el bin  2 EeV - 4 EeV  tuve incongruencias entre mi código y los valores reportados en el paper, pero si comparo los valores obtenidos con el código utilizado para el paper con mis resultados si se corresponden. En los demás bines los resultados entre el código implementado en \cite{Aab_2020}, los resultados publicados y los resultados de mi código se corresponden.


En el bin 2 EeV - 4 EeV, verifiqué sin cambiaba los números considerando los eventos hasta $80^o$, pero los parámetros de Rayleigh eran los mismos que usar $60^o$  como límite en $\theta$.  Cuando no considero los pesos en mi código, obtengo resultados congruentes con los publicados pero eso puedo ser una casualidad.

\begin{table}[H]
    \begin{small}
        \begin{center}
            \begin{tabular}[c]{l|c|c|c|c|}
                                            & \multicolumn{4}{|c|}{2 EeV - 4 EeV}                                                               \\ \hline
                Frecuencia:                 & 366.25              & 366.25 (Sin pesos)  & 366.25 \cite{codigo}    & 366.25 \cite{Aab_2020}   \\ \hline
                Amplitud r [\%]:            & $0.5^{+0.3}_{-0.2}$ & $0.4^{+0.3}_{-0.2}$ & $0.5^{+0.3}_{-0.2}$     & -                          \\
                $r_{99}$ [\%]:              & 0.8                 & 0.8                 & 0.8                     & -                          \\\hline
                Amplitud $d_\perp$[\%]:     & $0.7^{+0.4}_{-0.2}$ & $0.5^{+0.4}_{-0.2}$ & $0.7^{+0.4}_{-0.2}$ 	  & $0.5^{+0.4}_{-0.2}$                    \\
                $d_{99}$ [\%]:              & 1.0                 & 1.0                 & 1.0                     & -                         \\
                $d_{\perp,UL}$[\%]:         & 1.9                 & 1.7                 & -                       & 1.4                               \\\hline
                $\sigma_{x,y}$[\%]:         & 0.34	              & 0.34	            & 0.34	                  & 0.34                           \\
                Probabilidad      :         & 0.14                & 0.33                & 0.15               	  & 0.34                       \\
                Fase[$^o$]:                 & 355$\pm$29          & 351$\pm$38          & 346$\pm$29              & 349$\pm$55                    \\
            \end{tabular}
        \end{center}
    \end{small}
    \caption{Características para las frecuencias solar y sidérea con el método Rayleigh en el primer armónico en el rango de energía 2 EeV - 4 EeV, obtenidos con el código de este trabajo aplicado al conjunto de datos de la referencia \cite{Aab_2020} y comparados con los resultados reportados en el último.}
\end{table}


\begin{table}[H]
    \begin{small}
        \begin{center}
            \begin{tabular}[c]{l|c|c||c|c|}
                                            & \multicolumn{2}{|c||}{8 EeV - 16 EeV}              & \multicolumn{2}{|c|}{16 EeV - 32 EeV}                   \\ \hline
                Frecuencia:                 & 366.25                    & 366.25 \cite{Aab_2020} & 366.25                   & 366.25 \cite{Aab_2020}   \\ \hline
                Amplitud r [\%]:            & $4.4^{+1.0}_{-0.8}$ 	    & -                      & $5.8^{+1.8}_{-1.3}$ 	    & -                         \\
                $r_{99}$ [\%]:              & 2.6                       & -                      & 4.9                      & -                          \\\hline
                Amplitud $d_\perp$[\%]:     & $5.6^{+1.2}_{-1.0}$ 	    & $5.6^{+1.2}_{-1.0}$    & $7.5^{+2.3}_{-1.8}$ 	    & $7.5^{+2.3}_{-1.8}$                   \\
                $d_{99}$ [\%]:              & 3.3                       & -                      & 6.3                      & -                         \\
                $d_{\perp,UL}$[\%]:         & 10                        & -                      & 16                       & -                                 \\\hline
                $\sigma_{x,y}$[\%]:         & 1.1	                    & 1.1                    & 2.1	                    & 2.1                           \\
                Probabilidad      :         & $2.3\times10^{-6}$	    & $2.3\times10^{-6}$     & $1.5\times10^{-3}$	    & $1.5\times10^{-3}$              \\
                Fase[$^o$]:                 & 96$\pm$11                 & 97$\pm$12              & 80$\pm$16                & 80$\pm$17                     \\
            \end{tabular}
        \end{center}
    \end{small}
    \caption{Características para las frecuencias solar y sidérea con el método Rayleigh en el primer armónico en distintos rangos de energía, obtenidos con el código de este trabajo aplicado al conjunto de datos de la referencia \cite{Aab_2020} y comparados con los resultados reportados en el último.}
\end{table}


\section{Comparando amplitud en función de la frecuencia}

En las Figs.\ref{fig:primer_barrido_EW_Ray}, \ref{fig:segundo_barrido_EW_Ray} y \ref{fig:tercer_barrido_EW_Ray} se comparan el barrido en frecuencia con el método EW y el barrido con Rayleigh considerando los pesos de los hexágonos en distintos rangos de energía.


\begin{figure}[H]
    \begin{small}
        \begin{center}
            \includegraphics[width=0.4955\textwidth]{plot_bin_1_barrido_v3_EW.pdf}
            \includegraphics[width=0.4955\textwidth]{plot_bin_1_barrido_v1_Ray.pdf}
        \end{center}
        \caption{Barrido de frecuencias en el rango 0.25 EeV - 0.5 EeV .}
        \label{fig:primer_barrido_EW_Ray}
    \end{small}
\end{figure}    

\begin{figure}[H]
    \begin{small}
        \begin{center}
            \includegraphics[width=0.4955\textwidth]{plot_bin_2_barrido_v3_EW.pdf}
            \includegraphics[width=0.4955\textwidth]{plot_bin_2_barrido_v1_Ray.pdf}
        \end{center}
        \caption{Barrido de frecuencias en el rango 0.5 EeV - 1 EeV .}
        \label{fig:segundo_barrido_EW_Ray}
    \end{small}
\end{figure}    

\begin{figure}[H]
    \begin{small}
        \begin{center}
            \includegraphics[width=0.4955\textwidth]{plot_bin_3_barrido_v3_EW.pdf}
            \includegraphics[width=0.4955\textwidth]{plot_bin_3_barrido_v1_Ray.pdf}
        \end{center}
        \caption{Barrido de frecuencias en el rango 1 EeV - 2 EeV .}
        \label{fig:tercer_barrido_EW_Ray}
    \end{small}
\end{figure}    

