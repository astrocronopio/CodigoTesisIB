%%%%%%%%%%%%%%%%%%%%%%%%%%%%%%%%%%%%%%%%%%%%%%%%%%%%%%%%%%%%%%%%%%%%%%%%%%%%%%%%
%\documentclass[12pt,papel,twoside]{ibtesis}
% \documentclass[12pt]{ibtesis}

\documentclass[12pt,papel,oneside]{ibtesis}
% \documentclass[11pt,papel,oneside, singlespace]{ibtesis}

% \documentclass[12pt,papel,preprint,singlespace,oneside]{ibtesis}

%%%%%%%%%%%%%%%%%%%%% Paquetes extra %%%%%%%%%%%%%%%%%%%%%%%%%%%%%%%%%%%%%%%%%%%
% Por conveniencia: aqu\'{\i} puede cargar todos los paquetes y definir los comandos 
% que necesite
\usepackage{ibextra}
\usepackage[utf8]{inputenc}
\usepackage{subcaption}  % Enable figure captions or figure notes
\usepackage{float}
\usepackage{nicefrac}
\usepackage{mathtools}
\usepackage{textcomp}

\usepackage{amsfonts}

\newcommand{\done}{\item[\checkmark]}
%%%%%%%%%%%%%%%%%%%%%%%%%%%%%%%%%%%%%%%%%%%%%%%%%%%%%%%%%%%%%%%%%%%%%%%%%%%%%%%%
%%%%%%%%%%%%%%%%%%%%% Informacion sobre la tesis %%%%%%%%%%%%%%%%%%%%%%%%%%%%%%%
\title{Análisis de las direcciones de arribo de rayos cósmicos de ultra-alta energía en el Observatorio Pierre Auger}
\author{Evelyn~G.~Coronel}
\director{Dra.~Silvia Mollerach}
%\codirector{Dr.~J.~Otro m\'{a}s}b
\carrera{Tesis de Maestría en Ciencias F\'{\i}sicas}
\grado{Maestrando}
\laboratorio{Partículas y Campos -- Centro At\'{o}mico Bariloche}
\jurado{Dr.~Diego~Harari (Instituto Balseiro)}

\palabrasclave{Rayos Cósmicos, Análisis de datos, Instituto Balseiro}
%\keywords{Cosmic Rays, Data Analysis, Balseiro Institute}
%\neembaeguasu{Mba'e michĩ yvágagui ouva, Mbo'ehaoguasu Balseiro}
% Si queremos poner la fecha manualmente:
% \date{Diciembre de 2099}

%%%%%%%%%%%%%%%%%%%%%%%%%%%%%%%%%%%%%%%%%%%%%%%%%%%%%%%%%%%%%%%%%%%%%%%%%%%%%%%%
%\titlepagefalse % Si no quiere compilar la portada descomente esta linea
%\includeonly{apendices} % Compilar s\'{o}lo estos archivos 
%\graphicspath{{/h}} % Lugar donde encontrar las figuras generales (se puede poner uno en cada cap{\'{\i}}tulo)
%%%%%%%%%%%%%%%%%%%%%%%%%%%%%%%%%%%%%%%%%%%%%%%%%%%%%%%%%%%%%%%%%%%%%%%%%%%%%%%%


%\setcounter{tocdepth}{4}
%\setcounter{secnumdepth}{4}
\begin{document}

\begin{preliminary}

%%% \'{I}ndices %%%%

\begin{abreviaturas}

\begin{tabular}{l l}
CR: 		& Rayos cósmicos  (\emph{Cosmic Rays}) \\
CMB: 		& Radiación Cósmica de Fondo (\emph{Cosmic Microwave Background})\\
FD: 		& Detector de Fluorescencia (\emph{Fluorescence Detector}) \\
SD: 		& Detector de Superficie (\emph{Surface Detector})  \\
WCD: 		& Detector de radiación Cherenkov de agua\\
EAS: 		& Lluvia Atmosférica Extendida  (\emph{Extensive Air Shower})    \\
VAOD: 		& Profundidad atmosférica óptica vertical (\emph{Vertical Atmosferic Optical Depth})\\
CLF:		& \emph{Central Laser Facility}\\
XLF:		& \emph{eXtreme Laser Facility}\\
X$_{max}$: 	& Profundidad atmosférica del máximo de la lluvia \\
LDF: 		& Función de Distribución Lateral (\emph{Lateral Distribution Function}) \\
S(1000): 	& Señal a 1000\,m del núcleo de la lluvia y al nivel del suelo \\
S(1000)$_w$:& Señal de S(1000) corregida por la modulación del clima. \\
CIC: 		& Corte de Intensidad Constante (\emph{Constant Intensity Cut}) \\
S$_{38}$: 	& Señal a 1000\,m del núcleo y al nivel del suelo si el ángulo cenital del evento fuera de $38^o$\\
S$_{38,w}$: & Señal S$_{38}$ corregida por la modulación del clima \\
eV: 		& electrón Voltio, $1\,$eV$= 1.602\times 10^{-19}\,$J \\
EeV: 		& $1\,$EeV$=10^{18}\,$eV\\
PMT: 		& Tubo fotomultiplicador (\emph{Photo-Multiplier Tube})\\
VEM: 		& Muón vertical equivalente (\emph{Vertical Equivalent Muon})\\
ICRC: 		& Conferencia Internacional de Rayos Cósmicos (\emph{International Cosmic Ray Conference})\\
\end{tabular}
                     %Abreviaturas
\end{abreviaturas}

	\tableofcontents                %\'{I}ndice
	\listoffigures                  %Figuras
	%\listoftables                   %Tablas

	\begin{resumen}%
Cuando un rayo cósmico interactúa con una molécula en la parte superior de la atmósfera, se inicia un proceso en el cual se generan otras partículas secundarias. Este proceso es conocido como lluvia atmosférica extendida. Estas lluvias pueden ser detectadas sobre la superficie de la Tierra mediante varios experimentos. Este trabajo utiliza los datos recolectados por los detectores de superficie separados en 1500\,m entre sí del Observatorio Pierre Auger durante los años 2005-2020. 

Se estudian eventos obtenidos mediante distintos algoritmos de adquisición de datos. El \emph{Disparo Estándar} que alcanza eficiencia completa para eventos asociados a rayos cósmicos de energía mayor a $3\,$EeV, y el \emph{Todos los Disparos} llega a detectar, con una eficiencia del 100\%, eventos por encima de $1\,$EeV. El primer disparo contiene eventos registrados desde el año 2005 y el segundo disparo empezó funcionar desde el 2013. 

Las condiciones atmosféricas como la presión (P), la temperatura (T) y la densidad ($\rho \propto \nicefrac{P}{T}$) afectan el desarrollo de la lluvia a través de la atmósfera. Las variaciones de estas condiciones inducen una modulación en la señal producida en los detectores por un rayo cósmico de una dada energía. Mediante un estudio hecho por la Colaboración sobre eventos del Disparo Estándar, se corrigió el efecto de esta modulación en la estimación de la energía de los rayos cósmicos medidos por el Observatorio. En este trabajo extendimos el periodo de tiempo analizado de esta modulación, y se observó que los parámetros obtenidos son comparables con la reconstrucción oficial. También se estudia la modulación en los datos de Todos los Disparos, y se realiza una corrección sobre el mismo conjunto de datos usando los parámetros obtenidos por este trabajo.

Se  estudian las modulaciones en distintas frecuencias mediante el análisis en Rayleigh, y se propone una variable generalizada para hacer un barrido en frecuencias con el método de East-West. Se obtienen resultados de la modulación en ascensión recta para distintos rangos de energía y se comparan con resultados reportados por la Colaboración Pierre Auger.

\end{resumen}


% \begin{nemombyky}%
% Mbyjakua\'ape (\emph{astronomía} karaiñe'\~eme) ojeikuaase mba\textquotesingle e oik\'ova umi mba\textquotesingle e  michĩ yv\'agagui o\'uva (\emph{rayos cósmicos} karaiñe'\~eme) oguah\~evove amo yvatetépe (\emph{atmósfera} karaiñe'\~eme). Ombok\'aramo tuminguaave\textquotesingle \~yty (\emph{conjunto de átomos o molécula}  karaiñe'\~eme ) yvatetépe oĩva, oñepyr\~u ojapo het\~a umi tuminguaave\textquotesingle \~yjokaku\'era (\emph{partículas}  karaiñe'\~eme ) op\'arupi. Ko\textquotesingle a       ha\textquotesingle e  h\'ina peteĩ ama guasu tuminguaave\textquotesingle \~yjoka rehegua ( \emph{lluvia atmosf\'erica extendida} karaiñe'\~eme). Umi ama guasuku\'era tuichaterei ha ikatu eñeña\textquotesingle ã yvy ári op\'arupi. Mend\'osape oĩ peteĩ mba\textquotesingle etuicha h\'erava \emph{Pierre Auger} Mbyjañama\textquotesingle \~eha\~gua (\emph{Observatorio Pierre Auger}) oña'\~ava ko ama. Ko\textquotesingle  ape romba\textquotesingle  ap\'ota umi ama ko mbyjañama\textquotesingle \~eha\~gua oña\textquotesingle \~ava\textquotesingle  kue 2005-guive 2018-peve. Mba\textquotesingle \'eichapa umi amaku\'era oguah\~e yvy \'ari ikatu ojuavy hakúramo (T, \emph{temperatura} karaiñe\textquotesingle \~eme) tér\~a  poh\'yiramo pe pytundyry mbyjañama\textquotesingle \~eha\~gua áripe ($\rho$, \emph{densidad} h\'erava karaiñe\textquotesingle \~eme). Ko mbyjañama'\~eha\~gua ojapova\textquotesingle ekue peteĩ tembiapo ha ko\textquotesingle ape rojapojey up\'eva roikuaaha\~gua umi papapo oñenoh\~eva\textquotesingle ekue oiko gueteri ko'\~anga peve, ha rotopa kóva oikópa añetete.
% \end{nemombyky}



%%% Local Variables: 
%%% mode: latex
%%% TeX-master: "template"
%%% End: 


\end{preliminary}

%Podemos usar cualquiera de los dos comandos: \input o \include para incluir el texto
	\graphicspath{{../IntroduccionBachelor/}}
	

\chapterquote{We can only measure what Nature sends us}{Jim Cronin}  

Desde el descubrimiento de los rayos cósmicos en 1911 por Victor Hess, numerosos experimentos han intentado caracterizarlos. A partir del 2004, el Observatorio Pierre Auger ha detectado rayos cósmicos con el objetivo de estudiar su origen. Un análisis adecuado de los eventos registrados es necesario para estudiar las posibles fuentes de rayos cósmicos, además de su composición y su espectro de energía.

Un aspecto estudiado por varios trabajos \cite{collaboration2013pierre} \cite{data} es la distribución  de las direcciones de arribo de los rayos cósmicos. Las direcciones de arribo son prácticamente isotrópicas salvo variaciones muy pequeñas alrededor de la media. Dado que estas  anisotropías son pequeñas respecto a la media, es importante tener en cuenta todos los efectos que pueden ser fuentes de modulación de los datos. Un ejemplo claro de una modulación física que no aporta información sobre las anisotropías es la modulación del clima.

Este trabajo es parte del análisis de la direcciones de arribo de los rayos cósmicos de ultra alta energía obtenidas por el Observatorio Pierre Auger. En el mismo se estudia la modulación del clima sobre la determinación de la energía de los eventos medidos por los detectores de superficie. Las lluvias atmosféricas provocadas por los rayos cósmicos que llegan a la alta atmósfera, interactúan con los constituyentes de las atmósfera. Esta interacción puede ser afectada por los cambios en las condiciones atmosféricas en el momento de la lluvia. El trabajo está dividido en distintos capítulos organizados para introducir las rayos cósmicos, mencionar brevemente características del Observatorio Pierre Auger y presentar el análisis sobre la modulación del clima de la señal medida por el Observatorio.

\section{Rayos cósmicos}

Los rayos cósmicos (CRs) fueron descubiertos en el 1911 por Victor Hess \cite{hess1912}. Los mismos son partículas que llegan a la Tierra desde el espacio como  electrones, positrones, rayos gamma entre otros, además de núcleos atómicos. En 1962, John Linsley detectó un evento de CR con una energía de  $10^{20}\,$EeV, y otros experimentos  encontraron más eventos por encima de esta energía. A pesar de que han sido medidos y estudiados en experimentos alrededor del mundo, el origen de los CRs es incierto. Las partículas con energía por encima de $10^{18}\,$EeV se conocen como rayos cósmicos de ultra alta energía (UHECRs) y son las partículas con más energía presentes en el universo. Las direcciones de arribo de los UHECRs son casi isotrópicas  \cite{collaboration2013pierre} \cite{data} y se cree que son de origen extra-galáctico, es decir que no fueron producidos dentro de la Vía Láctea, debido a que los campos magnéticos galácticos no pueden confinarlos y la distribución de sus direcciones de arribo es cerca a ser isotrópica, sin correlación significativa con el plano o el centro galáctico. Para estudiar a los mismos, se disponen de tres observables principales: el espectro, la composición y la anisotropía. El espectro se refiere a la distribución de energía de los rayos cósmicos detectados, la composición es la distribución de masas nucleares, es decir, que elementos y en que proporción se encuentran en los rayos cósmicos y el tercero, la anisotropía, es la distribución de las direcciones de arribo a diferentes energías.

\section{Espectro de energías}

Los mecanismos de interacción de protones y núcleos de origen extra-galáctico  y su relevancia en la propagación fueron predichos por Greisen \cite{greisen1966end}, e independientemente por Zatsepin y Kuzmin \cite{zatsepin1966upper} tras el descubrimiento de la radiación cósmica de fondo (CMB). Primeramente todas las partículas sufren una pérdida de energía debido a la expansión del universo. Este el principal mecanismo de pérdida de energía para protones de $E < 2\times 10^{18}\,$eV y núcleos de $E/A < 0.5\times 10^{18}\,$eV. 

En la Fig.\,\ref{fig:spectra} se presenta el espectro de los rayos cósmicos medidos por los distintos experimentos que se desarrollaron para su estudio. La figura fue extraída de \cite{PGD}, donde los datos fueron multiplicado por $E^{2.6}$ para resaltar los cambios en la forma del espectro. Considerando que el espectro de energías por debajo de $\sim 0.1\times 10^{18}\,$eV es de origen galáctico, la rodilla correspondiente al cambio de pendiente en $\sim 3\times10^{15}\,$EeV podría reflejar el hecho que la mayoría de los aceleradores en la galaxia han alcanzado su energía máxima para la aceleración de protones. El experimento de Kascade-Grande ha reportado una segunda rodilla cercana a $8\times10^{16}\,$eV, que podría corresponder al límite de aceleración de primarios más pesados \cite{PGD}.

Considerando el tobillo en la Fig\,\ref{fig:spectra}, es posible que sea el resultado de que una población de mayor energía esté sobrepasando a una población de menor energía, por ejemplo un flujo extra-galáctico empiece a dominar sobre un flujo galáctico \cite{bird1994cosmic}. Otra posibilidad es que el cambio de la forma de la curva se deba a la pérdida de energía de los protones extra-galácticos, debido por el proceso $p\,\gamma \rightarrow\,e^+\,+\,e^-$, conocido como foto-desintegración con el CMB \cite{berezinsky2006astrophysical}. Para energías aun mayores ($\nicefrac{E}{A} \geq 60\times 10^{18}\,$eV ) el proceso dominante es la producción de mesones por colisiones entre núcleos y fotones de muy altas energías. 

El flujo de los rayos cósmicos en función de la energía puede aproximar una ley de potencias que tiene una forma del siguiente tipo
\begin{equation}
	    \frac{d\Phi_{CR}}{dE} \propto \ E^{-\gamma}   \label{eq:expresion1}
\end{equation}
donde $\gamma$ se lo denomina índice espectral, este valor varía ligeramente para distintos rangos de energía. %Para este trabajo se toma un valor de $\gamma = 3.29$ \cite{como_funciona_auger}. Este valor es un promedio de los distintos valores del índice espectral para UHCRs.

%\section{Desarrollo del primario desde su fuente hasta la cascada}

\section{Lluvias atmosféricas extendidas}

Una lluvia atmosférica extendida (EAS) es la cascada de partículas secundarias generadas por la interacción de un rayo cósmico, conocido como partícula primaria o el primario, con la atmósfera terrestre. Como se observa en la Fig.\,\ref{fig:spectra} el flujo de partículas decae rápidamente con la energía. Aunque para energías mayores a $10^{14}\,$EeV las partículas producidas en la atmósfera como secundarios pueden llegar a las montañas. Para energías mayores pueden llegar hasta el nivel del mar. El momento transversal que adquieren las partículas secundarias en el proceso de dispersión a través de la atmósfera es tal que los secundarios se dispersan sobre área de gran tamaño. Para energía mayores a 10$\,$EeV, por ejemplo, la lluvia puede llegar a cubrir más de 25\,km$^2$. 

El desarrollo de la lluvia puede describirse mediante la profundidad atmosférica, definida como la masa de aire por unidad de área que atravesó una partícula en su dirección de propagación, 
\begin{equation}
	X(L)= \int_L^\infty dx \rho(x)
\end{equation}
donde $\rho$ es la densidad del aire en función de la posición.


\section{Descripción de una anisotropía dipolar}
Las anisotropías en las direcciones de llegada de los RCs indican que ciertas zonas del cielo tienen una variación significativa con respecto a la media de flujo de RCs. Estas anisotropías pueden describirse mediante una superposición de funciones armónicas. El primer orden corresponde a una anisotropía dipolar.

\begin{figure}[H]
	\centering
	\includegraphics[width=0.7\textwidth]{auger_spectrum_v2.png}
	\caption{Espectro de rayos cósmicos medidos mediante lluvias atmosféricas en función de la energía $E$. Figura extraída de \cite{PGD}}
	\label{fig:spectra}
\end{figure}


Una anisotropía dipolar se puede describir de la siguiente forma:
\begin{equation}
    \Phi(\hat{\bf{u}}) = \Phi_0(1+\bf{d}\cdot\hat{\bf{u}})
    \label{eq:dipolo_general}
\end{equation}
\noindent donde $\Phi_0$ es el flujo medio de eventos, $\hat{\bf{u}}$ es un versor que apunta a la dirección a estudiar, y $\bf{d}$ es un vector con módulo igual a la amplitud del dipolo y cuya dirección está apuntando al máximo del flujo. Tomando coordenadas ecuatoriales \footnote{El sistema de coordenadas ecuatoriales se desarrolla en el apéndice \ref{apendice:ecuatorial}}, la dirección de $\bf{d}$ es $(\alpha_d, \delta_d$) y de $\hat{\bf{u}}$ es $(\alpha, \delta)$, entonces  el producto escalar  entre estos vectores se puede escribir de la siguiente manera:
\begin{equation}
    \textbf{d}\cdot\hat{\bf{u}}= d (\cos\delta_d \cos\delta \cos(\alpha - \alpha_d) + \sin\delta_d  \sin\delta)
    \label{eq:product_ud}
\end{equation}
El desarrollo para obtener esta expresión se encuentra en el apéndice \ref{cambio_coord}.

Otro aspecto importante de la representación del dipolo en coordenadas ecuatoriales, es que la proyección de la amplitud del dipolo sobre el plano ecuatorial $d_\perp$ se puede aproximar de la siguiente manera \cite{taborda} :
\begin{equation}
    d_\perp \simeq \frac{r_1}{ \langle \cos\delta \rangle}
    \label{eq:fourier_perp}
\end{equation}
donde $r_1$ es la amplitud del primer armónico en ascensión recta, y $\langle \cos\delta \rangle$ es el valor medio de $\cos\delta $ de los eventos.

\subsection{Representación en coordenadas locales de la anisotropía dipolar}

Podemos reescribir el producto escalar entre el dipolo $\textbf{d}$ y el versor $\hat{u}$ que apunta en una dirección cualquiera mediante las coordenadas locales $\theta$ y $\phi$\footnote{El sistema de coordenadas locales se desarrolla en el apéndice \ref{apendice:local}.} como se muestra en la siguiente expresión: 
\begin{align}
    \textbf{d} &=  d_{x'}(\alpha^0, \delta^0)\hat{x}' +  d_{y'}(\alpha^0, \delta^0)\hat{y}'+ d_{z'}(\alpha^0, \delta^0)\hat{z}' \\
    \hat{\bf{u}} &=\sin\theta \cos\phi \hat{x}' + \sin\theta \sin\phi \hat{y}' + \cos\theta\hat{z}'\\
    \textbf{d}\cdot\hat{\bf{u}} &= d_{x'}(\alpha^0, \delta^0)\sin\theta \cos\phi
    + d_{y'}(\alpha^0, \delta^0) \sin\theta \sin\phi  
     + d_{z'}(\alpha^0, \delta^0)\cos\theta \label{eq:dot-prod-local}
\end{align}
donde los versores $\hat{x}'$, $\hat{y}'$ y $\hat{z}'$ apuntan a la dirección Este, Norte y del cenit respectivamente. 

El dipolo $\textbf{d}$ está fijo en el cielo pero visto desde las coordenadas locales, para poder trabajar con $\theta$ y $\phi$, sus proyecciones  $d_{x'}$, $d_{y'}$ y $d_{z'}$ tienen una dependencia con la ascensión recta  $\alpha^0$ y declinación $\delta^0$ del cenit. 

	
	\graphicspath{{../IntroduccionAuger/}}
	\chapter{El Observatorio Pierre Auger}

\section{Introducción}

Para realizar un estudio con mucha estadística de los CRs hasta altas energías se diseñó el Observatorio Pierre Auger. Las propiedades medidas de los lluvias extendidas determinan la energía y la dirección de arribo de cada CR, además de proveer información sobre la distribución de la composición del CR. El Observatorio Pierre Auger en la Provincia de Mendoza, Argentina ha registrado eventos desde el año 2004 mientras se agregaban detectores hasta su terminación en el 2008.

\section{Detección de Rayos Cósmicos}

Una característica esencial del Observatorio es la capacidad de observar lluvias atmosférica extendidas (EAS) simultáneamente mediante dos técnicas distintas, combinando los detectores de superficie (SD) y los detectores de fluorescencia (FD). Los SD son un conjunto de 1660  detectores Cherenkov con agua hiper-pura colocados en un arreglo triangular, con una distancia de $1.5\,$km cubriendo $\sim3000\,$km$^2$, además de un arreglo más pequeño llamado \emph{Infill} separados por $750\,$m. El arreglo principal son los detectores de superficie distanciados 1500\,m, que en el presente trabajo se  referencia como \emph{SD 1500\,m} se muestra en la Fig.\,\ref{fig:auger_sd}. Los FD están colocados en cuatro edificios alrededor del arreglo de SD: Coihueco, Loma Amarilla, Los Morados y Los Leones indicados en el mapa en la Fig.\,\ref{fig:auger_sd}. Cada edificio contiene 6 FD, donde cada uno tiene un campo de visión de $30^o\times30^o$, cubriendo así cada uno $180^o$ en la horizontal.

El área del observatorio es generalmente plana, la altitud de los detectores varía entre $1340\,$m y $1610\,$m, con una altitud media de $\sim1400\,$m. Estos detectores están distribuidos entre las latitudes $35.0^o$ S y $35.3^o$ S y entre las longitudes $69.0^o$ W y $69.4^o$ W.


\subsection{ El detector de superficie y el detector de Fluorescencia}

Un SD consiste en un tanque de polietileno de $3.6\,$m de diámetro que contiene $12\,000$ litros de agua agua hiper-pura. Su interior está recubierto por una lámina de alta reflectividad. En la parte superior se encuentran tres foto-multiplicadores (PMT) distribuidos simétricamente  a $1.2\,$m respecto al centro del tanque. Los mismos colectan la radiación Cherenkov producida por una partícula cargada relativista que pasa por el agua del detector. La altura del tanque de $1.2\,$m lo hace sensible a fotones de altas energías, que pueden convertirse en pares electrón-positrón en el volumen de agua \cite{como_funciona_auger}.

El detector de fluorescencia (FD) consiste en 24 telescopios de fluorescencia, esquematizados en la Fig\,\ref{fig:FD}, distribuidos en 4 distintos lugares en los límite del observatorio. %Estos 4 sitios son: Los Leones, Loma Amarilla, Los Morados y Coihueco, donde 6 de estos telescopios están instalados. 
Cada telescopio tiene un espejo esférico segmentado de 13$\,m^2$ y una cámara que consiste en 440 PMTs ordenados en una grilla de 22x20. Cada telescopio tiene un campo de visión de $30^o\times30^o$.%, por lo tanto cada edificio cubre $180^o$ en acimut.

\begin{figure}[H]
	\centering
	\includegraphics[width=0.45\textwidth]{../auger/auger_sd.png}
	\caption{Distribución de los tanques del SD en el área del Observatorio Pierre Auger. Se muestra la ubicación de las estaciones del clima, otros módulos instalados sobre el observatorio y la posición de los detectores de fluorescencia (FD). Figura extraída de \cite{como_funciona_auger}}
	\label{fig:auger_sd}
\end{figure}

\begin{figure}[H]
    \begin{subfigure}[t]{0.45\textwidth}
	\includegraphics[width=\textwidth]{../auger/tanque.png}
	\caption{Detector de radiación Cherenkov con los elementos de la actualización para \emph{Auger Prime}} 	\label{fig:tanque}
    \end{subfigure}%
    \hspace{\fill}
    \begin{subfigure}[t]{0.5\textwidth}
	\includegraphics[width=\textwidth]{../auger/fd.png}
	\caption{Esquema simplificado de un telescopio de fluorescencia. Extraído de \cite{kit_oracle}}
	\label{fig:FD}
    \end{subfigure}%
    \caption{Detectores empleados por el Observatorio Pierre Auger para la detección de rayos cósmicos.}
	\end{figure}

El FD mide los fotones ultravioletas producidos por la componente electromagnética de la EAS. Mientras se produce la lluvia en la atmósfera, algunos átomos de nitrógeno se excitan y se desexcitan emitiendo fotones. El uso del FD para detectar estos fotones es solo posible en noches sin nubes y sin luna. La posible atenuación de los fotones en la atmósfera es tenida en cuenta para la estimación de energía. Ya que esta estimación se basa en la cantidad de fotones detectados. Otro factor a tener en cuenta es la presencia de aerosoles, como humo o polvo, esto se realiza midiendo la profundidad atmosférica óptica vertical \emph{Vertical Atmosferic Optical Depth (VAOD)}. Estas mediciones son realizadas por los láseres de las instalaciones de Central Laser Facility (CLF) y de eXtreme Laser Facility (XLF), cuyas ubicaciones se muestra en la Fig.\ref{fig:auger_sd}.

\subsection{Diseño híbrido}\label{seccion:sd_eff}

El SD detecta un corte de EAS que llega al nivel del suelo, los WCDs detectan la componentes electromagnética y muónica de la lluvia. Cabe resaltar que el SD funciona las 24 horas del día, por lo que detecta una mayor cantidad de eventos que el FD. Existen métodos para determinar la dirección de arribo y la energía del primario.  La exposición se calcula contando la cantidad de hexágonos activos en un tiempo dado, y multiplicado la apertura de una sola celda hexagonal que vale $4.59\,$km$^2$.sr para lluvias verticales. El SD tiene la propiedad de que la calidad de sus mediciones aumenta con la energía del EAS. La exposición instantánea del SD se calcula fácilmente, especialmente para energías mayores a 3 EeV, donde la EAS detectada por cualquier parte del SD es detectada con 100\% de eficiencia independientemente de la masa del primario que inicio la EAS.

El FD es usado para generar una imagen del desarrollo del EAS en la atmósfera. La luz de fluorescencia es emitida isotrópicamente en la parte ultravioleta del espectro, y es producida predominantemente por la componente electromagnética de la lluvia. Los períodos de observación están limitados a las noches sin luna y con buen clima, pero la ventaja del FD es la posibilidad de ver el desarrollo de la lluvia. Dado que la producción de la fotones por fotoluminiscencia es proporcional a la energía depositada en la atmósfera, se puede medir la energía del primario mediante calorimetría. Otro aspecto importante del FD es la posibilidad de medir la profundidad de la atmósfera donde la lluvia alcanza su máximo desarrollo, $X_{max}$, esta cantidad es uno de los más directos indicadores de la composición de masa. \cite{data}

\section{Reconstrucción de eventos de los detectores  de superficie}

\subsection{Selección de eventos}

La reconstrucción de la energía y la dirección de arribo de los CRs se realiza mediante las señales medidas por el SD. La dirección es reconstruida mediante  el tiempo de llegada de las señales registradas por estaciones individuales del SD. Para garantizar la selección de eventos bien contenidos en el SD, se aplica el corte llamado \emph{6T5}. Este corte considera solo a los eventos donde el tanque con mayor señal está rodeado por otros 6 tanques activos. Esta condición asegura una buena reconstrucción de la energía. Al mismo tiempo, este corte simplifica el cálculo de la exposición \cite{exposure}, importante  para el análisis del espectro. Para estudios de dirección de arribo pueden utilizar cortes menos estrictos.

\subsection{Reconstrucción de las lluvias}

En una primera aproximación para la dirección de arribo de la lluvia se obtiene ajustando los tiempos de llegada de la señal en cada tanque. Para eventos con suficientes tanques disparados, estos tiempos de llegada pueden ser descritas como la evolución un frente de lluvia como una esfera que crece con la velocidad de la luz. Los puntos de impacto del EAS con el suelo son obtenidas mediante ajustes a las señales de los tanques. Este ajuste se realiza con un función de distribución lateral (LDF). La LDF también tiene en cuenta la probabilidad de que los tanques no sean disparados y que los tanques con mayor señal estén saturados.

Un ejemplo de la señal que deja un evento sobre el SD 1500 m se muestra en la Fig.\,\ref{fig:evento_sd}. Este evento fue producido por un rayo cósmico de ($104\pm11$)\,EeV con un ángulo cenital de ($25.1\pm0.1 ^o$). La LDF de las señales para este evento se muestra en la Fig.\,\ref{fig:evento_S1000}. La función utilizada para el ajuste de la LDF es una función  $f_{LDF}$ propuesta por Nishimura-Kamata-Greisen \cite{data}
\begin{align*}
	%S(r) = S(r_{opt})\bigg(\frac{r}{r_{opt}}\bigg)^{\beta}\bigg(\frac{r+r_1}{r_{opt}+r_1}\bigg)^{\beta + \gamma}
	S(r) &= S(r_{opt})f_{LDF}(r)\\
	f_{LDF}(r)&=\bigg(\frac{r}{r_{opt}}\bigg)^{\beta}\bigg(\frac{r+r_1}{r_{opt}+r_1}\bigg)^{\beta + \gamma}
\end{align*}
donde $f_{LDF}$ está normalizado tal que $f_{LDF}(r_{opt})=1$ y $r_{opt}$ es la distancia óptima, %$r_1=700\,$m 
y $S(r_{opt})$ es usado para estimar la energía. Para el arreglo SD 1500\,m, el parámetro $r_{opt}=1000\,$m, por lo tanto el tamaño de la lluvia o \emph{shower size} es el valor de S(1000). Dado que la forma de la LDF es desconocida, la forma funcional propuesta para la función $f_{LDF}$ fue elegida empíricamente.  El parámetro $\beta$ depende del tamaño de la lluvia y del ángulo cenital. Los eventos verticales, es decir los eventos con $\theta < 60^o$, son medidas en una etapa menos desarrollada que eventos más inclinados. Los eventos con $\theta>60^o$ atraviesan un mayor cantidad de atmósfera.

\begin{figure}[H]
    \begin{subfigure}[t]{0.51\textwidth}
	\includegraphics[width=\textwidth]{../auger/evento_sd.png}
	\caption{Ejemplo de la señal dejada por un evento sobre el SD 1500 m. La flecha indica la dirección de arribo de la lluvia. Los colores de los círculo representa el tiempo de arribo de la lluvia, los primeros en amarillo y los últimos en rojo. En área de los círculo pintados es proporcional a logaritmo de la señal. Figura extraída de \cite{como_funciona_auger}. } 	\label{fig:evento_sd}
    \end{subfigure}%
    \hspace{\fill}
    \begin{subfigure}[t]{0.45\textwidth}
	\includegraphics[width=\textwidth]{../auger/evento_s1000.png}
	\caption{Dependencia de la señal con la distancia del núcleo de la lluvia. La función ajustada es la función de distribución lateral (LDF). Del ajuste se obtiene el valor de S(1000). Figura extraída de \cite{como_funciona_auger}. } 	\label{fig:evento_S1000}
    \end{subfigure}%
    \caption{Ejemplo de la reconstrucción de un evento de ($104\pm11$)\,EeV de energía con un ángulo cenital de ($25.1\pm0.1 ^o$).}
	\end{figure}


\subsection{Calibración de la energía}

Para una energía dada, el valor de S(1000) disminuye con $\theta$ debido a la atenuación de las partículas de la lluvia. Asumiendo un flujo isotrópico de los CR primarios sobre la parte superior de la atmósfera, se obtiene la atenuación de los datos mostrados en la Fig.\,\ref{fig:s1000_theta}  usando el método de Corte de Intensidad Constante (CIC) \cite{CIC}. La curva de atenuación $f_{CIC}(\theta)$ fue ajustado con un polinomio de orden 3 del tipo $f_{CIC}(\theta)=1+ax+bx^2+cx^3$, donde $x=\cos^2(\theta) - \cos^2(38^o)$. Según lo presentado por la colaboración \cite{collaboration2013pierre}, los valores son $a=0.980\pm0.004$, $b=-1.68\pm0.01$ y $c=-1.30\pm 0.45$, aunque estos coeficientes cambian ligeramente con la energía \cite{data}. El ángulo cenital $\theta=38^o$ se toma como un punto de referencia para convertir S(1000) a S$_{38}$ mediante $S_{38}=S(1000)/f_{CIC}(\theta)$. Este valor S$_{38}$ puede considerarse como la señal S(1000) que hubiera tenido un evento que fue detectado mediante el SD con $\theta=38^o$.

Los eventos con $\theta<60^o$  que fueron detectados por el SD y por el FD son utilizados para relacionar el tamaño de la lluvia con la energía  E$_{FD}$ medida por calorimetría por el FD.  La correlación entre S$_{38}$ y E$_{FD}$ se calcula mediante el método de máxima verosimilitud, que considera la evolución de las incertezas con la energía. La relación entre S$_{38}$ y $E_{FD}$ se describe mediante un función de potencia como se muestra en la Ec.\,\ref{eq:s38_energy}
\begin{equation}
	E_{FD}= A\, (S_{38}/VEM)^B
	\label{eq:s38_energy}
\end{equation}
donde los parámetros obtenidos son $A=(1.86\pm0.03)\times 10^{17}\,$eV y $B=(1.031\pm0.004)$  \cite{tobepublished}. En la Fig.\,\ref{fig:efd_s38} se observa el ajuste y la relación entre  S$_{38}$ y E$_{FD}$


\begin{figure}[H]
    \begin{subfigure}[t]{0.51\textwidth}
	\includegraphics[width=\textwidth]{../auger/s1000_theta.png}
	\caption{Curva de atenuación descrita por un polinomio de orden 3. En este ejemplo se deducen los coeficientes de la dependencia del S(1000) a S$_{38}\approx 50\,$VEM que corresponde a un energía de $10.5\,$EeV.} 	\label{fig:s1000_theta}
    \end{subfigure}%
    \hspace{\fill}
    \begin{subfigure}[t]{0.45\textwidth}
	\includegraphics[width=\textwidth]{../auger/efd_s38.png}
	\caption{Correlación entre el valor S$_{38}$ y la energía $E_{FD}$ medida por el FD.} 	\label{fig:efd_s38}
    \end{subfigure}%
    \caption{Distintas calibraciones hechas para los eventos reconstruidos en el Observatorio Pierre Auger.}
	\end{figure}



\subsection{Monitoreo del clima}\label{seccion:clima}

Las condiciones atmosféricas, como la temperatura, presión y humedad, se deben tener en cuenta para estudiar el desarrollo de los EAS, así como también para estudiar la cantidad de fotones de las lluvias sobre los moléculas de N$_2$, emitidos por fluorescencia. Distintas estaciones monitorean las condiciones atmosféricas sobre el Observatorio Pierre Auger, cuatro cerca  de los edificios donde se encuentran los FD y uno cerca del centro del SD 1500\,m. Para este trabajo se utilizaron las mediciones de la presión y temperatura registradas la mayor parte del tiempo en la estación del clima cerca del CLF, la misma realiza una medición cada intervalo de 5 minutos la mayor parte del tiempo. Cuando no se cuenta con datos registrados para intervalos entre 10 minutos hasta 3 horas, en estos casos se utiliza una interpolación de los datos medidos. Si el período de tiempo es mayor a 3 horas, los eventos durante este periodo no son considerados para la determinación de los efectos del clima en la señal detectada por el SD 1500\,m.

	\chapter{Introducción}
	\graphicspath{{../0_Introduccion/}}
	\input{../0_Introduccion/introduccion_v8}
	

 \chapter{Análisis de los efectos del clima sobre los datos del Observatorio Pierre Auger}
	\graphicspath{{../Clima/}}
	\input{../Clima/clima_1_intro.tex}
	\input{../Clima/clima_2_fit_main.tex}

\chapter{Anisotropías en las direcciones de arribo de los rayos cósmicos}
	\graphicspath{{../Anisotropia/}}
	\input{../Anisotropia/anisotropia_weather.tex}
	

	\chapter{Conclusiones}

%\begin{enumerate}
	%\item ¿Hizo alguna diferencia a la energía la corrección del clima?
	%\item ¿Hizo alguna diferencia la evolución del tiempo de los hexágonos a los parámetros del clima?
	%\item  ¿Disminuyó la modulación?
	%\item ¿El S38 sin corregir por clima me da los mismos resultados que lo anterior?
%	\item 
%\end{enumerate}


En este trabajo se analizaron los efectos de las variaciones de los parámetros del clima sobre el desarrollo en la atmósfera de las lluvias atmósferas. Se analizaron datos del arreglo de detectores espaciados 1500 m entre sí del Observatorio Pierre Auger, en el periodo 2005-2015 y 2005-2018 extendiendo los periodos de tiempo estudiados anteriormente. Se emuló los resultados de la corrección de la modulación del clima sobre el periodo 2005-2015 de la colaboración Pierre Auger, obteniéndose resultados compatibles. Se observó que posterior a la corrección, la modulación del clima se vio disminuida. Para eventos con energía mayor a $2\,$EeV, esta modulación es despreciable.

Posterior a los análisis anteriores, se estudió la modulación del clima mediante el valor del $S_{38}$ sin la corrección propuesta por trabajos anteriores. Se observó que los parámetros del clima obtenidos de estos datos son compatibles con los utilizados en la reconstrucción oficial. Se realizó un corrección a  la energía mediante los coeficientes nuevos, observándose que la modulación era despreciable para energías mayores a $2\,$EeV. 


%En este trabajo se estudió eventos con energía mayor a $1\,$EeV entre los años 2005-2018, extendiendo los periodos de tiempo estudiados anteriormente.


\chapter{Métodos}
	\graphicspath{{../1_Metodo/}}
	\input{../1_Metodo/metodo_v5}
	
	
\chapter{Dipolo en el rango 1 EeV - 2 EeV}
	\graphicspath{{../6_Dipole_1-2_EeV/}}
	\input{../6_Dipole_1-2_EeV/dipolo_1-2_EeV_v8}
	


% \appendix
% \chapter{Este capítulo es una ayuda-memoria}


Acá quiero describir paso por paso lo que hago en el análisis detalladamente en cada paso. Esto es de uso personal y no está pensado para que otra persona lo siga todavía.

	S\subsubsection{Cortes a los datos}
		\begin{itemize}
			\item ($2$) Number of Selected Stations > 2
			\item ($22$) Estimation and Reconstruction compatibles >0
			\item ($23$) Is T5 > 0
			\item ($43$) Number ot neighbours in the hottesk tank > 5 or 6. Depende del rango de energía. Este es el filtro de 5t5 o 6t5.
			\item ($44$) Flag with $43$ > 0
			\item ($48$) Bad period: 1. Si es 1 es un good period.
 		\end{itemize}

\section{Análisis de los parametros del  clima}

\subsection{Preparando los datos}
	 El archivo del datos tiene 48 columnas. La información sobre que es cada columna está en \url{http://ipnwww.in2p3.fr/~augers/AugerProtected/herald.php}.   Los datos que necesito del archivo de los eventos son estos:
			\begin{enumerate}
				\item (\$8)   UTC                  	: unix time 
				\item (\$4)   Phi                  	: angular value of $\phi$, 
				\item (\$3)   Theta                	: declinación
				\item (\$14)  Ra                   	: Ascensión recta
				\item (\$12)  S1000 sin corregir   	: Si corregir por el clima
				\item (\$47)  S38\_w                : Corregida por el clima, en el 2017 es $-1$
				\item (\$38)  Energy         		:
				\item (\$43)  Tanks                	: 
				\item (\$37)  S1000 con correccion 	: En el 2017 vale $-1$.
			\end{enumerate}

	Para el caso del archivo del clima tiene 12 columnas y la información de cada columna \url{http://auger.uis.edu.co/data/private.html}. Los datos se toman cada 5 minutos. Para los análisis necesitamos,

			\begin{enumerate}
				\item utc time
				\item Temperatura
				\item Presión
				\item rho
				\item rho media durante las 24 horas anteriores
				\item 6t5 (Está sumado durante 5 min, hay que dividir por 5 para el número posta)
				\item 5t5 (Está sumado durante 5 min, hay que dividir por 5 para el número posta)
				\item iw: bad weather
					\begin{itemize}
						\item 0: es lo que se registró
						\item 1: se interpoló con datos menos de 3 horas
						\item 4: Mayor a 3 horas.
					\end{itemize}
				\item bad period: 0 si estaba mal, 1 estaba bien
			\end{enumerate}

	Para cotinuar con el  análisis, tenemos que agregar dos columnas por nuestra cuenta
			\begin{enumerate}
				\item[10.] Densidad atrasada en dos horas
				\item[11.] Densidad media 12 horas antes y 12 horas después
			\end{enumerate}
		Para la densidad atrasada 2 horas, tengo que guardar la densidad de ahora e imprimirla dos horas despues


		Para la segunda parte, centrar alrededor de 24, necesito mandar lo de ahora 12 horas atrás.

	\subsubsection{Que hice de esto hasta ahora}

	Preparé el archivo del clima con las 11 columnas para todo el rango de  tiempo (2004 $\rightarrow$ 2019 incluido). 	Para los datos de eventos del 2019 , consideré solo los datos a partir de $1388910508$, antes tienen una tasa rara de eventos que tengo que preguntar.


	\subsection{Agregar a cada evento el clima}

	\subsection{Haciendo el ajuste}


	Considerando 


	\subsection{Problemas que tuve con el ajuste}

	\subsubsection{Tasa de eventos rara para AllTriggers ICRC 2019}
	\begin{figure}[htbp]
		\centering
		\includegraphics[width=\textwidth]{../Apendice/draft_s38_1EeV_weather_for_reconstruction.png}
		\caption{Hay una tasa de eventos baja en un rango de tiempo, esta media coincide con el main array.}
		\label{fig:tasarara}
	\end{figure}

\begin{biblio}
	\bibliography{mibib}
\end{biblio}


\end{document}