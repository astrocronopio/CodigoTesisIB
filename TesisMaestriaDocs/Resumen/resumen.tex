\begin{resumen}%
Cuando un rayo cósmico interactúa con una molécula en la parte superior de la atmósfera, se inicia un proceso en el cual se generan otras partículas secundarias. Este proceso es conocido como lluvia atmosférica extendida. Estas lluvias pueden ser detectadas sobre la superficie de la Tierra mediante varios experimentos. Este trabajo utiliza los datos recolectados por los detectores de superficie separados en 1500\,m entre sí del Observatorio Pierre Auger durante los años 2005-2015. Las condiciones atmosféricas como la presión (P), la temperatura (T) y la densidad ($\rho \propto \nicefrac{P}{T}$) afectan el desarrollo de la lluvia a través de la atmósfera. A partir de un análisis de la modulación inducida en la estimación de la energía por las variaciones en las condiciones atmosféricas. Estas fueron tenidas en cuenta en  la reconstrucción oficial de eventos. En este trabajo extendimos el periodo de tiempo analizado para analizar está modulación, y se observó que los parámetros obtenidos son comparables con la reconstrucción oficial. 
\end{resumen}

%\begin{abstract}%
%When a cosmic ray arrives to the upper atmospher
%\end{abstract}


\begin{nemombyky}%
Mbyjakua\'ape (\emph{astronomía} karaiñe'\~eme) ojeikuaase mba\textquotesingle e oik\'ova umi mba\textquotesingle e  michĩ yv\'agagui o\'uva (\emph{rayos cósmicos} karaiñe'\~eme) oguah\~evove amo yvatetépe (\emph{atmósfera} karaiñe'\~eme). Ombok\'aramo tuminguaave\textquotesingle \~yty (\emph{conjunto de átomos o molécula}  karaiñe'\~eme ) yvatetépe oĩva, oñepyr\~u ojapo het\~a umi tuminguaave\textquotesingle \~yjokaku\'era (\emph{partículas}  karaiñe'\~eme ) op\'arupi. Ko\textquotesingle a       ha\textquotesingle e  h\'ina peteĩ ama guasu tuminguaave\textquotesingle \~yjoka rehegua ( \emph{lluvia atmosf\'erica extendida} karaiñe'\~eme). Umi ama guasuku\'era tuichaterei ha ikatu eñeña\textquotesingle ã yvy ári op\'arupi. Mend\'osape oĩ peteĩ mba\textquotesingle etuicha h\'erava \emph{Pierre Auger} Mbyjañama\textquotesingle \~eha\~gua (\emph{Observatorio Pierre Auger}) oña'\~ava ko ama. Ko\textquotesingle  ape romba\textquotesingle  ap\'ota umi ama ko mbyjañama\textquotesingle \~eha\~gua oña\textquotesingle \~ava\textquotesingle  kue 2005-guive 2018-peve. Mba\textquotesingle \'eichapa umi amaku\'era oguah\~e yvy \'ari ikatu ojuavy hakúramo (T, \emph{temperatura} karaiñe\textquotesingle \~eme) tér\~a  poh\'yiramo pe pytundyry mbyjañama\textquotesingle \~eha\~gua áripe ($\rho$, \emph{densidad} h\'erava karaiñe\textquotesingle \~eme). Ko mbyjañama'\~eha\~gua ojapova\textquotesingle ekue peteĩ tembiapo ha ko\textquotesingle ape rojapojey up\'eva roikuaaha\~gua umi papapo oñenoh\~eva\textquotesingle ekue oiko gueteri ko'\~anga peve, ha rotopa kóva oikópa añetete.
\end{nemombyky}



%%% Local Variables: 
%%% mode: latex
%%% TeX-master: "template"
%%% End: 
