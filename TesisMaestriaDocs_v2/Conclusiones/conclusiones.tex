
En la primera etapa de este trabajo, se analizaron los efectos de las variaciones de los parámetros del clima sobre el desarrollo en la atmósfera de las lluvias atmósferas. Se analizaron datos del arreglo de detectores espaciados 1500 m entre sí del Observatorio Pierre Auger, en el periodo 2005-2015 y 2005-2018 extendiendo los periodos de tiempo estudiados anteriormente. Se emuló los resultados de la corrección de la modulación del clima sobre el periodo 2005-2015 realizados sobre eventos del Disparo Estándar, este disparo tiene una eficiencia del 100 \% en eventos de energía mayor a $3\,$EeV. Los resultados obtenidos en este trabajo son compatibles con los parámetros obtenidos por la colaboración Pierre Auger. Se observó que posterior a la corrección por parte del Observatorio, la modulación del clima se vio disminuida. Para eventos con energía mayor a $2\,$EeV, esta modulación es despreciable.

Posterior a los análisis anteriores, se estudió la modulación del clima mediante el valor del $S_{38}$ sin la corrección propuesta por trabajos anteriores. Se observó que los parámetros del clima obtenidos de estos datos son compatibles con los utilizados en la reconstrucción oficial. Se realizó un corrección a la energía mediante los coeficientes nuevos, observándose que la modulación era despreciable para energías mayores a $2\,$EeV. 

También se trabajó con el archivo de Todos los Disparos, el mismo alcanza una eficiencia del 100 \% para una energía $1\,$EeV siendo ideal para estudiar el rango $1\,$EeV - $2\,$EeV. Primeramente, se obtuvieron los parámetros del clima en el periodo 2014-2019 asociados a eventos de Todos los Disparos sin la corrección del clima, esto se logró mediante el análisis del valor de la señal $S_{38}$ asociada a $1\,$EeV. Los valores obtenidos no son compatibles con los parámetros de la reconstrucción oficial, esto puede deberse a que la eficiencia del disparo o a la poca cantidad de años de adquisición de datos de este disparo.

Se implementó el algoritmo del método Rayleigh  y East - West, se verificaron los códigos reobteniendo resultados reportados por la Colaboración Pierre Auger. Con el método de Rayleigh se estudió el rango de energía $1\,$EeV - $2\,$EeV obteniéndose resultados inconcluyentes acerca de la presencia de una anisotropía dipolar en este rango de energía. Esto se debe a la presencia de amplitudes espurias por encima del umbral de $1\%$  de probabilidad de que la señal provenga de una distribución isótropa de rayos cósmicos.  

Se aplicó nuevamente la corrección del clima a los eventos de Todos los Disparos, utilizando los parámetros obtenidos para este conjunto de datos,  luego se procedió nuevamente con el análisis de Rayleigh en el rango $1\,$EeV - $2\,$EeV con la energía corregida por este trabajo. Esta corrección no disminuyó considerablemente las amplitudes espurias en el barrido de frecuencias mencionadas anteriormente, por lo que los resultados siguen sin ser concluyentes en la frecuencia sidérea.

Con el método East-West fue posible estudiar rangos de menor energía que $1\,$EeV. Se propuso una variable para generalizar el análisis a frecuencias arbitrarias, por lo que sobre el archivo de Todos los Disparos se realizó un barrido de frecuencias en  los rangos $0.25\,$ - $0.5\,$EeV, $0.5\,$ - $1\,$EeV y $1\,$- $2$EeV. En estos intervalos de energía no se encontraron amplitudes por encima del umbral del $1\%$ en ninguna frecuencia, pero los  resultados obtenidos son consistentes con trabajos publicados por la Colaboración Pierre Auger realizados sobre eventos del Disparo Estándar con el método East-West. En particular, en el rango $1\,$EeV - $2\,$EeV los resultados obtenidos con el método East-West y Rayleigh (corregidos por la Colaboración y por este trabajo) son consistentes


