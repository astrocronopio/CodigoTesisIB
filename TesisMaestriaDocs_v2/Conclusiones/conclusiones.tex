\chapter{Conclusiones}

%\begin{enumerate}
	%\item ¿Hizo alguna diferencia a la energía la corrección del clima?
	%\item ¿Hizo alguna diferencia la evolución del tiempo de los hexágonos a los parámetros del clima?
	%\item  ¿Disminuyó la modulación?
	%\item ¿El S38 sin corregir por clima me da los mismos resultados que lo anterior?
%	\item 
%\end{enumerate}


En este trabajo se analizaron los efectos de las variaciones de los parámetros del clima sobre el desarrollo en la atmósfera de las lluvias atmósferas. Se analizaron datos del arreglo de detectores espaciados 1500 m entre sí del Observatorio Pierre Auger, en el periodo 2005-2015 y 2005-2018 extendiendo los periodos de tiempo estudiados anteriormente. Se emuló los resultados de la corrección de la modulación del clima sobre el periodo 2005-2015 de la colaboración Pierre Auger, obteniéndose resultados compatibles. Se observó que posterior a la corrección, la modulación del clima se vio disminuida. Para eventos con energía mayor a $2\,$EeV, esta modulación es despreciable.

Posterior a los análisis anteriores, se estudió la modulación del clima mediante el valor del $S_{38}$ sin la corrección propuesta por trabajos anteriores. Se observó que los parámetros del clima obtenidos de estos datos son compatibles con los utilizados en la reconstrucción oficial. Se realizó un corrección a  la energía mediante los coeficientes nuevos, observándose que la modulación era despreciable para energías mayores a $2\,$EeV. 


%En este trabajo se estudió eventos con energía mayor a $1\,$EeV entre los años 2005-2018, extendiendo los periodos de tiempo estudiados anteriormente.
