

El método de Rayleigh se basa en ajustar el flujo de CRs en función de la ascensión recta $\alpha$ mediante una función armónica. El mismo permite calcular la amplitud y fase de la anisotropía para distintos armónicos, además de la probabilidad de detectar la misma señal debido a fluctuaciones de una distribución isótropa de RCs. 

La dificultad de utilizar el método Rayleigh recae en su sensibilidad a efectos sistemáticos: efectos del clima, variaciones en el área del Observatorio, y la sensibilidad de los instrumentos deben tenerse en cuenta.  Los efectos mencionados deben ser corregidos de la señal medida de los eventos, ya que los mismos inducen modulaciones espurias en el análisis.

El método East - West consiste en el ajuste de una función armónica a la diferencia entre los flujos de eventos provenientes del Este y del Oeste. Si se consideran que las modulaciones espurias producidas por los efectos atmosféricos y sistemáticos son las mismas en ambas direcciones, la diferencia de flujos remueve estos efectos sin realizar correcciones adicionales. Una desventaja de este método es que su sensibilidad es menor que la del método de Rayleigh \cite{taborda}.


\section{Descripción formal del método East-West}
 

    \subsection{Flujo de eventos del Este y Oeste}
    El flujo de eventos observado $I_{\phi_1, \phi_2}^{obs}(\alpha^0)$  entre los ángulos azimutales $\phi_1$ y $\phi_2$ cuando el cenit se encuentra en la posición  $\alpha^0$ en el cielo,  puede calcularse  de la siguiente manera:
    \begin{equation}
        I_{\phi_1, \phi_2}^{obs}(\alpha^0) = \int_{\phi_1}^{\phi_2} d\phi \int_{0}^{\theta_{max}} d\theta \sin\theta \tilde{\omega}(\theta, \alpha^0) \Phi(\theta, \phi, \alpha^0),
        \label{eq:rate_general}
    \end{equation}
    \noindent  donde  el término $\tilde{\omega}(\theta, \alpha^0)$ representa la exposición del Observatorio y $\Phi(\theta, \phi, \alpha^0)$ es el flujo total de RCs expresado en coordenadas locales. El término $\tilde{\omega}(\theta, \alpha^0)$  también incluye los efectos sistemáticos, como la variación de los hexágonos del arreglo mediante  su dependencia con $\alpha^0$.

    Para calcular los flujos de eventos del Este y Oeste, $I^{obs}_E$ y $I_O^{obs}$ respectivamente, se integra la Ec.\ref{eq:rate_general} en los siguientes  rangos: para el Este entre $\phi_1=\nicefrac{-\pi}{2}$ y $\phi_2=\nicefrac{\pi}{2}$ y para el Oeste entre $\phi_1=\nicefrac{\pi}{2}$ y $\phi_2=\nicefrac{3\pi}{2}$.

    \subsection{Aproximaciones del método}
    Se considera que la exposición $\tilde{\omega}$ no depende de $\phi$ y que pueden  desacoplarse de las variaciones en $\theta$ y $\alpha^0$. Por lo tanto, podemos expresar $\tilde{\omega}$ de la siguiente manera:
    \begin{equation}
        \tilde{\omega}(\theta, \alpha^0) = \omega(\theta)F(\alpha^0)
    \end{equation}    
    
    A su vez, consideremos que las amplitudes de las variaciones temporales asociadas a $\tilde{\omega}$ son pequeñas con respecto al valor medio, por lo que se puede tomar la expansión en primer orden de la función $F(\alpha^0)$:
    \begin{equation}
        \tilde{\omega}(\theta, \alpha^0) = \omega(\theta)\big(1 + \eta(\alpha^0) \big)
        \label{eq:omega_expandido}
    \end{equation}
    \subsection{Cálculo de la diferencia de flujos}
  
    Teniendo en cuenta las expansiones hasta el primer orden de $\tilde{\omega}$ en la Ec.\ref{eq:omega_expandido} y del flujo de RCs $\Phi$ en la Ec. \ref{eq:dipolo_general}, se tiene la siguiente expresión:
    \begin{align}
        I_{\phi_1, \phi_2}^{obs}(\alpha^0) &= \int_{\phi_1}^{\phi_2} d\phi \int_{0}^{\theta_{max}} d\theta  \sin\theta \omega(\theta)\big(1 + \eta(\alpha^0) \big) \Phi_0 ( 1 +  \textbf{d}\cdot\hat{\bf{u}}) \label{eq:I-obs}
    \end{align}
    donde la segunda parte  de la igualdad  puede simplificarse con una definición apropiada  \footnote{ $
        \text{Por simplicidad, definimos la siguiente expresión: }
      \overline{f(\theta)} = \int_{0}^{\theta_{max}} d\theta \sin\theta \omega(\theta) f(\theta)
      \label{eq:media_angular}
  $
  \noindent, donde $\overline{f(\theta)}$ es la media de la función $f(\theta)$ sobre el ángulo cenital pesado por la exposición del Observatorio $\omega(\theta)$, hasta  un ángulo máximo de $\theta_{max}$. En este trabajo se centra en eventos hasta 2 EeV, por lo que $\theta_{max}=60^o$ para los datos del Observatorio.}. Dado que la integral sobre $\phi$ tiene el mismo valor para el Este y Oeste, se obtiene que la expresión asociada a orden cero de $\Phi$ puede escribirse de la siguiente forma
    \begin{align*}
        &\int_{\phi_1}^{\phi_2} d\phi \int_{0}^{\theta_{max}} d\theta \sin\theta \omega(\theta)\big(1 + \eta(\alpha^0) \big) \Phi_0 
        = \Phi_0 (1+ \eta(\alpha^0)) \pi\,\overline{1}. 
    \end{align*}
    Trabajando con la expresión asociada al primer orden de $\Phi$, si consideramos la expresión \ref{eq:dot-prod-local} del producto escalar $\textbf{d}\cdot\hat{\bf{u}}$ en coordenadas locales, e integramos el ángulo  $\phi$ entre $[\nicefrac{-\pi}{2}, \nicefrac{\pi}{2}]$ o $[\nicefrac{\pi}{2}, \nicefrac{3\pi}{2}]$, se obtiene que:
    \begin{align}
        &\int_{\phi_1}^{\phi_2} d\phi \int_{0}^{\theta_{max}} d\theta \sin\theta \omega(\theta)\big(1 + \eta(\alpha^0) \big) \Phi_0 \textbf{d}\cdot\hat{\bf{u}}=\\
        &=\Phi_0 (1+ \eta(\alpha^0)) \int_{0}^{\theta_{max}}  d\theta (\pm 2d_{x'}\sin\theta 
        + \pi d_{z'}\cos\theta) \label{segundo_term}
    \end{align}
    \noindent donde $+2$ corresponde al Este y $-2$ al Oeste. No hay una dependencia con la proyección del dipolo $d_{y'}$ porque en la integral aparece el término $\int_{\phi_1}^{\phi_2}\, d\phi\, d_{y'}(\alpha^0, \delta^0) \sin\theta \sin\phi $, que se anula al integrar sobre el Este y Oeste.

     Usando la definición dada en la nota de pie \ref{eq:media_angular} de la página anterior y la expresión \ref{segundo_term}, podemos reescribir la expresión \ref{eq:I-obs} y los flujos para el Este y el Oeste como:
    \begin{align*}
    I^{obs}_E&= \Phi_0 (1+ \eta(\alpha^0)) \Big( \pi\overline{1} + 2d_{x'}\overline{\sin\theta} + \pi d_{z'}(\alpha^0)\overline{\cos\theta}  \Big) \\
        I^{obs}_O&= \Phi_0 (1+ \eta(\alpha^0)) \Big( \pi \overline{1} - 2d_{x'}\overline{\sin\theta}   + \pi d_{z'}\overline{\cos\theta} \Big) \\
        I_{Total}^{obs}=I^{obs}_E +I^{obs}_O &= \Phi_0 (1+ \eta(\alpha^0)) \Big( 2\pi\overline{1} +2\pi d_{z'}\overline{\cos\theta}  \Big)
    \end{align*}
    Ya que se busca calcular la diferencia entre los flujos provenientes del Este y del Oeste, $I^{obs}_E $ y $  I^{obs}_O $ respectivamente, esta resta queda como:
    \begin{equation*}
        I^{obs}_E -  I^{obs}_O = 4 \Phi_0 (1+ \eta(\alpha^0)) \,  d_{x'}(\alpha^0)\overline{\sin\theta}
    \end{equation*}

    Para obtener las componentes del vector $\bf{d}$, tenemos que considerar que las proyecciones que están en el plano x'-z'\,  ya que no hay dependencia con la proyección $d_{y'}$. Para hacer esto, consideremos a los versores $\hat{\bf{u}}_{x'}$ y $\hat{\bf{u}}_{z'}$ que apuntan al cenit y al Este respectivamente. Podemos obtener las proyecciones con un producto escalar con los versores en las direcciones de interés:
    \begin{equation}
        d_{x'}(\alpha^0) \hat{x}' =  (\textbf{d}\cdot\hat{\bf{u}}_{x'})\hat{\bf{u}}_{x'} \rightarrow d_{x'}(\alpha^0) = \textbf{d}\cdot\hat{\bf{u}}_{x'},
    \end{equation}
    donde estos versores en coordenadas ecuatoriales se escriben como:
    $\hat{\bf{u}}_{z'}= (\alpha^0,\delta^0 )$ y $ \hat{\bf{u}}_{x'} = (\alpha^0 + \frac{\pi}{2},0)$ \footnote{Se suma  $\frac{\pi}{2}$ para apuntar al Este, cuando el versor recorre $\nicefrac{\pi}{2}$ en ascensión recta, llega al plano del ecuador que tiene declinación $0$.}.

    Usando la expresión \ref{eq:product_ud} para el producto escalar en coordenadas ecuatoriales, se obtienen las componentes:
    \begin{align}
        d_{z'}=\textbf{d}\cdot\hat{\bf{u}}_{z'} &= d (\cos\delta_d \cos\delta^0 \cos(\alpha^0 - \alpha_d) + \sin\delta_d  \sin\delta^0)\\
        d_{x'}=\textbf{d}\cdot\hat{\bf{u}}_{x'} &= d \cos\delta_d \cos(\alpha^0 +\frac{\pi}{2} - \alpha_d) 
        = -d\cos\delta_d \sin(\alpha^0  - \alpha_d) \label{eq:total_flux_eta}
    \end{align}
    
     Entonces la diferencia entre flujos queda como:
    \begin{equation}
        I^{obs}_E -  I^{obs}_O =-4d \Phi_0 (1+ \eta(\alpha^0)) \cos\delta_d \sin(\alpha^0  - \alpha_d)\overline{\sin\theta}
        \label{resta}
    \end{equation}

    Esta diferencia se debe relacionar con la variación del flujo total verdadero $I(\alpha^0)$, es decir el flujo que se observaría si no existieran variaciones temporales en ascensión recta en la exposición, que implicaría $\eta(\alpha^0)=0$. {Las ecuaciones relacionadas con el flujo total medido $I^{obs}_{Total}$ son válidas para el caso de $\eta(\alpha^0)=0$.}

  {  Considerando la Ec.\ref{eq:total_flux_eta} para el caso de $I(\alpha^0)$ con $\eta(\alpha^0)=0$, la variación del flujo verdadero en ascensión recta provee información sobre la componente $d_{z'}$ del dipolo.} 
    \begin{align}
        \dv{I(\alpha^0)}{\alpha^0}  & = 2\pi\Phi_0 \overline{\cos\theta} \dv{\,d_{z'}(\alpha^0) }{\alpha^0}\\ 
        \dv{I(\alpha^0)}{\alpha^0} &= -2d\pi\Phi_0 \overline{\cos\theta}\cos\delta_d \cos\delta^0 \sin(\alpha^0 - \alpha_d) \label{total_flux}
    \end{align}

    Para llegar a la expresión \ref{resta}, hicimos la expansión hasta el primer orden de $\tilde{\omega}(\theta, \alpha^0)$ y de $\bf\Phi(\alpha, \delta)$, por lo tanto, para ser consistentes en el orden de aproximación, se desprecia el término de segundo orden de la expresión \ref{resta} que es proporcional a $\eta \cdot d$ y la expresión \ref{resta} queda:
        \begin{equation}
            I^{obs}_E -  I^{obs}_O \approx -4d \Phi_0 \cos\delta_d \sin(\alpha^0  - \alpha_d)\overline{\sin\theta}
            \label{resta_final}
        \end{equation}

    Considerando las expresiones \ref{total_flux} y \ref{resta_final}, se tiene una relación entre la diferencia de flujo del Este y del Oeste medido por el Observatorio  y el flujo real de RCs \footnote{
    $
        \text{Se usa la expresión:  }
        \langle f(\theta) \rangle = \frac{\overline{f(\theta)}}{\overline{1}} = \displaystyle\frac{\int_{0}^{\theta_{max}} d \theta \sin\theta \omega(\theta) f(\theta) }{\int_{0}^{\theta_{max}} d \theta \sin\theta \omega(\theta)} 
    $,  
    que es equivalente a hacer la media  de todos los datos medidos de $f(\theta)$.} %Como consideramos una expansión de $\omega(\theta)$ }:
    \begin{equation}
        I^{obs}_E -  I^{obs}_O \approx  \frac{2}{\pi \cos \delta^0} \frac{\langle\sin\theta \rangle}{\langle\cos\theta \rangle}\dv{I(\alpha^0)}{\alpha^0} 
        \label{eq:final}
    \end{equation}
   

\section{Estimación de la componente ecuatorial del dipolo mediante el análisis del  primer armónico}


El objetivo del método  East - West es estimar la modulación dipolar de  $I(\alpha^0)$ a partir de la diferencia $I^{obs}_E -  I^{obs}_O$ mediante un análisis similar al método de  Rayleigh que se muestra en la Ec.\ref{eq:general-1-ray}, salvo modificaciones para tener en cuenta la dirección de los eventos. 
\begin{equation}
    I^{obs}_E -  I^{obs}_O = \frac{N}{2\pi} r_{EW} \cos(\alpha^0 -  \phi_{EW}) \label{eq:ray_ew_like}
\end{equation}
donde a diferencia de la expresión original, la amplitud $r_{EW}$ y fase $\phi_{EW}$ no son la amplitud y fase de la modulación en ascensión recta. Las mismas están asociadas a la modulación en la diferencia de flujos, a continuación se explica como se relacionan con $r_1$ y $\phi$  a partir del   método East - West.


Esta relación puede obtenerse reescribiendo la expresión \ref{eq:final}, teniendo en cuenta la proyección del dípolo físico sobre el ecuador $d_{\perp}= d\cos\delta^0$, la expresión  \ref{resta_final} y que $N \simeq 4\pi^2 \Phi_0 \overline{1} $ \footnote{Porque es la integral con respecto a los dos ángulos, $\theta$ y $\phi$}:
\begin{align}
    I^{obs}_E -  I^{obs}_O \approx -4 d_\perp \frac{N}{ 4\pi^2\overline{1}} \sin(\alpha^0  - \alpha_d)\overline{\sin\theta} \frac{\overline{1}}{\overline{1}}\\
    I^{obs}_E -  I^{obs}_O \approx -4 d_\perp \frac{N}{ 4\pi^2} \sin(\alpha^0  - \alpha_d)\langle\sin\theta \rangle\\
    I^{obs}_E -  I^{obs}_O \approx -\frac{N}{2\pi} d_\perp \frac{2\langle\sin\theta \rangle }{\pi}\sin(\alpha^0  - \alpha_d) \label{ultima_ew_ray}
\end{align}


Comparando las expresiones \ref{eq:ray_ew_like} y \ref{ultima_ew_ray} y considerando la ecuación \ref{eq:fourier_perp}, se puede inferir que las relaciones entre la amplitud y fase obtenidas mediante EW y el dipolo físico son las siguientes:

\begin{tabular}{@{}p{.4\linewidth}@{}p{.5\linewidth}@{}}
    \begin{align}
        d_{\perp} = \frac{\pi}{2\langle\sin\theta \rangle} r_{EW} \label{dperp} \\
        r_1   =\frac{\pi}{2} \frac{\langle\cos\delta \rangle}{\langle\sin\theta \rangle} r_{EW} \label{r_fisico}  \\
        \alpha_d = \phi_{EW} + \frac{\pi}{2} \label{phase_fisico}
    \end{align}
    &    \begin{align}
        \sigma_{x,y} = \frac{\pi}{2\langle\sin\theta \rangle} \sqrt{\frac{2}{N}}\\
        \sigma   = \frac{\pi \langle\cos\delta \rangle}{2\langle\sin\theta \rangle} \sqrt{\frac{2}{N}}
    \end{align}
  \end{tabular}

Como en el caso del análisis de Rayleigh, la probabilidad de obtener una amplitud mayor o igual a que $r_{EW}$ a partir de una distribución isótropa es una distribución acumulada de Rayleigh:
\begin{equation}
    P(\geq r_{EW}) = \exp \Big(-\frac{N}{4}r^2_{EW}\Big) \label{p99}
\end{equation}

\subsection{Cálculo  de la amplitud del dipolo para los eventos de Todos los Disparos}

\begin{enumerate}
    \item Definimos el rango de tiempo a estudiar, para los resultados para Todos los Disparos se utilizaron los límites: 1 de Enero del 2014 hasta el 1 de Enero del 2020.
    \item Se recorre cada evento que cumpla con las siguientes características:
     \begin{itemize}
        \item Pertenezca el rango de energía a estudiar
        \item Sea un evento 6T5 con ángulo cenital menor a $60^o$
        \item Se haya registrado en el rango de tiempo seleccionado
    \end{itemize}
    En cada evento se calcula los siguientes valores:
    \begin{align}
        a_i' = \cos(X_i - \beta) \qquad
        b_i' = \sin(X_i - \beta)
    \end{align}
    el valor de $X_i$ depende la frecuencia a estudiar, la misma es igual a la ascensión recta del cenit $\alpha^0_i$ al momento del evento  si se estudia la frecuencia sidérea, en cambio para la frecuencia solar es igual al equivalente en grados de la hora local de Malargüe. El valor de $\beta$ depende si el evento provino del Este donde $\beta=180^o$ o $\beta=0$ caso contrario.
    
    \item Una vez corridos todos los  eventos se calculan los parámetros:
    \begin{align*}
        a_{EW} &= \frac{2}{N} \sum^N_{i=1}a_i' =\frac{2}{N} \sum^N_{i=1} \cos(X_i - \beta_i)\\
        b_{EW} &= \frac{2}{N} \sum^N_{i=1}b_i' =\frac{2}{N} \sum^N_{i=1} \sin(X_i - \beta_i)
    \end{align*}
    donde N indica la cantidad eventos considerados. La cantidad de eventos por rango de energía se muestran en la tabla \ref{tab:datasets}.

    Con esto puedo calcular la amplitud asociada al análisis $r_{EW}$ y la fase $\phi_{EW}$:
    \begin{align*}
        r_{EW} = \sqrt{a_{EW}^2 + b_{EW}^2}\\
        \phi_{EW} = \tan^{-1}(\nicefrac{b_{EW}}{a_{EW}})
    \end{align*}

    Estos valores se traducen a los valores de amplitud $r_1$, $d_\perp$ y fase $\phi$ del dipolo físico mediante las expresiones \ref{dperp}, \ref{r_fisico} y \ref{phase_fisico}.   Los valores $\langle\cos\delta \rangle$ y $\langle\sin\theta \rangle$ son los valores medios de estas variables en los eventos estudiados. En el caso de $\phi$ se espera que el mismo sea un estimador del valor de $\alpha_d$.

    \item Se calcula la amplitud límite $r_{99}$ y la probabilidad  $P(r_{EW})$ utilizando la expresión \ref{p99}:
    \begin{align*}
        r_{99} &= \frac{\pi}{2} \frac{\langle\cos\delta \rangle}{\langle\sin\theta \rangle}\sqrt{\frac{4}{N}\ln(100)}\\
        d_{\perp,99} &= \frac{r_{99}}{\langle\cos\delta \rangle}    
    \end{align*}

    \item Se calculan los límites de confianza de las variables $r$,$\phi$ y $d_\perp$ mediante los densidad de probabilidad de la amplitud y fase. Las mismas se describen en el capítulo \ref{PDFs}.


\end{enumerate}


Por último, estos resultados se comparan con los valores obtenidos con el método EW en el trabajo \cite{Aab_2020} en frecuencia sidérea, aplicado al conjunto de eventos del disparo estándar registrados entre el 1 de Enero del 2004 y el 1 de Agosto del 2018. 


\subsection{Cálculo para frecuencias  arbitrarias}

Cambiamos las variable de la ascensión recta del cenit $\alpha^0$ por
\begin{equation}
    \tilde{\alpha} = 2\pi f_x t_i  \label{ra_arb}
  \end{equation}
donde $f_x$ es la frecuencia arbitraria a estudiar y $t_i$ es el momento donde ocurre el evento a estudiar. Luego se realizan el mismo procedimiento que lo anterior para calcular el valor de la amplitud $r$.

En la siguiente sección se verifica que se obtiene los mismo resultados con esta variable general que con el valor de $\alpha^0$ para la frecuencia sidérea.

\section{Verificación del código}

\subsection{Comparación con el trabajo de la Colaboración Pierre Auger}
Se verificó el código escrito en este trabajo de la siguiente manera:

\begin{enumerate}
    \item El conjunto de eventos del disparo estándar registrados entre el 1 de Enero del 2004 y el 1 de Agosto del 2018 fue analizado en el trabajo \cite{Aab_2020}.
    \item Utilizando el código y los datos de los eventos del paper \cite{Aab_2020}, obtenidos de la página del \emph{Publications Committee} de la colaboración Auger, se replicaron los datos del paper. 
    \item Luego utilizando el código escrito para este trabajo, se realizó el análisis de EW con los datos del trabajo \cite{Aab_2020}. 
    \item Finalmente se verificó que los valores obtenidos en los item 2 y 3, con  ambos códigos, sean el mismo.
\end{enumerate}

\subsection{Comparando con la variable $\tilde{\alpha}$ con la ascensión recta del cenit }

Para verificar que la variable de la Ec.\ref{ra_arb} es útil para estudiar otras frecuencias, en la Tabla~\ref{tab:comp_vars} se comparan los resultados de la referencia para el rango 0.25 - 0.5 EeV, los obtenidos usando la ascensión recta del cenit y los valores obtenidos con la Ec.\ref{ra_arb} en el mismo rango de energía. Se observan que los valores son consistentes.

\begin{table}[H]
    \begin{small}
        \begin{center}
            \begin{tabular}[c]{l|l|l|l}
                                    & \cite{Aab_2020} & $\alpha^0$   & $\alpha=2\pi f_xt_i$   \\ \hline
                Frecuencia:         & 366.25          &  366.25      &  366.25            \\
                $d_\perp$[\%]:      & 0.60            &  0.60        &  0.60              \\
                $\sigma_{x,y}$[\%]  & 0.48            &  0.48        &  0.48              \\ 
                Probabilidad:       & 0.45            &  0.45        &  0.45              \\
                Fase[$^o$]:         & 225$\pm$64      &  225$\pm$64  &  225$\pm$64          \\
                $r_{99}$[\%]:       & 1.5             &  1.5         &  1.5             \\
                $d_{\perp,99}$[\%]: & 1.8             &  1.8         &  1.8             \\
            \end{tabular}
        \end{center}
        \caption{Verificando la  variable $\tilde{\alpha}=2\pi ft$ para el análisis de frecuencias arbitrarias en el método East-West.}
        \label{tab:comp_vars}
    \end{small}
\end{table}

