
\chapter{Método East-West}

El método de Rayleigh se basa ajustar el flujo de CRs en función de la ascensión recta mediante una función armónica. El mismo permite calcular la amplitud de la anisotropía para distintos armónicos, su fase y la probabilidad de detectar la misma señal debido a fluctuaciones de una distribución isótropa de RCs. 

La dificultad en utilizar el método Rayleigh recae en procesamiento de los datos: efectos del clima,  variaciones en el área del Observatorio, y la sensibilidad de los instrumentos deben tenerse en cuenta.  Los efectos mencionados deben ser corregidos de la señal medida de los eventos, ya que los mismos inducen modulaciones espurias en el análisis.

En el método East - West consiste en el ajuste de una función armónica a la diferencia entre los flujos de eventos provenientes del Este y del Oeste. Si se consideran que las modulaciones espurias producidas por los efectos atmosféricos y sistemáticos son las mismas en ambas direcciones, la diferencia de flujos remueve estos efectos sin realizar correcciones adicionales. Una desventaja de este método es que su sensibilidad es menor que el método de Rayleigh \cite{taborda}.


\section{Descripción de una anisotropía dipolar}
Las anisotropías en las direcciones de llegada de los RCs indican que ciertas zonas del cielo tienen una variación significativa con respecto a la media de flujo de RCs. Esta anisotropía puede describirse mediante funciones armónicas, la más sencilla es la anisotropía dipolar es una aproximación a primer orden.

Una anisotropía dipolar se puede describir de la siguiente forma:
\begin{equation}
    \Phi(\hat{\bf{u}}) = \Phi_0(1+\bf{d}\cdot\hat{\bf{u}})
    \label{eq:dipolo_general}
\end{equation}
\noindent donde $\Phi_0$ es el flujo medio de eventos, $\hat{\bf{u}}$ es un versor que apunta a la dirección a estudiar, y $\bf{d}$ es el vector con módulo igual a la amplitud del dipolo y con dirección al eje del mismo. Tomando coordenadas ecuatoriales \footnote{Referencia al apéndice de ecuatoriales}, la dirección de $\bf{d}$ es $(\alpha_d, \delta_d$) y  de $\hat{\bf{u}}$ es $(\alpha, \delta)$, por lo tanto  el producto escalar  entre estos vectores se puede escribir de la siguiente manera \footnote{Falta mencionar el producto de versor de esta representación para decir sale de acá. Eso va a estar en el apéndice porque el cálculo me  sale fácil poniendo  todo en cartesianas.)}:
\begin{equation}
    \textbf{d}\cdot\hat{\bf{u}}= d (\cos\delta_d \cos\delta \cos(\alpha - \alpha_d) + \sin\delta_d  \sin\delta)
    \label{eq:product_ud}
\end{equation}


Otro aspecto importante de la representación del dipolo en coordenadas ecuatoriales, es que la proyección de la amplitud del dipolo sobre el plano ecuatorial se puede aproximar de la siguiente manera:
\begin{equation}
    r_1 \simeq d_\perp \langle \cos\delta \rangle
    \label{eq:fourier_perp}
\end{equation}
donde $r_1$ es la amplitud de la aproximación a primer orden en Fourier, y $\langle \cos\delta \rangle$ es el valor medio de $\cos\delta $ de los eventos utilizados para realizar la aproximación mencionada.

\subsection{Representación en coordenadas locales de la anisotropía dipolar}
Podemos reescribir el producto escalar entre dipolo $\textbf{d}$ y el versor $\hat{u}$ apunta en la dirección cualquiera mediante las coordenadas locales $\theta$ y $\phi$ \footnote{Referencia locales apéndice}.
\begin{align}
    \textbf{d} &=  d_x(\alpha^0)\hat{x} +  d_y(\alpha^0)\hat{y}+ d_z(\alpha^0)\hat{z} \\
    \hat{\bf{u}} &=\sin\theta \cos\phi \hat{x} + \sin\theta \sin\phi \hat{y} + \cos\theta\hat{z}\\
    \textbf{d}\cdot\hat{\bf{u}} &= d_x(\alpha^0)\sin\theta \cos\phi
    + d_y(\alpha^0) \sin\theta \sin\phi  
     + d_z(\alpha^0)\cos\theta \label{eq:dot-prod-local}
\end{align}
donde los versores $\hat{x}$, $\hat{y}$ y $\hat{z}$ apuntan a la dirección Este, Norte  y  del cenit respectivamente. 

El dipolo está fijo en el cielo pero visto desde las coordenadas locales para poder trabajar con $\theta$ y $\phi$, sus proyecciones  proyecciones $d_x$, $d_y$ y $d_z$  tienen una dependencia con la ascensión recta  $\alpha^0$ y declinación $\delta_0$ del cenit. 

\section{Descripción formal del método East-West}

    \subsection{Flujo de eventos del Este y Oeste}
    El flujo de eventos observado $I^{obs}(\alpha^0)$ para la ascensión recta del cenit $\alpha^0$ entre los ángulos azimutales $\phi_1$ y $\phi_2$ puede calcularse mediante el flujo total de RCs $\Phi(\theta, \phi, \alpha^0)$ (expresado en coordenadas locales) como
    \begin{equation}
        I^{obs}(\alpha^0) = \int_{\phi_1}^{\phi_2} d\phi \int_{0}^{\theta_{max}} d\theta \sin\theta \tilde{\omega}(\theta, \alpha^0) \Phi(\theta, \phi, \alpha^0),
        \label{eq:rate_general}
    \end{equation}
    \noindent  donde  el término $\tilde{\omega}(\theta, \alpha^0)$ representa la exposición del Observatorio. Este término también incluye los efectos sistemáticos y atmosféricos, como la variación de los hexágonos del arreglo y las correcciones de la modulación del clima, mediante  su dependencia con $\alpha^0$.

    Para calcular los flujos de eventos del Este y Oeste, $I^{obs}_E$ y $I_O^{obs}$ respectivamente, se integra la Ec.\ref{eq:rate_general} en los siguientes  rangos: para el Este: entre $\phi_1=\nicefrac{-\pi}{2}$ y $\phi_2=\nicefrac{\pi}{2}$ y para el Oeste: entre $\phi_1=\nicefrac{\pi}{2}$ y $\phi_2=\nicefrac{3\pi}{2}$.

    \subsection{Aproximaciones del método}
    Se considera que pueden  desacoplarse de las variables $\theta$ y $\alpha^0$, por lo tanto, podemos expresar $\tilde{\omega}$ de la siguiente manera:
    \begin{equation}
        \tilde{\omega}(\theta, \alpha^0) = \omega(\theta)F(\alpha^0)
    \end{equation}    
    
    A su vez, consideremos que las amplitudes de las variaciones asociadas a $\tilde{\omega}$ son pequeñas con respecto al valor medio, por lo que se puede tomar la expansión en primer orden de la función $F(\alpha^0)$:
    \begin{equation}
        \tilde{\omega}(\theta, \alpha^0) = \omega(\theta)\big(1 + \eta(\alpha^0) \big)
        \label{eq:omega_expandido}
    \end{equation}
    \subsection{Cálculo de la diferencia de flujos}
  
    Teniendo en cuenta la Ec.\ref{eq:omega_expandido} y \ref{eq:dipolo_general}, se tiene la siguiente expresión:
    \begin{align}
        I^{obs}(\alpha^0) &= \int_{\phi_1}^{\phi_2} d\phi \int_{0}^{\theta_{max}} d\theta  \sin\theta \omega(\theta)\big(1 + \eta(\alpha^0) \big) \Phi_0 ( 1 +  \textbf{d}\cdot\hat{\bf{u}}) \label{eq:I-obs}
    \end{align}
    donde la primera parte  de la igualdad  puede simplificarse con una definición apropiada  \footnote{ $
        \text{Por simplicidad, definimos la siguiente expresión: }
      \overline{f(\theta)} = \int_{0}^{\theta_{max}} d\theta \sin\theta \omega(\theta) f(\theta)
      \label{eq:media_angular}
  $
  \noindent, donde $\overline{f(\theta)}$ es la media de la función $f(\theta)$ sobre el ángulo cenital pesado por la exposición del Observatorio $\omega(\theta)$, hasta  un ángulo máximo de $\theta_{max}$. En este trabajo se centra en eventos hasta 2 EeV, por lo que $\theta_{max}=60^o$ para los datos del Observatorio.}. Además dado que la integral sobre $\phi$ tiene el mismo valor para el Este y Oeste, se  obtiene que la primera parte se puede escribir de la siguiente forma
    \begin{align*}
        &\int_{\phi_1}^{\phi_2} d\phi \int_{0}^{\theta_{max}} d\theta \sin\theta \omega(\theta)\big(1 + \eta(\alpha^0) \big) \Phi_0 
        = \Phi_0 (1+ \eta(\alpha^0)) \overline{1}. 
    \end{align*}
    Trabajando con la segunda parte de la expresión \ref{eq:I-obs}, si  consideramos la expresión \ref{eq:dot-prod-local} del producto escalar $\textbf{d}\cdot\hat{\bf{u}}$ en coordenadas locales, e integramos el ángulo  $\phi$ entre $[\nicefrac{-\pi}{2}, \nicefrac{\pi}{2}]$ o $[\nicefrac{\pi}{2}, \nicefrac{3\pi}{2}]$, se obtiene que:

    \begin{align}
        &\int_{\phi_1}^{\phi_2} d\phi \int_{0}^{\theta_{max}} d\theta \sin\theta \omega(\theta)\big(1 + \eta(\alpha^0) \big) \Phi_0 \textbf{d}\cdot\hat{\bf{u}}=\\
        &=\Phi_0 (1+ \eta(\alpha^0)) \int_{0}^{\theta_{max}}  d\theta (\pm 2d_x(\alpha^0)\sin\theta 
        + \pi d_z(\alpha^0)\cos\theta) \label{segundo_term}
    \end{align}
    \noindent donde $+2$ corresponde al Este y $-2$ al Oeste. No hay una dependencia con la proyección del dipolo $d_y$ porque en la integral aparece el término $\int_{\phi_1}^{\phi_2}\, d\phi\, d_y(\alpha^0) \sin\theta \sin\phi $, que se anula al integrar sobre el Este y Oeste.


    %  Al integrar el ángulo  $\phi$ entre $[\nicefrac{-\pi}{2}, \nicefrac{\pi}{2}]$ o $[\nicefrac{\pi}{2}, \nicefrac{3\pi}{2}]$, el segundo término  se anula, por lo que la expresión \ref{segundo_term} queda como:
    % \begin{align*}
    %     &\int_{\phi_1}^{\phi_2} d\phi \int_{0}^{\theta_{max}}  d\theta \sin\theta \omega(\theta)\textbf{d}\cdot\hat{\bf{u}} =\\
    %     &\int_{0}^{\theta_{max}}  d\theta (\pm 2d_x(\alpha^0)\sin\theta 
    %     + \pi d_z(\alpha^0)\cos\theta)
    % \end{align*}     
    % \noindent donde $+2$ corresponde al Este y $-2$ al Oeste. 
    
     Usando la definición dada en la nota \ref{eq:media_angular} y la expresión \ref{segundo_term}, podemos reescribir la expresión \ref{eq:I-obs} y los flujos para el Este y el Oeste como:
    \begin{align*}
    I^{obs}_E&= \Phi_0 (1+ \eta(\alpha^0)) \Big( \pi\overline{1} + 2d_x(\alpha^0)\overline{\sin\theta} + \pi d_z(\alpha^0)\overline{\cos\theta}  \Big) \\
        I^{obs}_O&= \Phi_0 (1+ \eta(\alpha^0)) \Big( \pi \overline{1} - 2d_x(\alpha^0)\overline{\sin\theta}   + \pi d_z(\alpha^0)\overline{\cos\theta} \Big) \\
        I^{obs}&= \Phi_0 (1+ \eta(\alpha^0)) \Big( 2\pi\overline{1} +\pi d_z(\alpha^0)\overline{\cos\theta}  \Big)
    \end{align*}
    Ya que se busca calcular la diferencia entre los flujos provenientes del Este y del Oeste, $I^{obs}_E $ y $  I^{obs}_O $ respectivamente, esta resta queda como:
    \begin{equation*}
        I^{obs}_E -  I^{obs}_O = 4 \Phi_0 (1+ \eta(\alpha^0)) \,  d_x(\alpha^0)\overline{\sin\theta}
    \end{equation*}

    Para obtener las componentes del vector $\bf{d}$, tenemos que considerar que las mismas están en el plano x-z. Para hacer esto, consideremos a los versores $\hat{\bf{u}}_x$ y $\hat{\bf{u}}_z$ que apuntan al cenit y al Este respectivamente. Podemos obtener las proyecciones con un producto escalar con los versores en las direcciones de interés:
    \begin{equation}
        d_x(\alpha^0) \hat{x} =  (\textbf{d}\cdot\hat{\bf{u}}_x)\hat{\bf{u}}_x \rightarrow d_x(\alpha^0) = \textbf{d}\cdot\hat{\bf{u}}_x,
    \end{equation}
    donde estos versores en coordenadas ecuatoriales se escriben como:
    $\hat{\bf{u}}_z = (\alpha^0,\delta_0 )$ y $ \hat{\bf{u}}_x = (\alpha^0 + \frac{\pi}{2},0)$ \footnote{Se suma  $\frac{\pi}{2}$ para apuntar al Este, cuando el versor recorre $\nicefrac{\pi}{2}$ en ascensión recta, llega al plano del ecuador que tiene declinación $0$.}.

    % \item 
   
    % Volviendo a la expresión de $I^{obs}$, teniendo en cuenta que necesitamos  $I^{obs}_E$ y $I^{obs}_O$:
    % \begin{align*}
    %     I^{obs}_E&= \Phi_0 (1+ \eta(\alpha^0)) \Big( \pi\overline{1} + 2d_x(\alpha^0)\overline{\sin\theta} + \pi d_z(\alpha^0)\overline{\cos\theta}  \Big) \\
    %     I^{obs}_O&= \Phi_0 (1+ \eta(\alpha^0)) \Big( \pi \overline{1} - 2d_x(\alpha^0)\overline{\sin\theta}   + \pi d_z(\alpha^0)\overline{\cos\theta} \Big) 
    % \end{align*}

    % \item 
    
    % \item 
    Usando la expresión \ref{eq:product_ud} para el producto escalar en coordenadas ecuatoriales, se obtienen las componentes:
    \begin{align*}
        d_z=\textbf{d}\cdot\hat{\bf{u}}_z &= d (\cos\delta_d \cos\delta_0 \cos(\alpha^0 - \alpha_d) + \sin\delta_d  \sin\delta_0)\\
        d_x=\textbf{d}\cdot\hat{\bf{u}}_x &= d (\cos\delta_d \cos(\alpha^0 +\frac{\pi}{2} - \alpha_d) 
        = -d\cos\delta_d \sin(\alpha^0  - \alpha_d)
    \end{align*}
    
     Entonces la diferencia entre flujos queda como:
    \begin{equation}
        I^{obs}_E -  I^{obs}_O =-4d \Phi_0 (1+ \eta(\alpha^0)) \cos\delta_d \sin(\alpha^0  - \alpha_d)\overline{\sin\theta}
        \label{resta}
    \end{equation}

    Esta diferencia se debe relacionar con la variación del flujo verdadero $I(\alpha^0)$, es decir el flujo que se observaría si no existieran variaciones temporales en ascensión recta en la exposición, que implicaría $\eta(\alpha^0)=0$. 

    La variación del flujo verdadero en ascensión recta provee información sobre la componente $d_z$ del dipolo. 
    \begin{align}
        \dv{I(\alpha^0)}{\alpha^0}  & = 2\pi\Phi_0 \overline{\cos\theta} \dv{\,d_z(\alpha^0) }{\alpha^0}\\ 
        \dv{I(\alpha^0)}{\alpha^0} &= -2d\pi\Phi_0 \overline{\cos\theta}\cos\delta_d \cos\delta_0 \sin(\alpha^0 - \alpha_d) \label{total_flux}
    \end{align}

    Para llegar a la expresión \ref{resta}, hicimos la expansión hasta el primer orden de $\tilde{\omega}(\theta, \alpha^0)$ y de $\bf\Phi(\alpha, \delta)$, por lo tanto, para ser consistentes en el orden de aproximación, se desprecia el término de segundo orden de la expresión \ref{resta} que es proporcional de $\eta \cdot d$ y la expresión \ref{resta} queda:
        \begin{equation}
            I^{obs}_E -  I^{obs}_O \approx -4d \Phi_0 \cos\delta_d \sin(\alpha^0  - \alpha_d)\overline{\sin\theta}
            \label{resta_final}
        \end{equation}


    % \begin{enumerate}
    % \item 

    % \item 
    Considerando las expresiones \ref{total_flux} y \ref{resta_final}, se tiene una relación entre el flujo observado por el Observatorio  y el flujo real de RCs \footnote{
    $
        \text{Se usa la expresión:  }
        \langle f(\theta) \rangle = \frac{\overline{f(\theta)}}{\overline{1}} = \displaystyle\frac{\int_{0}^{\theta_{max}} d \theta \sin\theta \omega(\theta) f(\theta) }{\int_{0}^{\theta_{max}} d \theta \sin\theta \omega(\theta)} 
    $,  
    que es equivalente a hacer la media ponderada de todos los datos de $f(\theta)$.} %Como consideramos una expansión de $\omega(\theta)$ }:
    \begin{equation}
        I^{obs}_E -  I^{obs}_O \approx  \frac{2}{\pi \cos \delta_0} \frac{\langle\sin\theta \rangle}{\langle\cos\theta \rangle}\dv{I(\alpha^0)}{\alpha^0} 
        \label{eq:final}
    \end{equation}
   
% \end{enumerate}

\section{Estimación de la componente ecuatorial del dipolo mediante el análisis del  primer armónico}

%  Los coeficientes de Fourier en este caso se determinan con las siguientes expresiones:
% \begin{align*}
%     a_{EW} &= \frac{2}{N} \sum^N_{i=1} \cos(\alpha^0_i - \beta_i)\\
%     b_{EW} &= \frac{2}{N} \sum^N_{i=1} \sin(\alpha^0_i - \beta_i)
% \end{align*}
% donde $N$ es la cantidad de eventos en el rango de tiempo estudiado y $\beta_i=0$ si el evento proviene del Este, caso contrario $\beta_i=1$. La amplitud  $r_{EW} = \sqrt{a_{EW}^2 + b_{EW}^2}$ y la fase $\phi_{EW} = \tan^{-1}(\nicefrac{b_{EW}}{a_{EW}})$ mediante este análisis en frecuencia es posible estimar los valores $r$ y $\phi$ de la Ec.\ref{eq:dipolo_flujo}:
% \begin{align*}
%     r &= \frac{\pi \cos\delta_0}{2} \frac{\langle\cos\theta \rangle}{\langle\sin\theta \rangle} r_{EW} \\ 
%     &\text{integración} \rightarrow r_I =\frac{N}{2\pi}r \\
%     \phi &= \phi_{EW} \\
%     &\text{integración} \rightarrow \phi_I = \phi_{EW} + \frac{\pi}{2}
% \end{align*}



% \begin{equation}
%     P(\geq r_{EW}) = \exp{-\frac{N}{4}r^2_{EW}} = \exp{-\frac{N}{4} \Big ( \frac{2 \langle\sin\theta \rangle }{\pi \langle\cos\delta \rangle} \Big)^2 r^2_{1} }
% \end{equation}


% \section{Cálculo de la amplitud del dipolo para la frecuencia sidérea con el método East-West}

El objetivo del método  East - West es estimar la modulación dipolar de  $I(\alpha^0)$ a partir de la diferencia $I^{obs}_E -  I^{obs}_O$ mediante un análisis similar al método de  Rayleigh, salvo modificaciones para tener en cuenta la dirección de los eventos. 
\begin{equation}
    I^{obs}_E -  I^{obs}_O = \frac{N}{2\pi} r_{EW} \cos(\alpha^0 -  \phi_{EW}) \label{eq:ray_ew_like}
\end{equation}
La amplitud $r_{EW}$ y fase $\phi_{EW}$ obtenidas por el Método  East - West no es la amplitud del dípolo físico aunque está relacionada con la misma. 


Esta relación puede obtenerse reescribiendo la expresión \ref{eq:final}, teniendo en cuenta la proyección del dípolo físico sobre el ecuador $d_{\perp}= d\cos\delta_0$, la expresión  \ref{resta_final} y que $N \simeq 4\pi^2 \Phi_0 \overline{1} $ \footnote{Porque es la integral con respecto a los dos ángulos, $\theta$ y $\phi$}:
% \begin{align}
%     I^{obs}_E -  I^{obs}_O \approx -4 d_\perp \Phi_0 \sin(\alpha^0  - \alpha_d)\overline{\sin\theta},
% \end{align}
% si multiplicamos la expresión por una identidad y consideramos $N \simeq 4\pi^2 \Phi_0 \overline{1} $ \footnote{Porque es la integral con respecto a los dos ángulos, $\theta$ y $\phi$}:
\begin{align}
    I^{obs}_E -  I^{obs}_O \approx -4 d_\perp \frac{N}{ 4\pi^2\overline{1}} \sin(\alpha^0  - \alpha_d)\overline{\sin\theta} \frac{\overline{1}}{\overline{1}}\\
    I^{obs}_E -  I^{obs}_O \approx -4 d_\perp \frac{N}{ 4\pi^2} \sin(\alpha^0  - \alpha_d)\langle\sin\theta \rangle\\
    I^{obs}_E -  I^{obs}_O \approx -\frac{N}{2\pi} d_\perp \frac{2\langle\sin\theta \rangle }{\pi}\sin(\alpha^0  - \alpha_d) \label{ultima_ew_ray}
\end{align}
% donde N es  la cantidad de eventos a considerar. Además podemos estimar la derivada de $I(\alpha^0)$ a partir de la amplitud $r$ y la fase $\phi$ del dípolo físico:
% \begin{equation}
%     \dv{I(\alpha^0)}{\alpha^0} = r \cos(\alpha^0 - \phi),
%     \label{eq:dipolo_flujo}
% \end{equation}
Comparando las expresiones \ref{eq:ray_ew_like} y \ref{ultima_ew_ray} y considerando la ecuación \ref{eq:fourier_perp}, se puede inferir que las relaciones entre la amplitud y fase obtenidas mediante EW y el dipolo físico son las siguientes:
% \begin{align}
%     d_{\perp}  &= \frac{2\langle\sin\theta \rangle}{\pi} r_{EW} \label{dperp} \\
%     r_1 &= \frac{\pi}{2} \frac{\langle\cos\delta \rangle}{\langle\sin\theta \rangle} r_{EW} \label{r_fisico} \\ 
%     \alpha_d &= \phi_{EW} + \frac{\pi}{2} \label{phase_fisico}
% \end{align}

\begin{tabular}{@{}p{.4\linewidth}@{}p{.5\linewidth}@{}}
    \begin{align}
        d_{\perp} = \frac{2\langle\sin\theta \rangle}{\pi} r_{EW} \label{dperp} \\
        r_1   =\frac{\pi}{2} \frac{\langle\cos\delta \rangle}{\langle\sin\theta \rangle} r_{EW} \label{r_fisico}  \\
        \alpha_d = \phi_{EW} + \frac{\pi}{2} \label{phase_fisico}
    \end{align}
    &    \begin{align}
        \sigma_{x,y} = \frac{\pi}{2\langle\sin\theta \rangle} \sqrt{\frac{2}{\mathcal{N}}}\\
        \sigma   = \frac{\pi \langle\cos\delta \rangle}{2\langle\sin\theta \rangle} \sqrt{\frac{2}{\mathcal{N}}}
    \end{align}
  \end{tabular}
% Como esto es equivalente a $-r_I\sin(\alpha^0  - \alpha_d)$ por la ecuación \ref{eq:dipolo_flujo}, además de considerar la ecuación \ref{eq:fourier_perp}:
% \begin{align}
%     r  &= r_1 \frac{2\langle\sin\theta \rangle }{\pi}\\
%     r &= \frac{\pi \cos\delta_0}{2} \frac{\langle\cos\theta \rangle}{\langle\sin\theta \rangle} r_{EW} \\
%     &\Rightarrow  r_1 = \frac{\pi}{2} \frac{\langle\cos\delta \rangle}{\langle\sin\theta \rangle} r_{EW}
% \end{align}
% donde la última ecuación es la relación entre la amplitud del dipolo y la amplitud obtenida obtenida con el método East-West. 

Como en el caso del análisis de Rayleigh, la probabilidad de obtener una amplitud mayor o igual a que $r_{EW}$ a partir de una distribución isótropa es una distribución acumulada de Rayleigh:
\begin{equation}
    P(\geq r_{EW}) = \exp \Big(-\frac{N}{4}r^2_{EW}\Big) \label{p99}
\end{equation}

\subsection{Cálculo  de la amplitud del dipolo para los eventos de Todos los Disparos}

\begin{enumerate}
    \item Definimos el rango de tiempo a estudiar, para los resultados para Todos los Disparos se utilizaron los límites: 1 de Enero del 2014 hasta el 1 de Enero del 2020.
    \item Se recorre cada evento que cumpla con las siguientes características:
     \begin{itemize}
        \item Pertenezca el rango de energía a estudiar
        \item Sea un evento 6T5 con ángulo cenital menor a $60^o$
        \item Se haya registrado en el rango de tiempo seleccionado
    \end{itemize}
    En cada evento se calcula los siguientes valores:
    \begin{align}
        a_i' = \cos(X_i - \beta) \qquad
        b_i' = \sin(X_i - \beta)
    \end{align}
    el valor de $X_i$ depende la frecuencia a estudiar, la misma es igual a la ascensión recta del cenit $\alpha^0_i$ al momento del evento  si se estudia la frecuencia sidérea, en cambio para la frecuencia solar es igual al equivalente en grados de la hora local de Malargüe. El valor de $\beta$ es depende si el evento provino del Este donde $\beta=180^o$ o $\beta=0$ caso contrario.
    % Se intentó hacer un barrido de frecuencias análogo al análisis de Rayleigh pero la variable utilizada para generalizar el análisis a frecuencias arbitrarias:
    % \begin{equation}
    %     \tilde{\alpha} = 2\pi f_x t_i + \alpha_i - \alpha_i^0(t_i) \label{ra_mod}
    %   \end{equation}
    % es tal que la variable es igual a la ascensión recta del evento a estudiar y no al cenit como es el caso del EW. 
    \item Una vez corridos todos los  eventos se calculan los parámetros:
    \begin{align*}
        a_{EW} &= \frac{2}{N} \sum^N_{i=1}a_i' =\frac{2}{N} \sum^N_{i=1} \cos(X_i - \beta_i)\\
        b_{EW} &= \frac{2}{N} \sum^N_{i=1}b_i' =\frac{2}{N} \sum^N_{i=1} \sin(X_i - \beta_i)
    \end{align*}
    donde N indica la cantidad eventos considerados. La cantidad de eventos por rango de energía se muestran en la tabla \ref{tab:}.

    Con esto puedo calcular la amplitud asociada al análisis $r_{EW}$ y la fase $\phi_{EW}$:
    \begin{align*}
        r_{EW} = \sqrt{a_{EW}^2 + b_{EW}^2}\\
        \phi_{EW} = \tan^{-1}(\nicefrac{b_{EW}}{a_{EW}})
        % r &= \frac{\pi \cos\delta_0}{2} \frac{\langle\cos\theta \rangle}{\langle\sin\theta \rangle} r_{EW} 
    %     &\text{integración} \rightarrow r_I =\frac{N}{2\pi}r \\
    %     \phi &= \phi_{EW} \\
    %     &\text{integración} \rightarrow \phi_I = \phi_{EW} + \frac{\pi}{2}
    \end{align*}

    Estos valores se traducen a los valores de amplitud $r$, $d_\perp$ y fase $\phi$ del dipolo físico mediante las expresiones \ref{dperp}, \ref{r_fisico} y \ref{phase_fisico}.   Los valores $\langle\cos\delta \rangle$ y $\langle\sin\delta \rangle$ son los valores medios de estas variables en los años estudiados. 

    \item Se calcula la amplitud límite $r_{99}$ y la probabilidad  $P(r_{EW})$ utilizando la expresión \ref{p99}:
    \begin{align*}
        r_{99} &= \frac{\pi}{2} \frac{\langle\cos\delta \rangle}{\langle\sin\theta \rangle}\sqrt{\frac{4}{N}\ln(100)}\\
        d_{\perp,99} &= \frac{r_{99}}{\langle\cos\delta \rangle}    
    \end{align*}

    \item Se calculan los límites de confianza de las variables $r$,$\phi$ y $d_\perp$ mediante los densidad de probabilidad de la amplitud y fase. Las mismas se describen en el capítulo \ref{PDFs}.
    
    % \begin{itemize}
    %     \item Error asociado a la amplitud $r$ y $d_\perp$
    %     \begin{align*}
    %       \text{r} &\rightarrow  \sigma   = \frac{\pi \langle\cos\delta \rangle}{2\langle\sin\theta \rangle} \sqrt{\frac{2}{\mathcal{N}}}\\
    %       d_\perp &\rightarrow   \sigma_{x,y} = \frac{\sigma}{\langle\cos\delta \rangle}
    %     \end{align*}
    %     \item Error asociado a la fase $\phi$ de la amplitud:
    %     \begin{align*}
    %         \sigma_{\phi} = \frac{1}{r_{EW}}\sqrt{\frac{2}{\mathcal{N}}}
    %     \end{align*}
        

    % \end{itemize}

\end{enumerate}

% Por último, estos resultados se comparan con los valores obtenidos con el método EW en el trabajo \cite{Aab_2020} en frecuencia sidérea, aplicado al conjunto de eventos del disparo estándar registrados entre el 1 de Enero del 2004 y el 1 de Agosto del 2018. Para esto se ejecutó el programa implementado en el trabajo mencionado sobre los datos utilizados en el mismo, estos se obtuvieron de \emph{Publications Committee} de la colaboración Auger.


Por último, estos resultados se comparan con los valores obtenidos con el método EW en el trabajo \cite{Aab_2020} en frecuencia sidérea, aplicado al conjunto de eventos del disparo estándar registrados entre el 1 de Enero del 2004 y el 1 de Agosto del 2018. 
% Para esto se ejecutó el programa implementado en el trabajo mencionado sobre los datos utilizados en el mismo, estos se obtuvieron de \emph{Publications Committee} de la colaboración Auger.



\subsubsection{Cálculo para frecuencias  arbitrarias}

Cambiamos las variable de la ascensión recta del cenit $\alpha^0$ por
\begin{equation}
    \tilde{\alpha} = 2\pi f_x t_i  \label{ra_arb}
  \end{equation}
donde $f_x$ es la frecuencia arbitraria a estudiar y $t_i$ es el momento donde ocurre el evento a estudiar. Luego se realizan el mismo procedimiento que lo anterior para calcular el valor de la amplitud $r$.

En la siguiente sección se verifica que se obtiene los mismo resultados con esta variable general que con el valor de $\alpha^0$ para la frecuencia sidérea.

\section{Verificación del código}

\subsection{Comparación con el trabajo \cite{Aab_2020} de la colaboración}
Se verificó el código escrito en este trabajo de la siguiente manera:

\begin{enumerate}
    \item El conjunto de eventos del disparo estándar registrados entre el 1 de Enero del 2004 y el 1 de Agosto del 2018 fue analizado en el trabajo \cite{Aab_2020}.
    \item Utilizando el código y los datos de los eventos del paper \cite{Aab_2020}, obtenidos de la página del \emph{Publications Committee} de la colaboración Auger, se replicaron los datos del paper. 
    \item Luego utilizando el código escrito para este trabajo, se realizó el análisis de EW con los datos del trabajo \cite{Aab_2020}. 
    \item Finalmente se verificó que los valores obtenidos en los item 2 y 3, con  ambos códigos, sean el mismo.
\end{enumerate}

\subsection{Tabla comparando con la variable $\tilde{\alpha}$ con la ascensión recta del cenit }

Para verificar que la variable de la Ec.\ref{ra_arb} es útil para estudiar otras frecuencias, en la Tabla~\ref{tab:comp_vars} se comparan los resultados de la referencia para el rango $0.25-0.5$ EeV, los obtenidos usando la ascensión recta del cenit y los valores obtenidos con la Ec.\ref{ra_arb} en el mismo rango de energía. Se observan que los valores son comparables entre sí.

(FALTA ACTUALIZAR LA TABLA)
\begin{table}[H]
    \begin{small}
        \begin{center}
            \begin{tabular}[c]{l|l|l|l}
                                    & \cite{Aab_2020} & $\alpha^0$   & $\alpha=2\pi f_xt_i$   \\ 
                Frecuencia:         & 366.25          &  366.25      &  366.25            \\
                $d_\perp$[\%]:      & 0.60            &  0.60        &  0.60              \\
                $\sigma_{x,y}$[\%]  & 0.48            &  0.48        &  0.48              \\ 
                Probabilidad:       & 0.45            &  0.45        &  0.45              \\
                Fase[$^o$]:         & 225$\pm$64\cite{discrepancia} & 225$\pm$45   &  227$\pm$45          \\
                $r_{99}$[\%]:       & 1.5             &  1.5       &  1.5             \\
                $d_{\perp,99}$[\%]: & 1.8             &  1.8       &  1.8             \\
            \end{tabular}
        \end{center}
        \caption{Verificando la  variable $\alpha=2\pi ft$}
        \label{tab:comp_vars}
    \end{small}
\end{table}

