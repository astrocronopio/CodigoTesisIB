\begin{resumen}%
Cuando un rayo cósmico interactúa con una molécula en la parte superior de la atmósfera, se inicia un proceso en el cual se generan otras partículas secundarias. Este proceso es conocido como lluvia atmosférica extendida. Estas lluvias pueden ser detectadas sobre la superficie de la Tierra mediante varios experimentos. Este trabajo utiliza los datos recolectados por los detectores de superficie separados en 1500\,m entre sí del Observatorio Pierre Auger durante los años 2005-2020. 

Se estudian eventos obtenidos mediante distintos algoritmos de adquisición de datos. El \emph{Disparo Estándar} que alcanza eficiencia completa para eventos asociados a rayos cósmicos de energía mayor a $3\,$EeV, y el \emph{Todos los Disparos} llega a detectar, con una eficiencia del 100\%, eventos por encima de $1\,$EeV. El primer disparo contiene eventos registrados desde el año 2005 y el segundo disparo empezó funcionar desde el 2013. 

Las condiciones atmosféricas como la presión (P), la temperatura (T) y la densidad ($\rho \propto \nicefrac{P}{T}$) afectan el desarrollo de la lluvia a través de la atmósfera. Las variaciones de estas condiciones inducen una modulación en la cantidad de eventos registrados por el Observatorio Pierre Auger. Mediante un estudio hecho por la Colaboración sobre eventos del Disparo Estándar, se corrigió está modulación en la señal medida por el Observatorio. En este trabajo extendimos el periodo de tiempo analizado de esta modulación, y se observó que los parámetros obtenidos son comparables con la reconstrucción oficial. También se estudia la modulación en los datos de Todos los Disparos, y se realiza una corrección sobre el mismo conjunto de datos usando los parámetros obtenidos por este trabajo.

Se  estudian las modulaciones en distintas frecuencias mediante el análisis en Rayleigh, y se propone una variable generalizada para hacer un barrido en frecuencias con el método de East-West. Se obtienen resultados de la modulación en ascensión recta para distintos rangos de energía y se comparan con resultados reportados por la Colaboración Pierre Auger.





\end{resumen}


% \begin{nemombyky}%
% Mbyjakua\'ape (\emph{astronomía} karaiñe'\~eme) ojeikuaase mba\textquotesingle e oik\'ova umi mba\textquotesingle e  michĩ yv\'agagui o\'uva (\emph{rayos cósmicos} karaiñe'\~eme) oguah\~evove amo yvatetépe (\emph{atmósfera} karaiñe'\~eme). Ombok\'aramo tuminguaave\textquotesingle \~yty (\emph{conjunto de átomos o molécula}  karaiñe'\~eme ) yvatetépe oĩva, oñepyr\~u ojapo het\~a umi tuminguaave\textquotesingle \~yjokaku\'era (\emph{partículas}  karaiñe'\~eme ) op\'arupi. Ko\textquotesingle a       ha\textquotesingle e  h\'ina peteĩ ama guasu tuminguaave\textquotesingle \~yjoka rehegua ( \emph{lluvia atmosf\'erica extendida} karaiñe'\~eme). Umi ama guasuku\'era tuichaterei ha ikatu eñeña\textquotesingle ã yvy ári op\'arupi. Mend\'osape oĩ peteĩ mba\textquotesingle etuicha h\'erava \emph{Pierre Auger} Mbyjañama\textquotesingle \~eha\~gua (\emph{Observatorio Pierre Auger}) oña'\~ava ko ama. Ko\textquotesingle  ape romba\textquotesingle  ap\'ota umi ama ko mbyjañama\textquotesingle \~eha\~gua oña\textquotesingle \~ava\textquotesingle  kue 2005-guive 2018-peve. Mba\textquotesingle \'eichapa umi amaku\'era oguah\~e yvy \'ari ikatu ojuavy hakúramo (T, \emph{temperatura} karaiñe\textquotesingle \~eme) tér\~a  poh\'yiramo pe pytundyry mbyjañama\textquotesingle \~eha\~gua áripe ($\rho$, \emph{densidad} h\'erava karaiñe\textquotesingle \~eme). Ko mbyjañama'\~eha\~gua ojapova\textquotesingle ekue peteĩ tembiapo ha ko\textquotesingle ape rojapojey up\'eva roikuaaha\~gua umi papapo oñenoh\~eva\textquotesingle ekue oiko gueteri ko'\~anga peve, ha rotopa kóva oikópa añetete.
% \end{nemombyky}



%%% Local Variables: 
%%% mode: latex
%%% TeX-master: "template"
%%% End: 
