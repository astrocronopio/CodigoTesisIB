%%%%%%%%%%%%%%%%%%%%%%%%%%%%%%%%%%%%%%%%%%%%%%%%%%%%%%%%%%%%%%%%%%%%%%%%%%%%%%%%%%%%%%%%%%%%%%%%%%%%%%%%%
%%%%%%%%%%%%%%%%%%%%%%%%%%%%%%%%%%%%%%%%%%%%%%%%%%%%%%%%%%%%%%%%%%%%%%%%%%%%%%%%%%%%%%%%%%%%%%%%%%%%%%%%%
% 								ORDEN DE ESTE CAPITULO								
%%%%%%%%%%%%%%%%%%%%%%%%%%%%%%%%%%%%%%%%%%%%%%%%%%%%%%%%%%%%%%%%%%%%%%%%%%%%%%%%%%%%%%%%%%%%%%%%%%%%%%%%%
%%%%%%%%%%%%%%%%%%%%%%%%%%%%%%%%%%%%%%%%%%%%%%%%%%%%%%%%%%%%%%%%%%%%%%%%%%%%%%%%%%%%%%%%%%%%%%%%%%%%%%%%%
%
% INTRODUCCION
% METODOS
% ARCHIVOS DE DATOS Y SUS DIFERENCIAS
% RESULTADOS PARA ANISOTROPÍAS EN RA
% ---> 8 EeV PARA LOS ICRC
% ---> CARACTERISTICAS
% ------> ICRC 2015
% ------> ICRC 2018

% RESULTADOS PARA ANISOTROPÍAS EN RA PARA ALL TRIGGERS
% ---> 1 EeV 
% ---> CARACTERISTICAS
% ------> ALL TRIGGERS OSCAR
% ------> ALL TRIGGERS HERALD
% ---> COMPARACION CON LOS ICRC  (ANEXO)
% ------> ALL TRIGGERS OSCAR
% ------> ALL TRIGGERS HERALD

% ---> 1-2 EeV 
% ---> CARACTERISTICAS
% ------> ALL TRIGGERS OSCAR
% ------> ALL TRIGGERS HERALD
% ---> COMPARACION CON LOS ICRC  (ANEXO)
% ------> ALL TRIGGERS OSCAR
% ------> ALL TRIGGERS HERALD

% ---> 4-8 EeV 
% ---> CARACTERISTICAS
% ------> ALL TRIGGERS OSCAR
% ------> ALL TRIGGERS HERALD
% ---> COMPARACION CON LOS ICRC  (ANEXO)
% ------> ALL TRIGGERS OSCAR
% ------> ALL TRIGGERS HERALD



% ---> 8 EeV 
% ---> CARACTERISTICAS
% ------> ALL TRIGGERS OSCAR
% ------> ALL TRIGGERS HERALD
% ---> COMPARACION CON LOS ICRC  (ANEXO)
% ------> ALL TRIGGERS OSCAR
% ------> ALL TRIGGERS HERALD



\subsection{Energía por encima de 1 EeV}

\begin{figure}[H]
	\centering
	\includegraphics[width=0.8\textwidth]{../Anisotropia/AllTriggers_Oscar_1EeV_Eraw_Eraw_hex.png}
	\caption{Comparando el primer armónico en ascensión recta para el archivo de Oscar}
\end{figure}

\begin{figure}[H]
	\centering
	\includegraphics[width=0.8\textwidth]{../Anisotropia/AllTriggers_Herald_1EeV_Eraw_Eraw_hex.png}
	\caption{Comparando el primer armónico en ascensión recta para el archivo del Herald}
\end{figure}


Para corroborar los parámetros del clima, primero calculé las tasas de eventos de ambos conjuntos de datos para energías mayores a 1  EeV, donde obtuve los siguientes gráficos Fig.\ref{fig:rate_daily_oscar_1EeV} y \ref{fig:rate_daily_herald_1EeV}.  Lo que me resulta extraño es que, lo que tengo entendido es que para este conjunto de datos, se vea la modulación anual para  1 EeV, siendo que los triggers son ``100\%''? eficientes para estas energías. Para el archivo del Herald, también se ve esta modulación.



\textbf{P R E F A C T O R E S}
\begin{figure}[H]

	\begin{subfigure}[b]{0.5\textwidth}
	\centering
	\includegraphics[width=\textwidth]{../Anisotropia/daily_rate_AllTriggers_oscar_1EeV.png}
	\caption{Archivo de Oscar} 	\label{fig:rate_daily_oscar_1EeV}
	\end{subfigure}%
\hfill
	\begin{subfigure}[b]{0.5\textwidth}
	\centering
	\includegraphics[width=\textwidth]{../Anisotropia/daily_rate_AllTriggers_herald_1EeV.png}
	\caption{Archivo del Herald} 	\label{fig:rate_daily_herald_1EeV}
	\end{subfigure}
	\caption{Tasa de eventos diaria por encima de 1 EeV para los datos de todos los disparos.}
\end{figure}

Después calculé los parámetros del clima para energía mayores a 1 EeV. Para el archivo de Oscar obtuve la Fig.\,\ref{fig:parameters_oscar_1EeV}. Los comparé con el paper del weather del main array, para ver si dan algo razonable. Tengo que volver a verificar el $b_\rho$, ya que me había pasado lo mismo en la tesis de licenciatura por la convergencia del ajuste. En cambio, los otros dos dan cosas parecidas.

\begin{figure}[H]
	\begin{subfigure}[b]{0.5\textwidth}
	\includegraphics[width=\linewidth]{../Anisotropia/ap_Oscar_above_1EeV.png}
	\caption{Parámetro $a_P$ }
	\label{fig:ap_oscar_1EeV}
	\end{subfigure}%
	\hspace{\fill}
	\begin{subfigure}[b]{0.5\textwidth}
	\includegraphics[width=\linewidth]{../Anisotropia/arho_Oscar_above_1EeV.png}
	\caption{Parámetro $a_{\rho}$ }
	\label{fig:arho_oscar_1EeV}
	\end{subfigure}%
	\hspace{\fill}
	\begin{subfigure}[b]{\textwidth}
	\centering
	\includegraphics[width=0.5\linewidth]{../Anisotropia/brho_Oscar_above_1EeV.png}
	\caption{Parámetro  $b_\rho$	 }
	\label{fig:brho_oscar_1EeV}
	\end{subfigure}%
	\caption{Parámetros de la modulación del clima considerando los datos para todos los disparos de Oscar. Los mismos se comparan con los ajustes obtenidos en \cite{aab2017impact}.}\label{fig:parameters_oscar_1EeV}
\end{figure}

