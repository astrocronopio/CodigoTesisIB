Fecha: 13/05/2020
Comentarios:
\begin{itemize}
	\item sobre la selecci=C3=B3n de los eventos: cuando vamos a energias bajas, por ej hasta 1 EeV, hay eventos que disparan pocas estaciones, de modo que no podemos permitir que haya estaciones apagadas cerca de la de mayor se=C3=B1al para asegurarnos que estamos haciendo una buena reconstruccion, o sea hay que considerar solo eventos 6T5, toda la corona activa. 
\item Cuando analizamos eventos de mayores energias, en particular arriba de 4EeV, estos hacen diparar tipicamente mas de 5-6 detectores, de modo que se puede permitir que uno de
los detectores de la corona este apagado sin afectar demasiado la reconstruccion, solo ahi es que se consideran los eventos 5T5.
Para calcular los coeficientes del weather, queremos los mejores eventos, asi que tambien se usan solo los 6T5 (no es tan importante ganar un poquito mas de estadistica
arribade 4 EeV para eso). El parametro ib es el el de bad period? Decis que es irrelevante porque ya filtras esos eventos durante la selecci=C3=B3n de eventos?

\item La aceptancia es la eficiencia de disparo, que que es casi 1 para eventos arriba de 1 EeV y hasta 60 grados cuando usamos el dataset con todos los disparos. No se que archivo es el energy filter Alltriggers.sh. Para que lo usas?

\item En resumen usa los cortes 6T5 y theta<60 para anisotropias y para weather correction. El corte de quality weather flag solo se usa para seleccionar los eventos
para calcular las correcciones del weather. NO HAY que usarlo para seleccionar los datos para analisis de anisotropias. Si sacas esos eventos podes estar introduciendo efectos espurios (porque no se descartan los hexagonos en los momentos que no funcionan los registros del weather). Tal vez la correccion de weather que se les hace a esos eventos no es tan precisa como la de los demas, pero es mejor que nada.

\item Ahi entiendo que ya hay algo para corregir. Los resultados nuevos son un poco raros, en siderea desaparecio toda la se=C3=B1al cuando pones pesos 1, y despues crece algo con los pesos. Revisa con los cortes bien puestos. Pone una tabla con amplitud y fase con y sin peso tambien.
\end{itemize}
