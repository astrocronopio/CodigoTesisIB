\input{Preamblev2.sty}

\begin{document}
%%%%%%%%%%%%%%%%%%%%%%%%%%%%%%%%%%Título%%%%%%%%%%%%%%%%%%%%%%%%%%%%%%%%%%%%%%
%%%%%%%%%%%%%%%%%%%%%%%%%%%%%%%%%%%%%%%%%%%%%%%%%%%%%%%%%%%%%%%%%%%%%%%%%%%%%%

\title{Modulación del clima para el archivo de todos los disparos en el rango 2014-2020}
\author{Evelyn~G.~Coronel}

\affiliation{
Tesis de Maestría en Ciencias Físicas\\ Instituto Balseiro\\}

\date[]{\lowercase{\today}} %%lw para lw, [] sin date

%\begin{abstract}

%\end{abstract} 
\maketitle
%%%%%%%%%%%%%%%%%%%%%%%%%%%%%%%%%%%%%%%%%%%%%%%%%%%%%%%%%%%%%%%%%%%%%%%%%%%%%%%%%%%

La selección de los eventos genera dos conjuntos de datos: uno para el análisis de anisotropía en el bin 1 EeV - 2 EeV, y el segundo de los eventos con energía mayor a 1 Eev para obtener los parámetros del clima. En esta selección se tiene en cuentan los eventos de $\theta < 60^o$ \footnote{el archivo que bajo de \url{http://ipnwww.in2p3.fr/~augers/AugerProtected/herald.php} }, como también  los mismos que no se encuentren en un periodo de mala adquisición datos, este parámetro se denomina $ib$ de los \textbf{eventos del herald}. Este periodo consiste en momento donde el obsevatorio no recibe datos de las estaciones de clima o de los hexágonos. 

El parámetro de $ib$ de los \textbf{datos del clima} es irrelevante durante el proceso de filtrar eventos. Entra en juego cuando hago el análisis del clima, donde desecho los eventos que fueron recabados durante \emph{bad weather} \textbf{y} no fueron filtrados ya antes. 


\section{Pesos de los hexágonos}

Para constatar que no exista ninguna anomalía en los pesos de los hexágonos, se realiza el cálculo de los mismos para tres frecuencias de referencia para el análisis de anisotropías.  Los pesos se muestran en la Fig.\,\ref{fig:wei_14_20}. El rango de tiempo en el que se calculan estas curvas es entre 1 de Enero del 2014 y el 1 de Enero del 2020.

\begin{figure}[H]
	\centering
	\includegraphics[width=0.5\textwidth]{weigth2014-2020_jan.png} 	
	\caption{Pesos de los hexágonos}
	\label{fig:wei_14_20}
\end{figure}

\section{Anisotropía}
El archivo de de todos los disparon empieza el Mon, 1 July 2013 12:05:08 GMT \footnote{$1372680308$}. Para trabajar en una cantidad entera de años, se trabaja a partir del  Thur, 1 January 2014 12:00:00 GMT \footnote{$1388577600$} y hasta el Thursday, 1 January 2020 12:00:00 GMT \footnote{$1577880000$}.  En este rango se tiene la tasa de eventos por día que se muestra en la Fig.\,\ref{tasa_total_diaria}.

% En el rango $1372680308$ \footnote{Mon, 1 July 2013 12:05:08 GMT} y $1388577600$ \footnote{Thur, 1 January 2014 12:00:00 GMT}, la tasa de eventos del archivo $\text{All Triggers}$, tenía una tasa de eventos por debajo de lo normal. Por esto, se utiliza los eventos a partir del  1388577600. La tasa de eventos que se utiliza se puede ver a continuación:

\begin{figure}[H]
	\centering
	\includegraphics[width=0.5\textwidth]{rate_total.png}
	\caption{Tasa  de eventos en el rango de tiempo a trabajar}
	\label{tasa_total_diaria}
\end{figure}

\subsection{Lista detallada de los filtros aplicados de datos del herald}

\subsubsection{Datos para el análisis de anisotropía}
Esta sección muestra los filtros para los datos del análisis de anisotropía en el rango 1 EeV - 2 EeV.

\begin{enumerate}
	\item Energía entre  [1 EeV , 2 EeV)
	\item Rango de tiempo:
	\begin{itemize}
		\item[-] Inicial:1388577600 \\ (Thursday, 1 January 2014 12:00:00 GMT)
		\item[-] Final: 1577880000  \\ (Thursday, 1 January 2020 12:00:00 GMT)
	\end{itemize}
	\item Sectancia:  $\theta < 60^o$
	\item 6T5
	\item $ib=1$ Bad period flag. Un valor de 1 indica un buen periodo
\end{enumerate}

Con estos filtros se tienen $1\,092\,753$ eventos

\subsubsection{Datos para el cálculo de las correcciones del clima}

Estos son los filtros para los datos a utilizar para el cálculo de los parámetros del clima:

\begin{enumerate}
	\item Eventos con valor de señal de $S_{38}$\footnote{Valor de S38 sin la correccón del clima del paper del 2017} por encima de  $5.36\,\text{VEM}$. Este valor corresponde a $\sim 1\,$ EeV  en VEM.
	\item Rango de tiempo:
	\begin{itemize}
		\item[-] Inicial:1388577600 \\ (Thursday, 1 January 2014 12:00:00 GMT)
		\item[-] Final: 1577880000  \\ (Thursday, 1 January 2020 12:00:00 GMT)
	\end{itemize}
	\item Sectancia:  $\theta < 60^o$
	\item $iw<4$ (weather quality flag)
	\item 6T5
	\item $ib=1$ Bad period flag del herald.  Un valor de 1 indica un buen periodo
	\item $ib=1$ Bad period flag de los datos del clima. Un valor de 1 indica un buen periodo
\end{enumerate}


Con estos filtros se tienen $1\,208\,615$ eventos, con una tasa de eventos que se muestra en la Fig.\,\ref{tasa_total_diaria_ajuste_weather}. En la figura se observa que utilizando el corte en la señal de S38 sin corregir por la modulación del clima del herald \footnote{Las correcciones se calcularon para el archivo del disparo estándar} se observa una modulación anual.

\begin{figure}[H]
	\centering
	\includegraphics[width=0.5\textwidth]{rate_total_ajuste_weather.png}
	\caption{Tasa  de eventos en el rango de tiempo a trabajar para el ajuste de los parámetros del clima.}
	\label{tasa_total_diaria_ajuste_weather}
\end{figure}


\subsection{Análisis en frecuencia}

\begin{figure}[H]
	\centering
	\includegraphics[width=0.5\textwidth]{2019_AllTriggers_1_2_EeV_con_vs_sin_peso.png}
	\caption{Análisis en frecuencia en ascensión recta en rango 1 EeV - 2 EeV}
	\label{fig:consin}
\end{figure}


\section{Corrección del clima}

\begin{figure}[H]
	\centering
	\includegraphics[width=0.5\textwidth]{rate_Ajuste.png}
\end{figure}



\begin{figure}[H]
	\centering
	\includegraphics[width=0.5\textwidth]{ap_6t5.png}
	\caption{Parámetro de clima $a_P$ calculado para la corrección del archivo de todos los disparos}
\end{figure}

\begin{figure}[H]
	\centering
	\includegraphics[width=0.5\textwidth]{arho_6t5.png}
	\caption{Parámetro de clima $a_\rho$ calculado para la corrección del archivo de todos los disparos}
\end{figure}

\begin{figure}[H]
	\centering
	\includegraphics[width=0.5\textwidth]{brho_6t5.png}
	\caption{Parámetro de clima $b_\rho$ calculado para la corrección del archivo de todos los disparos}
\end{figure}


\end{document}

