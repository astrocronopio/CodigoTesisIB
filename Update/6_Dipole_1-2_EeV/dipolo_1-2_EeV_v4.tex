	\section{Características del conjunto de datos} \label{specs}
	

	Además de los filtros aplicados mencionados en la sección \ref{filtro}, se aplican filtros adicionales sobre la energía y el rango de tiempo. Para estudiar los eventos en esta sección, consideramos los eventos entre 1\,EeV y 2\,EeV de energía y que ocurrieron entre las 12:00:00 GMT del 1 de enero de 2013 y las 12:00:00 GMT del 1 de enero de 2020. Se  eligió ese rango de tiempo, ya que el registro de eventos más reciente al que se tuvo para hacer este trabajo termina el 1 de Enero del 2020  a las 11:59:43 GMT, además de para estudiar una cantidad entera de años, se optó por considerar los eventos desde el 1 de Enero del 2013 a las 12:00:00 GMT.

	Un resumen de todos los filtros aplicados se encuentra a continuación
		\begin{enumerate}
			\item Son eventos obtenidos mediante todos los disparos.
			\item Energía entre  [1 EeV , 2 EeV)
			\item Rango de tiempo:
			\begin{itemize}
				\item[-] Inicial:1388577600 (Jueves, 1 de Enero de 2014 12:00:00 GMT)
				\item[-] Final: 1577880000  (Jueves, 1 de Enero de 2020 12:00:00 GMT)
			\end{itemize}
			\item Ángulo cenital $\theta < 60^o$
			\item 6T5
			\item $ib=1$ Bad period flag. Un valor de 1 indica un buen periodo
		\end{enumerate}
	Aplicando estos filtros, se tienen $1\,081\,844$ eventos para estudiar en este rango de energía.
			
			\begin{figure}[H]
				\centering
				\includegraphics[width=0.75\textwidth]{weights_2013_2020.png}
				\caption{Variaciones de los hexágonos para frecuencias características en rango mencionado. }
			\end{figure}

\section{Pesos de los eventos para frecuencias de referencia}

Se toman las frecuencia anti-sidérea ($f_a=364.25\,$ciclos), solar ($f_{Solar}= 365.25\,$ciclos) y sidérea ($f_{sid}= 366.25\,$ciclos) como referencia para obtener una aproximación a primer orden del análisis en frecuencias. A cada una de estas frecuencias, se ajusta una función del tipo  $f(x)=a\cdot \cos{(\alpha-\phi)} + 1$, con el se busca aproximar la amplitud $a$ y el desfase $\phi$ de las curvas de los pesos en función de la ascensión recta $\alpha$. Los ajustes se observan en las Figs. \ref{fig:ajuste_antisiderea}, \ref{fig:ajuste_solar} y \ref{fig:ajuste_siderea}.


\subsection{Gráficos de los ajustes}


\begin{figure}[H]
\begin{subfigure}{.5\textwidth}
	\centering
	\includegraphics[width=\linewidth]{eventos_RA_ajuste_cos_antisiderea_v2.png}
	\caption{Frecuencia anti-sidérea}
	\label{fig:ajuste_antisiderea}
\end{subfigure}%
\begin{subfigure}{.5\textwidth}
	\centering
	\includegraphics[width=\linewidth]{eventos_RA_ajuste_cos_solar_v3.png}
	\caption{Frecuencia solar}
	\label{fig:ajuste_solar}
\end{subfigure}\\
\centering
\begin{subfigure}{.5\textwidth}
	\centering
	\includegraphics[width=\linewidth]{eventos_RA_ajuste_cos_siderea_v2.png}
	\caption{Frecuencia sidérea}
	\label{fig:ajuste_siderea}
\end{subfigure}%
\caption{Ajuste de los pesos de los eventos para varias frecuencias a primer orden en ascensión recta}
\end{figure}

	
\subsection{Tabla comparando los ajustes:}
		
		\begin{table}[H]
		\centering
		\begin{tabular}{c|c|c|c}
					& Anti-sidérea			& Solar 				& Sidérea\\ \hline
		Amplitud $a$& $0.0109\pm 0.0003 $ 	&	$0.0038 \pm 0.0003$	&  $0.0047\pm 0.0007$		\\
		Fase $\phi$ & $15    \pm 1$ 		&   $360 \pm 5   $ 		&  $31    \pm 8    $ 		\\
		\end{tabular}
		\caption{Parámetros obtenidos del ajuste a primer orden en $\alpha$ sobre los pesos.}
		\end{table}


\section{Gráfico de la anisotropía}
		
	\subsubsection{Análisis de anisotropías en ascensión recta}
		
		\begin{figure}[H]
			\centering
			\includegraphics[width=\linewidth]{pesos_sin_con_1_2_EeV.png}
			\caption{Anisotropía en función de la frecuencia, se comparan los análisis sin los pesos y con los pesos de los hexágonos}
		\end{figure}
		
		
		\begin{table}[H]
		\centering
		\begin{tabular}{c|c|c|c|c|c}
					& Solar (sin peso)	& Solar (con peso)	&& Sidérea (sin peso) 	& Sidérea (con peso)	 \\ \hline
		Fase $\phi$ & 251	    		& 288	    		&& 289				& 335				\\
		Amplitud $r$& 0.0061	    	& 0.0038	  		&&0.0018		& 0.0039			\\
		\end{tabular}
		\caption{Comparación de los parámetros de fase y amplitud para las frecuencias sidérea y solar, analizando sin pesos y con los pesos de los hexágonos con el análisis de Rayleigh}
		\end{table}


\subsubsection{Bineado de eventos }


Considerando que estamos trabajando con la frecuencia solar al hacer el análisis con pesos, se obtiene la siguiente distribución de eventos en función de su ascensión recta.
\begin{figure}[H]
	\centering
	\includegraphics[width=\linewidth]{eventos_clasificados_por_RA_v4.png}
	\caption{Distribución de la cantidad relativa de eventos en función de la ascensión recta.}
\end{figure}

% Si realizamos un ajuste de una función del tipo $f(x) = a\cdot\cos{(x - \phi)}) + 1 $, se obtiene los siguientes valores
		
% 	Fase $\phi$ : $288(60)^o$  \\
% 	Amplitud $a$: 0.002(2)  	\\

