\section{Anisotropías  }
\begin{table}[H]
\centering
\begin{tabular}{l|c|c}
				& Con Peso 	& Sin peso 		\\ \hline
Frecuencia:		& 366.25 	& 366.25 		\\
Fase:			& 329.865 	& 292.312		\\
$P_{99}$:		& 0.76398\%	& 26.6838 \% 	\\
$r_{99}$:		& 0.004676 	& 0.00243515	\\
\end{tabular}
\caption{Fase, $r_{99}$ y $P_{99}$ del análisis de anisotropía entre en 1 de Enero del 2014 y el 1 de Enero del 2020}
\end{table}


\begin{figure}[H]
	\centering
	\includegraphics[width=0.8\linewidth]{../report_4_12_05_2020/2019_AllTriggers_1_2_EeV_con_vs_sin_peso.png}
	\caption{Anisotropía para el intervalo 2014-2020}
	\label{fig:anis}
\end{figure}

En la Fig.\ref{fig:zoom} se muestra el pico que se presenta en  el intervalo de energía entre 1 EeV - 2 EeV, cercano a la frecuencia sidérea. El pico tiene un máximo para un período de $366.21$. En la Tabla.\,\ref{tabla:pico} se muestran los valores de la fase, $r_{99}$ y $P_{99}$ para el periodo anterior.

\begin{figure}[H]
	\centering
	\includegraphics[width=0.5\linewidth]{zoom_anis.png}
	\caption{Zoom en el pico de anisotropía cercana para la frecuencia sidérea para el intervalo 2014-2020}
	\label{fig:zoom}
\end{figure}



\begin{table}[H]
\centering
\begin{tabular}{l|c|c|c|c}
				& Con Peso 		& Sin peso 		& Con Peso 		& Sin peso 		\\ \hline
Frecuencia:		& 366.21 		& 366.21 		& $\sim$366.505 & 366.506 		\\
Fase:			& 151.032 		& 121.695		& $\sim$190 	& 73.8188		\\
$P_{99}$:		& 0.289882\%	& 46.9691 \% 	& $\sim$96\%	& 0.24013 \% 	\\
$r_{99}$:		& 0.00512146	& 0.0018417		& $\sim$0.0006	& 0.00520328	\\
\end{tabular}
\caption{Fase, $r_{99}$ y $P_{99}$ del análisis de anisotropía entre en 1 de Enero del 2014 y el 1 de Enero del 2020}
\label{tabla:pico}
\end{table}


\section{Ajuste a orden 0 de la variación de hexágonos y pesos}

Para verificar los valores de amplitud y fase en la frecuencia sidérea, se ajusta una función del tipo $f(x) = a\cos{(2\pi\omega x + \phi)} +c$ a la variación de los hexágonos por ángulos de ascensión recta, así como también a la variación de los pesos de los hexágonos en ascensión recta. La variación y el ajuste puede verse en las Figs.\ref{fig:pesos_ajuste} y \ref{fig:pesos_hexagonos}.

\begin{figure}[H]
	\centering
	\includegraphics[width=0.5\linewidth]{ajuste_pesos.png}
	\caption{Pesos de los hexágonos para la frecuencia sidérea en el periodo 2014-2020}
	\label{fig:pesos_ajuste}
\end{figure}


\begin{figure}[H]
	\centering
	\includegraphics[width=0.5\linewidth]{ajuste_hexagonos.png}
	\caption{Hexágonos para la frecuencia sidérea en el periodo 2014-2020}
	\label{fig:pesos_hexagonos}
\end{figure}


Los valores de los ajustes, comparados con el análisis de Rayleigh se muestran en la Tabla\,\ref{tabla:ajuste_orden_cero}. SE observa que el valor de la amplitud para el caso de la variación de los pesos es más cercana al que se obtuvo en el análisis de Rayleigh. Esto puede deberse que los pesos están normalizados por la integral de todos los hexágonos dada un frecuencia, por lo que si existe alguna constante multiplicativa en la cantidad de hexágonos, la amplitud la tabla para la primera columna puede no ser igual a la segunda columna.

\begin{table}[H]
\centering
\begin{tabular}{l|c|c|c}
				& Hexágonos 				& Pesos						& Con peso \\ \hline
Figura:			& \ref{fig:pesos_hexagonos} &\ref{fig:pesos_ajuste}		&\ref{fig:zoom} \\
Fase(Mínimo):	& $\sim 317$ 				& $\sim 317$				&329.865	\\
Amplitud:		& 0.00969282 				& 0.0047421 				&0.004676\\
\end{tabular}
\caption{Fase y amplitud del ajuste a primer orden en los hexágonos y  pesos para la frecuencia sidérea}
\label{tabla:ajuste_orden_cero}
\end{table}
