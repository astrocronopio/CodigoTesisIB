%%%%%%%%%%%%%%%%%%%%%%%%%%%%%%%%%%%%%%%%%%%%%%%%%%%%%%%%%%%%%%%%%%%%%%%%%%%%%%%%
%\documentclass[12pt,papel,twoside]{ibtesis}
% \documentclass[12pt]{ibtesis}

% \documentclass[11pt,papel,oneside,singlespace]{ibtesis}
\documentclass[12pt,papel,oneside]{ibtesis}
% \documentclass[12pt,papel,preprint,singlespace,oneside]{ibtesis}

%%%%%%%%%%%%%%%%%%%%% Paquetes extra %%%%%%%%%%%%%%%%%%%%%%%%%%%%%%%%%%%%%%%%%%%
% Por conveniencia: aqu\'{\i} puede cargar todos los paquetes y definir los comandos 
% que necesite
\usepackage{ibextra}
\usepackage[utf8]{inputenc}
\usepackage{subcaption}  % Enable figure captions or figure notes
\usepackage{float}
\usepackage{nicefrac}
\usepackage{mathtools}
\usepackage{textcomp}

\usepackage{multirow}
\usepackage{amsfonts}

\usepackage{hyperref}
\usepackage{tablefootnote}

\usepackage{scalerel}
\usepackage{physics}


\usepackage{xparse}
\let\realItem\item % save a copy of the original item
\makeatletter
\NewDocumentCommand\myItem{ o }{%
   \IfNoValueTF{#1}%
      {\realItem}% add an item
      {\realItem[#1]\def\@currentlabel{#1}}% add an item and update label
}
\makeatother

\usepackage{enumitem}    
\setlist[enumerate]{
    before=\let\item\myItem%,       % use \myItem in enumerate
    %label=\textnormal{(\arabic*)}, % format the label
    %widest=(2')                    % set the widest label
}

%%%%%%%%%%%%%%%%%%%%%%%%%%%%%%%%%%%%%%%%%%%%%%%%%%%%%%%%%%%%%%%%%%%%%%%%%%%%%%%%
%%%%%%%%%%%%%%%%%%%%% Informacion sobre la tesis %%%%%%%%%%%%%%%%%%%%%%%%%%%%%%%
\title{Análisis de las direcciones de arribo de rayos cósmicos de ultra-alta energía en el Observatorio Pierre Auger}
\author{Evelyn~G.~Coronel}
\director{Dra.~Silvia Mollerach}
%\codirector{Dr.~J.~Otro m\'{a}s}b
\carrera{Tesis de Maestría en Ciencias F\'{\i}sicas}
\grado{Maestrando}
\laboratorio{Partículas y Campos -- Centro At\'{o}mico Bariloche}
\jurado{Dr.~Diego~Harari (Instituto Balseiro)}

\palabrasclave{Rayos Cósmicos, Análisis de datos, Instituto Balseiro}
%\keywords{Cosmic Rays, Data Analysis, Balseiro Institute}
%\neembaeguasu{Mba'e michĩ yvágagui ouva, Mbo'ehaoguasu Balseiro}
% Si queremos poner la fecha manualmente:
% \date{Diciembre de 2099}

%%%%%%%%%%%%%%%%%%%%%%%%%%%%%%%%%%%%%%%%%%%%%%%%%%%%%%%%%%%%%%%%%%%%%%%%%%%%%%%%
\titlepagefalse % Si no quiere compilar la portada descomente esta linea
%\includeonly{apendices} % Compilar s\'{o}lo estos archivos 
%\graphicspath{{/h}} % Lugar donde encontrar las figuras generales (se puede poner uno en cada cap{\'{\i}}tulo)
%%%%%%%%%%%%%%%%%%%%%%%%%%%%%%%%%%%%%%%%%%%%%%%%%%%%%%%%%%%%%%%%%%%%%%%%%%%%%%%%


\setcounter{tocdepth}{6}
\setcounter{secnumdepth}{6}
\begin{document}


\chapter{Método East-West}

El método de Rayleigh se basa ajustar la tasa de eventos en función de la ascensión recta mediante una función armónica. El mismo permite calcular la amplitud de la anisotropía para distintos armónicos, su fase y la probabilidad de detectar la misma señal debido a fluctuaciones de una distribución isótropa de RCs. 

La dificultad en utilizar el método Rayleigh recae en procesamiento de los datos: efectos del clima,  variaciones de la área del Observatorio y al sensibilidad de los instrumentos deben tenerse en cuenta.  Los efectos mencionados deben ser corregidos de la tasa de eventos medida, ya que los mismos inducen modulaciones espurias en la tasa de eventos.

En el método East - West consiste en el ajuste de una función armónica a la diferencia entre las tasas de eventos provenientes del Este y del Oeste. Si se consideran que las modulaciones espurias producidas por los efectos atmosféricos y sistemáticos son las mismas en ambas direcciones, la diferencia de tasas remueve estos efectos sin realizar correcciones adicionales. Una desventaja de este método es que su sensibilidad es menor que el método de Rayleigh \cite{taborda}.


% (?????????)La exposición del observatorio en un momento dado es la misma para el Este como para el Oeste, si se considera que la misma depende del ángulo cenital $\theta$ solamente, y no del ángulo azimutal $\phi$. \footnote{Tengo que agregar un apéndice explicando las coordenadas locales.}

\section{Descripción de una anisotropía dipolar}


\section{Descripción formal del método East-West}

    % \item Un forma de obtener flujo de eventos $I(\alpha)$  para un $\alpha$ dado es la siguiente:
    %     \begin{equation}
    %         I(\alpha) = \int_{\delta_{min}}^{\delta_{max}} d\delta \cos \delta \dv{N(\alpha,\delta)}{\Omega}
    %         \label{eq:i_alpha_phi}
    %     \end{equation}
    % \noindent donde $\Omega$ es el ángulo sólido en la esfera celeste expresada en las coordenadas ecuatoriales.

    % \item Considerando que la distribución de direcciones observada es un convolución entre e flujo de RCs $\Phi$ y la exposición direccional $\omega$:
    
    %     \begin{equation}
    %         \tilde{N} = \int d\Omega \Phi(\alpha, \delta) \omega(\alpha, \delta),
    %         \label{eq:conv}
    %     \end{equation}
    % \noindent y junto a la Ec.\ref{eq:i_alpha_phi}  se obtiene lo siguiente:
    % \begin{equation}
    %     I(\alpha) =  \int_{\delta_{min}}^{\delta_{max}} d\delta \cos\delta \,\, \Phi(\alpha, \delta) \omega(\alpha, \delta)
    %     \label{eq:i_coor_ecua}
    % \end{equation}


    El flujo de eventos observado $I^{obs}(\alpha_0)$ para la ascensión recta del cenit $\alpha^0$, entre los ángulos azimutales $\phi_1$ y $\phi_2$ puede calcularse mediante el flujo total de RCs $\Phi$ (expresado en coordenadas locales) como
    \begin{equation}
        I^{obs}(\alpha^0) = \int_{\phi_1}^{\phi_2} d\phi \int_{0}^{\theta_{max}} d\theta \sin\theta \tilde{\omega}(\theta, \alpha^0) \Phi(\theta, \phi, \alpha^0),
        \label{eq:rate_general}
    \end{equation}
    \noindent  donde  el término $\tilde{\omega}$ representa la exposición del observatorio. Este término también incluye los efectos sistemáticos y atmosféricos, como la variación de los hexágonos del arreglo y las correcciones de la modulación del clima, mediante  su dependencia con $\alpha^0$.

    Se considera que las amplitudes de las variaciones asociadas a $\tilde{\omega}$ son pequeñas con respecto al valor medio de $\tilde{\omega}$, y que pueden  desacoplarse de la dependencia de $\theta$. Por lo tanto, por lo que podemos expresar $\tilde{\omega}$ de la siguiente manera:
    \begin{equation}
        \tilde{\omega}(\theta, \alpha^0) = \omega(\theta)\big(1 + \eta(\alpha^0) \big)
        \label{eq:omega_expandido}
    \end{equation}

     Una anisotropía dipolar se puede describir de la siguiente manera:
    \begin{equation}
        \Phi(\hat{\bf{u}}) = \Phi_0(1+\bf{d}\cdot\hat{\bf{u}})
        \label{eq:dipolo_general}
    \end{equation}
    \noindent donde $\Phi_0$ es el flujo medio, $\hat{\bf{u}}$ es un versor que apunta a alguna dirección a estudiar y $\bf{d}$ es el vector con módulo $d$ igual a la amplitud del dipolo y  con dirección  con eje del dipolo, Tomando coordenadas ecuatoriales \footnote{Agregar también un apéndice de ecuatoriales}, la dirección de $\bf{d}$ es $(\alpha_d, \delta_d$)\footnote{Agregar fórmulas de cambio de sistema de referencia ecuatorial-local} y  de $\hat{\bf{u}}$ es $(\alpha, \delta)$, por lo tanto  el producto se puede escribir de la siguiente manera \footnote{Faltaría mencionar el producto de versor de esta representación para decir sale de acá, en el apéndice capaz. No sé, al final el cálculo me  sale fácil poniendo  todo en cartesianas.)}:
    \begin{equation*}
        \textbf{d}\cdot\hat{\bf{u}}= d (\cos\delta_d \cos\delta \cos(\alpha - \alpha_d) + \sin\delta_d  \sin\delta)
        \label{eq:product_ud}
    \end{equation*}
    Otro aspecto importante de la representación del dipolo en coordenadas ecuatoriales es que la proyección de la amplitud del dipolo sobre el plano ecuatorial se puede aproximar de la siguiente manera:
    \begin{equation}
        r_1 \simeq d_\perp \langle \cos\delta \rangle
        \label{eq:fourier_perp}
    \end{equation}
    donde $r_1$ es la amplitud de la aproximación a primer orden en Fourier.

     Por una cuestión de notación, definimos la siguiente expresión:
    \begin{equation}
        \overline{f(\theta)} = \int_{0}^{\theta_{max}} d\theta \sin\theta \omega(\theta) f(\theta)
        \label{eq:media_angular}
    \end{equation}
    \noindent donde $\overline{f(\theta)}$ es la media de la función $f(\theta)$ sobre el ángulo cenital pesado por la exposición del observatorio, hasta  un ángulo máximo. En este trabajo se centra en eventos hasta 2 EeV, por lo que $\theta_{max}=60^o$ para los datos del observatorio. 
    \begin{enumerate}
    \item Teniendo en cuenta la Ec.\ref{eq:omega_expandido} y \ref{eq:dipolo_general}, se tiene la siguiente expresión:
    \begin{align*}
        I^{obs}(\alpha^0) &= \int_{\phi_1}^{\phi_2} d\phi \int_{0}^{\theta_{max}} d\theta  \sin\theta \omega(\theta)\big(1 + \eta(\alpha^0) \big) \Phi_0 ( 1 +  \textbf{d}\cdot\hat{\bf{u}})
        % \\
        % &= \int_{\phi_1}^{\phi_2} d\phi \int_{0}^{\theta_{max}} d\theta \sin\theta \omega(\theta)\big(1 + \eta(\alpha^0) \big) \Phi_0 +\\
        % &+\int_{\phi_1}^{\phi_2} d\phi \int_{0}^{\theta_{max}} d\theta \sin\theta \omega(\theta)\big(1 + \eta(\alpha^0) \big) \Phi_0 \textbf{d}\cdot\hat{\bf{u}}
    \end{align*}
    \noindent la primera parte  de la igualdad  puede simplificarse con la definición \ref{eq:media_angular} e integrando sobre $\phi$
    \begin{align*}
        &\int_{\phi_1}^{\phi_2} d\phi \int_{0}^{\theta_{max}} d\theta \sin\theta \omega(\theta)\big(1 + \eta(\alpha^0) \big) \Phi_0 =\\
        &= \Phi_0 (1+ \eta(\alpha^0)) \pi \int_{0}^{\theta_{max}}  d\theta \sin\theta \omega(\theta)\\
        &= \Phi_0 (1+ \eta(\alpha^0)) \overline{1} 
    \end{align*}
    \noindent la integral sobre $\phi$ tiene el mismo valor para el Este y Oeste. Para la segunda parte de la expresión del item 8
    \begin{align}
        &\int_{\phi_1}^{\phi_2} d\phi \int_{0}^{\theta_{max}} d\theta \sin\theta \omega(\theta)\big(1 + \eta(\alpha^0) \big) \Phi_0 \textbf{d}\cdot\hat{\bf{u}}=\\
        &=\Phi_0 (1+ \eta(\alpha^0))\int_{\phi_1}^{\phi_2} d\phi \int_{0}^{\theta_{max}}  d\theta \sin\theta \omega(\theta)\textbf{d}\cdot\hat{\bf{u}} \label{segundo_term}
    \end{align}
    \noindent El dipolo está fijo en el cielo pero visto desde las coordenadas locales para poder trabajar con $\theta$ y $\phi$, sus proyecciones en los ejes de interés tienen una dependencia con la ascensión recta  $\alpha^0$ y declinación $\delta_0$ del cenit. %El versor $\hat{\bf{u}}$ expresado en coordenadas ecuatoriales apunta en la dirección $(\alpha^0, \delta_0)$. Consideremos las proyecciones del versor sobre el eje del cenit y sobre el plano horizontal del observador,  
    %\noindent donde $\hat{x}$ apunta en la dirección Este, $\hat{y}$ en la dirección Norte  y $\hat{z}$ en la dirección de cenit.
    Consideremos el dipolo proyectado en las dirección de los versores $\hat{x}$  que apunta en la dirección Este, $\hat{y}$ en la dirección Norte  y $\hat{z}$ en la dirección de cenit.
    \begin{equation*}
        \textbf{d} =  d_x(\alpha^0)\hat{x} +  d_y(\alpha^0)\hat{y}+ d_z(\alpha^0)\hat{z} ,
    \end{equation*}
    \noindent mientras que el versor apunta en la dirección de integración
    \begin{equation*}
        \hat{\bf{u}} =\sin\theta \cos\phi \hat{x} + \sin\theta \sin\phi \hat{y} + \cos\theta\hat{z}
    \end{equation*}
    Finalmente,
    \begin{align*}
        \textbf{d}\cdot\hat{\bf{u}} &= d_x(\alpha^0)\sin\theta \cos\phi
        + d_y(\alpha^0) \sin\theta \sin\phi  \\
        & + d_z(\alpha^0)\cos\theta
    \end{align*}
    Al integrar el ángulo  $\phi$ entre $[\nicefrac{-\pi}{2}, \nicefrac{\pi}{2}]$ o $[\nicefrac{\pi}{2}, \nicefrac{3\pi}{2}]$, el segundo término  se anula, por lo que la expresión \ref{segundo_term} queda como:
    \begin{align*}
        &\int_{\phi_1}^{\phi_2} d\phi \int_{0}^{\theta_{max}}  d\theta \sin\theta \omega(\theta)\textbf{d}\cdot\hat{\bf{u}} =\\
        &\int_{0}^{\theta_{max}}  d\theta (\pm 2d_x(\alpha^0)\sin\theta 
        + \pi d_z(\alpha^0)\cos\theta)
    \end{align*}     
    \noindent donde $+2$ corresponde al Este y $-2$ al Oeste. Podemos simplificar la expresión usando la definición \ref{eq:media_angular}
    \begin{align*}
    &\int_{0}^{\theta_{max}}  d\theta (\pm 2d_x(\alpha^0)\sin\theta 
    + \pi d_z(\alpha^0)\cos\theta)=\\ 
    & =\pm 2d_x(\alpha^0)\overline{\sin\theta} 
    + \pi d_z(\alpha^0)\overline{\cos\theta}\\
    \end{align*}



    \item 
    Para calcular los flujos de eventos del Este y Oeste, $I^{obs}_E$ y $I_O^{obs}$ respectivamente, se integra la Ec.\ref{eq:rate_general} en los siguientes  rangos:
    \begin{itemize}
        \item Para el Este: entre $\phi_1=\nicefrac{-\pi}{2}$ y $\phi_2=\nicefrac{\pi}{2}$.
        \item Para el Oeste: entre $\phi_1=\nicefrac{\pi}{2}$ y $\phi_2=\nicefrac{3\pi}{2}$.
    \end{itemize}


    Volviendo a la expresión de $I^{obs}$, teniendo en cuenta que necesitamos  $I^{obs}_E$ y $I^{obs}_O$:
    \begin{align*}
        I^{obs}_E&= \Phi_0 (1+ \eta(\alpha^0)) \Big( \pi\overline{1} + 2d_x(\alpha^0)\overline{\sin\theta} + \pi d_z(\alpha^0)\overline{\cos\theta}  \Big) \\
        I^{obs}_O&= \Phi_0 (1+ \eta(\alpha^0)) \Big( \pi \overline{1} - 2d_x(\alpha^0)\overline{\sin\theta}   + \pi d_z(\alpha^0)\overline{\cos\theta} \Big) 
    \end{align*}

    \item Como estamos buscando la diferencia entre estos valores, la resta queda como:
    \begin{equation*}
        I^{obs}_E -  I^{obs}_O = \Phi_0 (1+ \eta(\alpha^0)) \times 4  d_x(\alpha^0)\overline{\sin\theta}
    \end{equation*}
    
    \item Solo necesitamos las componentes del vector $\bf{d}$, para obtenerlas tenemos que considerar que los componentes que necesitamos están en el plano x-z. Para hacer esto, consideremos que los versores $\hat{\bf{u}}_z$ y $\hat{\bf{u}}_x$ que apuntan al cenit y al Este respectivamente. Considerando la fórmula para proyectar un vector sobre la dirección de un versor:
    \begin{equation}
        d_x(\alpha^0) \hat{x} =  (\textbf{d}\cdot\hat{\bf{u}}_x)\hat{\bf{u_x}} \rightarrow d_x(\alpha^0) = \textbf{d}\cdot\hat{\bf{u}}_x
    \end{equation}
    podemos obtener las proyecciones con un producto escalar con versores a las direcciones de interés. Estos versores en coordenadas ecuatoriales son los siguientes:
    \begin{align*}
        \hat{\bf{u}}_z &= (\alpha^0,\delta_0 )\\
        \hat{\bf{u}}_x &= (\alpha^0 + \frac{\pi}{2},0), 
    \end{align*}
    se suma  $\frac{\pi}{2}$ para apuntar al Este, cuando el versor recorre $\nicefrac{\pi}{2}$ en ascensión recta, llega al plano del ecuador que tiene declinación $0$.
    \item Finalmente para obtener las componentes:
    \begin{align*}
        \textbf{d}\cdot\hat{\bf{u}}_z &= d (\cos\delta_d \cos\delta_0 \cos(\alpha^0 - \alpha_d) + \sin\delta_d  \sin\delta_0)\\
        \textbf{d}\cdot\hat{\bf{u}}_x &= d (\cos\delta_d \cos(\alpha^0 +\frac{\pi}{2} - \alpha_d) 
        = -d\cos\delta_d \sin(\alpha^0  - \alpha_d)
    \end{align*}
    
     Entonces,
    \begin{equation}
        I^{obs}_E -  I^{obs}_O =-4d \Phi_0 (1+ \eta(\alpha^0)) \cos\delta_d \sin(\alpha^0  - \alpha_d)\overline{\sin\theta}
        \label{resta}
    \end{equation}
    \item Esta diferencia se debe relacionar con la variación del flujo verdadero, es decir el flujo que se observaría si no existieran variaciones temporales (ascensión recta) en la exposición. Esto implica que $\eta(\alpha^0)=0$.  Además el flujo total $I$ es la suma de los flujos de ambas direcciones, por lo tanto se puede afirmar que:
    \begin{align}
        I&=I_E +  I_O = 2\pi\Phi_0 (1+ 0) \Big( \overline{1} + d_z(\alpha^0)\overline{\cos\theta}  \Big)\\
        \dv{I^{obs}}{\alpha^0}  & = 2\pi\Phi_0 \overline{\cos\theta} \dv{\,d_z(\alpha^0) }{\alpha^0}\\ 
        \dv{I^{obs}}{\alpha^0} &= -2d\pi\Phi_0 \overline{\cos\theta}\cos\delta_d \cos\delta_0 \sin(\alpha^0 - \alpha_d) \label{total_flux}
    \end{align}

    \item Para llegar a la expresión \ref{resta}, hicimos la expansión hasta el primer orden de $\omega(\theta, alpha_0)$ y de $\Phi(\alpha, \delta)$. Para ser consistentes, desperdiciemos el término de segundo orden de la expresión \ref{resta} que es proporcional de $\eta \cdot d$ y la expresión \ref{resta} queda:
        \begin{equation}
            I^{obs}_E -  I^{obs}_O \approx -4d \Phi_0 \cos\delta_d \sin(\alpha^0  - \alpha_d)\overline{\sin\theta}
            \label{resta_final}
        \end{equation}
    Por lo tanto,
    \begin{equation}
        I^{obs}_E -  I^{obs}_O \approx  \frac{2}{\pi \cos \delta_0} \frac{\langle\sin\theta \rangle}{\langle\cos\theta \rangle}\dv{I^{obs}}{\alpha^0}
        \label{eq:final}
    \end{equation}
    donde se usa la expresión:
    \begin{equation*}
        \langle f(\theta) \rangle = \frac{\overline{f(\theta)}}{\overline{1}} = \displaystyle\frac{\int_{0}^{\theta_{max}} d \theta \sin\theta \omega(\theta) f(\theta) }{\int_{0}^{\theta_{max}} d \theta \sin\theta \omega(\theta)} 
    \end{equation*}
    que es equivalente a hacer la media ponderada con $\sin\theta\omega(\theta)$ de todos los datos de $f(\theta)$. %Como consideramos una expansión de $\omega(\theta)$ 
\end{enumerate}

\section{Estimación de la componente ecuatorial del dipolo mediante el análisis del  primer armónico}

Ya con la Ec.\ref{eq:final} podemos estimar la modulación dipolar de $I(\alpha^0)$ a partir de la amplitud $r$ y la fase $\phi_0$:
\begin{equation}
    \dv{I(\alpha^0)}{\alpha^0} = r \cos(\alpha^0 - \phi),
    \label{eq:dipolo_tasa}
\end{equation}
podemos estimar estos parámetros con un análisis similar a  Rayleigh, salvo modificaciones menores para tener en cuenta la dirección de los eventos, así podemos restar los coeficientes de los sectores Este y Oeste. Los coeficientes de Fourier en este caso se determinan con las siguientes expresiones:

\begin{align*}
    a_{EW} &= \frac{2}{N} \sum^N_{i=1} \cos(\alpha^0_i - \beta_i)\\
    b_{EW} &= \frac{2}{N} \sum^N_{i=1} \sin(\alpha^0_i - \beta_i)
\end{align*}
donde $N$ es la cantidad de eventos en el rango de tiempo estudiado y $\beta_i=0$ si el evento proviene del Este, caso contrario $\beta_i=1$. La amplitud  $r_{EW} = \sqrt{a_{EW}^2 + b_{EW}^2}$ y la fase $\phi_{EW} = \tan^{-1}(\nicefrac{b_{EW}}{a_{EW}})$ mediante este análisis en frecuencia es posible estimar los valores $r$ y $\phi$ de la Ec.\ref{eq:dipolo_tasa}:
\begin{align*}
    r &= \frac{\pi \cos\delta_0}{2} \frac{\langle\cos\theta \rangle}{\langle\sin\theta \rangle} r_{EW} \\ 
    &\text{integración} \rightarrow r_I =\frac{N}{2\pi}r \\
    \phi &= \phi_{EW} \\
    &\text{integración} \rightarrow \phi_I = \phi_{EW} + \frac{\pi}{2}
\end{align*}


La amplitud obtenida por el Método E-W no es la amplitud del dípolo físico aunque está relacionada con la misma. La Ec.\ref{eq:final} puede expresarse con la proyección del dípolo físico sobre el ecuador $d_{\perp}= d\cos\delta_0$, teniendo en cuenta la ecuación  \ref{resta_final}:
\begin{align}
    I^{obs}_E -  I^{obs}_O \approx -4 d_\perp \Phi_0 \sin(\alpha^0  - \alpha_d)\overline{\sin\theta},
\end{align}
si multiplicamos la expresión por una identidad y consideramos $N \sim 4\pi^2 \Phi_0 \overline{1} $ \footnote{Porque es la integral con respecto a los dos ángulos, $\theta$ y $\phi$}:
\begin{align}
    I^{obs}_E -  I^{obs}_O \approx -4 d_\perp \frac{N}{ 4\pi^2\overline{1}} \sin(\alpha^0  - \alpha_d)\overline{\sin\theta} \frac{\overline{1}}{\overline{1}}\\
    I^{obs}_E -  I^{obs}_O \approx -4 d_\perp \frac{N}{ 4\pi^2} \sin(\alpha^0  - \alpha_d)\langle\sin\theta \rangle\\
    I^{obs}_E -  I^{obs}_O \approx -\frac{N}{2\pi} d_\perp \frac{2\langle\sin\theta \rangle }{\pi}\sin(\alpha^0  - \alpha_d)
\end{align}

% Si consideramos que $\overline{1}$ es aproximadamente igual a la media de $\sin\theta$ de los valores posibles de $\theta$:
% \begin{align*}
%     \overline{1} &= \int_0^{\theta_{max}} d\theta \sin\theta \omega(\theta)
%                  \approx \int_0^{\frac{2\pi}{3}} \frac{\sin\theta}{\nicefrac{2\pi}{3} - 0} =  \frac{3}{4\pi}
% \end{align*}

% Entonces,
% \begin{align}
%     I^{obs}_E -  I^{obs}_O \approx -4 d_\perp \Phi_0 \sin(\alpha^0  - \alpha_d)\langle\sin\theta \rangle \frac{3}{4\pi}\\
%     I^{obs}_E -  I^{obs}_O \approx -4 d_\perp \Phi_0 \langle\sin\theta \rangle \frac{3}{4\pi} \sin(\alpha^0  - \alpha_d)
% \end{align}
Como esto es equivalente a $-r_I\sin(\alpha^0  - \alpha_d)$ por la ecuación \ref{eq:dipolo_tasa}, además de considerar la ecuación \ref{eq:fourier_perp}:
\begin{align}
    r  &= r_1 \frac{2\langle\sin\theta \rangle }{\pi}\\
    r &= \frac{\pi \cos\delta_0}{2} \frac{\langle\cos\theta \rangle}{\langle\sin\theta \rangle} r_{EW} \\
    &\Rightarrow  r_1 = \frac{\pi}{2} \frac{\langle\cos\delta \rangle}{\langle\sin\theta \rangle} r_{EW}
\end{align}
donde la última ecuación es la relación entre la amplitud del dipolo y la amplitud obtenida obtenida con el método East-West. Como en el caso del análisis de Rayleigh, la probabilidad de obtener una amplitud mayor o igual a que $r_EW$ a partir de una distribución isótropa una distribución acumulada de Rayleigh:

\begin{equation}
    P(\geq r_{EW}) = \exp{-\frac{N}{4}r^2_{EW}} = \exp{-\frac{N}{4} \Big ( \frac{2 \langle\sin\theta \rangle }{\pi \langle\cos\delta \rangle} \Big)^2 r^2_{1} }
\end{equation}


\section{Cálculo de la amplitud del dipolo para la frecuencia sidérea con el método East-West}

\begin{enumerate}
    \item Definimos el rango de tiempo a estudiar, para estos resultados se utilizaron los límites: 1 de Enero del 2014 hasta el 1 de Enero del 2020.
    \item Se recorre cada evento que cumpla con las siguientes características:
     \begin{itemize}
        \item Pertenezca el rango de energía a estudiar
        \item Sea un evento 6T5 con ángulo cenital menor a $60^o$
        \item Se haya registrado en el rango de tiempo seleccionado
    \end{itemize}
    En cada evento se calcula los siguientes valores:
    \begin{align}
        a' = \cos(X - \beta)\\
        b' = \sin(X - \beta)
    \end{align}
    el valor de $X$ depende la frecuencia a estudiar, la misma es igual a la ascensión recta del cenit $\alpha^0_i$ al momento del evento  si se estudia la frecuencia sidérea, en cambio para la frecuencia solar es igual al equivalente en grados de la hora local de Malargüe. El valor de $\beta$ es depende si el evento provino del Este donde $\beta=180^o$ o $\beta=0$ caso contrario.
    Se intentó hacer un barrido de frecuencias análogo al análisis de Rayleigh pero la variable utilizada para generalizar el análisis a frecuencias arbitrarias:
    \begin{equation}
        \tilde{\alpha} = 2\pi f_x t_i + \alpha_i - \alpha_i^0(t_i) \label{ra_mod}
      \end{equation}
    es tal que la variable es igual a la ascensión recta del evento a estudiar y no al cenit como es el caso del EW. 
    \item Una vez corridos todos los  eventos se calculan los parámetros:
    \begin{align*}
        a_{EW} &= \frac{2}{N} \sum^N_{i=1} a \qquad
        b_{EW} = \frac{2}{N} \sum^N_{i=1} b
    \end{align*}
    que es equivalente a haber calculado
    \begin{align*}
        a_{EW} &= \frac{2}{N} \sum^N_{i=1} \cos(\alpha^0_i - \beta_i)\\
        b_{EW} &= \frac{2}{N} \sum^N_{i=1} \sin(\alpha^0_i - \beta_i)
    \end{align*}
    donde N indica la cantidad eventos considerados. La cantidad de eventos por rango de energía se muestran en la tabla \ref{tab:}.

    Con esto puedo calcular la amplitud asociada al análisis $r_{EW}$ y la fase $\phi_{EW}$:
    \begin{align*}
        r_{EW} = \sqrt{a_{EW}^2 + b_{EW}^2}\\
        \phi_{EW} = \tan^{-1}(\nicefrac{b_{EW}}{a_{EW}})
    \end{align*}

    Estos valores se traducen a los valores de amplitud $r$ y fase $\phi$ del dipolo físico mediante las expresiones:
    \begin{align*}
        r &= \frac{\pi}{2} \frac{\langle\cos\delta \rangle}{\langle\sin\theta \rangle} r_{EW} \\
        d_\perp&= \frac{\pi}{2 \langle\sin\theta \rangle} r_{EW} = \frac{r}{\langle\cos\delta \rangle}\\
        \phi &= \phi_{EW} + \frac{\pi}{2}
    \end{align*}
    Se suma $\frac{\pi}{2}$ por el  artificio de agregar $\pi$ en los coeficientes para obtener la diferencia entre tasas del este y oeste. Los valores $\langle\cos\delta \rangle$ y $\langle\sin\delta \rangle$ son los valores medios de estas variables en los años estudiados. 

    \item Se calcula la amplitud límite $r_{99}$ y la probabilidad de que las amplitudes calculadas sea ruido  $P(r_{EW})$ mediante:
    \begin{align*}
        P(\geq r_{EW}) &= \exp{\Big(-\frac{N}{4}r^2_{EW}\Big)}\\
        r_{99} &= \frac{\pi}{2} \frac{\langle\cos\delta \rangle}{\langle\sin\theta \rangle}\sqrt{\frac{4}{N}\ln(100)}\\
        d_{\perp,99} &= \frac{r_{99}}{\langle\cos\delta \rangle}    
    \end{align*}

    \item Una vez obtenidos los valores a considerar, se calculan los errores asociados a cada variable, con las expresión a continuación:
    
    \begin{itemize}
        \item Error asociado a la amplitud $r$ y $d_\perp$
        \begin{align*}
          \text{r} &\rightarrow  \sigma   = \frac{\pi \langle\cos\delta \rangle}{2\langle\sin\theta \rangle} \sqrt{\frac{2}{\mathcal{N}}}\\
          d_\perp &\rightarrow   \sigma_{x,y} = \frac{\sigma}{\langle\cos\delta \rangle}
        \end{align*}
        \item Error asociado a la fase $\phi$ de la amplitud:
        \begin{align*}
            \sigma_{\phi} = \frac{1}{r_{EW}}\sqrt{\frac{2}{\mathcal{N}}}
        \end{align*}
        

    \end{itemize}

\end{enumerate}

% Por último, estos resultados se comparan con los valores obtenidos con el método EW en el trabajo \cite{Aab_2020} en frecuencia sidérea, aplicado al conjunto de eventos del disparo estándar registrados entre el 1 de Enero del 2004 y el 1 de Agosto del 2018. Para esto se ejecutó el programa implementado en el trabajo mencionado sobre los datos utilizados en el mismo, estos se obtuvieron de \emph{Publications Committee} de la colaboración Auger.


Por último, estos resultados se comparan con los valores obtenidos con el método EW en el trabajo \cite{Aab_2020} en frecuencia sidérea, aplicado al conjunto de eventos del disparo estándar registrados entre el 1 de Enero del 2004 y el 1 de Agosto del 2018. 
% Para esto se ejecutó el programa implementado en el trabajo mencionado sobre los datos utilizados en el mismo, estos se obtuvieron de \emph{Publications Committee} de la colaboración Auger.



\section{Cómo se hace el cálculo para frecuencias  arbitrarias}

Cambiamos las variable de la ascensión recta del cenit $\alpha_0$ por
\begin{equation}
    \tilde{\alpha} = 2\pi f_x t_i  \label{ra_arb}
  \end{equation}
donde $f_x$ es la frecuencia arbitraria a estudiar y $t_i$ es el momento donde ocurre el evento a estudiar. Luego se realizan el mismo procedimiento que lo anterior para calcular el valor de la amplitud $r$.

En la siguiente sección se verifica que se obtiene los mismo resultados con esta variable general que con el valor de $\alpha_0$ para la frecuencia sidérea.

\section{Verificación del código}

\subsection{Comparación con el trabajo \cite{Aab_2020} de la colaboración}
Se verificó el código escrito en este trabajo de la siguiente manera:

\begin{enumerate}
    \item El conjunto de eventos del disparo estándar registrados entre el 1 de Enero del 2004 y el 1 de Agosto del 2018 fue analizado en el trabajo \cite{Aab_2020}.
    \item Utilizando el código y los datos de los eventos del paper \cite{Aab_2020}, obtenidos de la página del \emph{Publications Committee} de la colaboración Auger, se replicaron los datos del paper. 
    \item Luego utilizando el código escrito para este trabajo, se realizó el análisis de EW con los datos del trabajo \cite{Aab_2020}. 
    \item Finalmente se verificó que los valores obtenidos en los item 2 y 3, con  ambos códigos, sean el mismo.
\end{enumerate}

\subsection{Tabla comparando con Right ascension}

Para verificar que la variable de la Ec.\ref{ra_arb} es útil para estudiar otras frecuencias, en la Tabla~\ref{tab:comp_vars} se comparan los resultados de la referencia para el rango $0.25-0.5$ EeV, los obtenidos usando la ascensión recta del cenit y los valores obtenidos con la Ec.\ref{ra_arb} en el mismo rango de energía. Se observan que los valores son comparables entre sí.


\begin{table}[H]
    \begin{small}
        \begin{center}
            \begin{tabular}[c]{l|l|l|l}
                                    & \cite{Aab_2020} & $\alpha_0$   & $\alpha=2\pi f_xt_i$   \\ 
                Frecuencia:         & 366.25          &  366.25      &  366.25            \\
                $d_\perp$[\%]:      & 0.60            &  0.60        &  0.60              \\
                $\sigma_{x,y}$[\%]  & 0.48            &  0.48        &  0.48              \\ 
                Probabilidad:       & 0.45            &  0.45        &  0.45              \\
                Fase[$^o$]:         & 225$\pm$64\cite{discrepancia} & 225$\pm$45   &  227$\pm$45          \\
                $r_{99}$[\%]:       & 1.5             &  1.5       &  1.5             \\
                $d_{\perp,99}$[\%]: & 1.8             &  1.8       &  1.8             \\
            \end{tabular}
        \end{center}
        \caption{Verificando la  variable $\alpha=2\pi ft$}
        \label{tab:comp_vars}
    \end{small}
\end{table}




En el trabajo \cite{linsley1975fluctuation} se estudian los límites de confianza para la amplitud $r_1$ y la fase $\phi$ obtenidos mediante el análisis del primer armónico en Fourier. Las distribuciones de probabilidad describen a un conjunto de N mediciones cuya anisotropía está caracterizada por el vector $\vec{s}$ con una dispersión $\sigma = \sqrt{\nicefrac{2}{N}}$.  Sin pérdida de generalidad, se puede restar a las mediciones la fase $\phi$ para que las mismas varíen alrededor del 0. Este vector $\vec{s}$ puede ser obtenido mediante distintos métodos, en este trabajo se  utilizaron el método de Rayleigh e East - West, en este caso, el módulo del vector $\vec{s}$  es igual a $r_1$.

La distribución de probabilidad de la amplitud y la fase está dada por la Ec.\ref{eq:full_pdf}. Las variables $r$ y $\psi$ representan las variaciones de módulo y fase de las mediciones con respecto a $\vec{s}$
\begin{equation}
    p(r,\psi) =dr\,d\psi\,\frac{r}{2\pi\sigma^2}\exp{ -\frac{(r^2+s^2 - 2rs\cos\psi)}{2\sigma^2} } \label{eq:full_pdf}
\end{equation}  

\section{Distribución de probabilidad de la amplitud}

Integrando la Ec.\ref{eq:full_pdf} con respecto a $\psi$, se obtiene la función de densidad de probabilidad $p(r)$ y el nivel de confianza $ CL_r(r_i,r_f,s)$ entre en rango $[r_i,r_f]$:
\begin{align}
    p(r) &=\frac{r}{\sigma^2}\exp{ -\frac{(r^2+s^2)}{2\sigma^2} }K_0(\frac{rs}{\sigma^2})    \label{ec:pdf}\\
    CL_r(r_i,r_f,s) &= \int_{r_i}^{r_f} dr \, p(r)
    \label{ec:integral}
\end{align}
donde $K_0(x)$ es la función de Bessel modificada de primer orden.
Estas ecuaciones nos permiten determinar el nivel de confianza $CL$ con el cual se puede afirmar que el módulo del dipolo se encuentra entre los valores $r_i$ y $r_f$, dado un conjunto de mediciones.

Se define el valor $r^{UL}$ como el límite superior donde se puede afirmar que el módulo de dipolo se encuentra en el rango $[0, r^{UL}]$ con un $99\%$ de certeza.
\begin{align}
    CL_r(0,r^{UL},s) = 0.99 = \int_{0}^{r^{UL}} dr \, p(r)
    \label{ec:r_upper_limit}
\end{align} 
% Para alcanzar un  nivel del confianza  del  CL[\%] \footnote{ Donde CL=.99 para un 99\% o CL=0.68 para un 68\%,},  se toma el valor de amplitud $r^{UL}$ y la integral de la función \ref{ec:pdf} desde 0 hasta $r^{UL}$, donde el resultado debe ser el nivel de confianza CL.
% \begin{align}
%     CL = \int_{0}^{r^{UL}} dr \frac{r}{\sigma^2}\exp{\Big( -\frac{(r^2+s^2)}{2\sigma^2} + \frac{rs}{\sigma^2}\Big)}K_0(\frac{rs}{\sigma^2})
%     \label{ec:integral}
% \end{align}

Suponiendo que mediante el análisis de un conjunto de eventos, se obtiene que $s=0.0047$ y $\sigma=0.0038$. El gráfico de la función $p(r)$ se muestra a continuación:

\begin{figure}[H]
    \begin{small}
        \begin{center}
            \includegraphics[width=0.75\textwidth]{bessel_prob_value_s_v2.pdf}
        \end{center}
        \caption{El gráfico de la densidad de probabilidad $p(r)$ de la amplitud $r$ para $s=0.0047$ y $\sigma=0.0038$ }
    \end{small}
\end{figure}

\subsection{Haciendo la cuenta de los márgenes de confianza de la amplitud}

Calculemos los márgenes de confianza para el ejemplo anterior de $s=0.0047$ y $\sigma=0.0038$. En este trabajo los márgenes que se obtuvieron nos dicen que el nivel de confianza en ese intervalo del $68.27\%$. Se toma este límite, dado que si $N>>1$, la distribución $p(r)$ tiende a una distribución normal y el nivel de confianza sería $1\sigma$.

Los pasos para el cálculo sigo son los siguientes: 

\begin{enumerate}
    \item 
    Dado que la distribución tiene una función de bessel modificada de primer orden que diverge en el 0, se toma una aproximación a la función con los primeros 8 términos de la sucesión. Por lo que la función no es exacta y la norma difiere de $1$. 
    
    Para normalizar el área, se calcula la integral hasta $r_{max}=s +  10\sigma$, dado que está tan alejada del valor de amplitud obtenida, el nivel de confianza en $CL_r(0,r_{max},s)\simeq 1$, por lo que se  usa este valor para normalizar la Ec. \ref{ec:pdf} en el código.

    \item Una vez que se tiene la función normalizada, se calcula la integral de la ecuación \ref{ec:integral} $CL_r(0,s,s)$ en el intervalo  $[0,s]$ y se obtiene el valor de la función $p(s)=p_1$.

    \item Si $CL_r(0,s,s)< 0.6827$:
    \begin{enumerate}
        \item Teniendo en cuenta el valor inicial de $p_1$, se actualiza el valor  $p_2 \leftarrow p_1 - 0.01 p_1$ \label{itm:1}.
        \item Se calcula la integral entre los dos puntos con valores igual a $p(r)_2$. 
        \item \label{itm:3} Si la integral es menor a $0.6827$, se repite el proceso desde el paso \ref{itm:1}. Caso contrario, si esta integral es mayor o igual a $0.6827$, se calculan los valores límites de $r$ mediante el valor $p_2$ en el paso \ref{pasofinal}. 

        La Fig.\ref{fig:itera} se muestra el área calculada en la primera iteración que se muestra verde, el valor de área obtenido no es el nivel de confianza buscada se sigue iterando hasta alcanzar el valor $p_N$, donde la integral entre esos extremos es de $0.6827$.
        \begin{figure}[H]
            \begin{small}
                \begin{center}
                    \includegraphics[width=0.75\textwidth]{bessel_prob_iterations_v2.pdf}
                \end{center}
                \caption{Iteraciones para encontrar los márgenes de confianza del $68.27\%$ de la distribución de probabilidad de la amplitud. En la N-ésima iteración se obtiene los límite de confianza buscados.}
                \label{fig:itera}
            \end{small}
            
        \end{figure}
    \end{enumerate}
    \item Si $CL_r(0,s,s)> 0.6827$:
    \begin{enumerate}
        \item Se toma como límite inferior $r^-$el valor $s$ y se busca el límite superior $r^+$ de tal forma que $CL(s,s+\sigma^+,s) \simeq 0.6827$.
    \end{enumerate}
    \item \label{pasofinal} Los límites de confianza superior $r^+$  y inferior $r^-$, teniendo en cuenta el valor final $p_N$ del paso \ref{itm:3}, son tales que se cumple $p(r^+)=p(r^-)=p_N$. Finalmente los márgenes de confianza se calculan como:
    \begin{align*}
        \sigma^- = s-r^-\\
        \sigma^+ = r^+ -s
    \end{align*}
\end{enumerate}

En la Fig.\ref{margenes} se muestran los márgenes de confianza obtenidos para el ejemplo de $s=0.0047$ y $\sigma=0.0038$, el área sombreada es igual al $0.6827$
\begin{figure}[H]
    \begin{small}
        \begin{center}
            \includegraphics[width=0.75\textwidth]{bessel_prob_ej_v2.pdf}
        \end{center}
        \caption{Densidad de probabilidad de la amplitud $r$ para $s=0.0047$ y $\sigma=0.0038$. Se muestran los márgenes de confianza del $68.27\%$ }
        \label{margenes}
    \end{small}
\end{figure}

\section{Distribución de probabilidad de la fase del dipolo}

Integrando la ecuación \ref{eq:full_pdf} con respecto a $r$ en el rango $[0,\infty]$, se obtiene la distribución de probabilidad de la fase $\psi$ de la Ec.\ref{eq:phase_pdf}. Este apartado considera que las fases de la mediciones varían alrededor del cero. De esta forma, la distribución de probabilidad tiene la característica  de ser simétrica respecto a 0, por eso los límites de integración para obtener un nivel de confianza igual a 1 son $[-\pi, \pi]$.
\begin{align}
    p(\psi) &=d\psi\,\frac{1}{2\pi}e^{-k} \Bigg[ 1 + (\pi k)^{\nicefrac{1}{2}} \cos\psi e^{(k\cos^2\psi)} \Big( 1 + L \erf(L k^{\nicefrac{1}{2}} \cos\psi \Big) \Bigg ] \\ \label{eq:phase_pdf}
    CL_{\psi}(\phi_1, \phi_2, s) &= \int_{\phi_1}^{\phi_2} d\psi \, p(\psi)
\end{align}  
donde $k =\nicefrac{s^2}{2\sigma^2}$ y $\erf (x)$ es la función error, y
\begin{align*}
    L =
    \begin{cases} 
        +1 & \text{ Si } -\frac{\pi}{2} \leq x\geq \frac{\pi}{2} \\
        -1 & \text{ Caso contrario }  \\
     \end{cases}
\end{align*}

% La distribución de probabilidad tiene la característica  de ser simétrica respecto a 0, por eso los límites de integración son $[-\pi, \pi]$.

Se definió que el nivel de confianza para la fase reportada en este trabajo sea del $68.27\%$, ya que $k>>1$ la distribución de la fase se acerca a una distribución normal y este nivel de confianza es equivalente a $\sigma_\phi$. 

Para calcular el margen de confianza $\sigma_\psi$,  dada la simetría de la función \ref{eq:phase_pdf}  con respecto al 0, se siguen los siguientes  pasos:

\begin{enumerate}
    \item Se toma un valor inicial de $\sigma_{\psi,0}=0.01|\phi|$, donde  $\phi$ es valor de fase obtenida ya sea por el método Rayleigh o East-West. Se eligió este valor inicial por conveniencia.
    \item Se integra la Ec.\ref{eq:phase_pdf} en el rango $[-\sigma_{\psi,0}, \sigma_{\psi,0}]$ y se verifica si $CL_{\psi}(-\sigma_{\psi,0}, \sigma_{\psi,0},s) = 0.9545$.  Si ese es el caso, se reporta la fase como $\psi \pm \sigma_{\psi,0}$, caso contrario se vuelve al paso anterior con $\sigma_{\psi,1} \leftarrow \sigma_{\psi,0} + 0.01\sigma_{\psi,0}$. \label{paso2} y se itera hasta obtener el valor de $\sigma_{\psi,N}$ que cumpla $CL_{\psi}(-\sigma_{\psi,N}, \sigma_{\psi,N},s) = 0.6827$
\end{enumerate}

En la Fig.\ref{fig:phase_prob_ej} se muestra la distribución de probabilidad de la fase para $s=0.0047$ y $\sigma = 0.0038$, también se incluye los límites de confianza obtenidos.

\begin{figure}[H]
    \begin{small}
        \begin{center}
            \includegraphics[width=0.75\textwidth]{phase_prob_ej_v2.pdf}
        \end{center}
        \caption{La distribución de probabilidad de la fase $\psi$ para $s=0.0047$ y $\sigma = 0.0038$ con los márgenes de confianza del $68.27\%$.}
        \label{fig:phase_prob_ej}
    \end{small}
\end{figure}


\chapter{Resultados del método East - West}

En este capítulo se presentan los resultados obtenidos mediante el método East-West con los eventos de Todos los Disparos, para  distintos rangos de energía. Estos resultados se comparan con los valores obtenidos en \cite{Aab_2020} sobre los eventos  del Disparo Estándar. 
% \section{Tabla cantidad de eventos para distintos rangos de energía}

Los eventos son clasificados en los distintos rangos mediante la energía reportada por la Colaboración. El conjunto de eventos registrados mediante de Todos los Disparos abarca eventos medidos entre el 2014 y 2019, y para el Disparo Estándar se listan eventos medidos entre el 2004 y 2018. Las características de estos dos conjuntos de datos se especifican en la Tabla \ref{tab:datasets}

\begin{table}[H]
    \begin{small}
        \begin{center}
            \begin{tabular}{lc|l|l|l|}
\hline
\multicolumn{1}{|l|}{\multirow{4}{*}{\begin{tabular}[c]{@{}c@{}}Rango \\ Tiempo\end{tabular}}}    & Todos       & Inicio &\multicolumn{2}{l|}{1 de Enero, 2014 } \\ \cline{3-5} 
\multicolumn{1}{|l|}{}                                                                            & 6 años      & Fin    &\multicolumn{2}{l|}{1 de Enero, 2020} \\ \cline{2-5} 
\multicolumn{1}{|l|}{}                                                                            & Estándar    & Inicio &\multicolumn{2}{l|}{1 de Enero, 2004} \\ \cline{3-5}
\multicolumn{1}{|l|}{}                                                                            & 14.8 años   & Fin    &\multicolumn{2}{l|}{1 de Octubre, 2018} \\ \hline  \\

\hline                                                                          \multicolumn{2}{|c|}{Rango [EeV]}                                                    & \multicolumn{1}{c|}{0.25 - 0.5}  & \multicolumn{1}{c|}{ 0.5  - 1 } &\multicolumn{1}{c|}{ 1 - 2 } \\ \hline
\multicolumn{1}{|l|}{\multirow{2}{*}{Eventos}}                            & Todos    & $3\,967\,368$     & $3\,638\,226$   & $1\,081\,846$ \\ \cline{2-5} 
\multicolumn{1}{|l|}{}                                                    & Estándar & $770\,323$        & $2\,388\,468$   & $1\,243\,098$ \\ \hline
\multicolumn{1}{|l|}{\multirow{2}{*}{\begin{tabular}[c]{@{}c@{}}Energía \\ Media\end{tabular}}} & Todos    & $0.38$           & $0.69$         & $1.32$       \\ \cline{2-5} 
\multicolumn{1}{|l|}{}                                                                             & Estándar & $0.42$            & $0.71$          & $1.34$       \\ \hline
\end{tabular}
            \caption{Características de los conjuntos de datos para distintos rangos de energía }
            \label{tab:datasets}
        \end{center}
    \end{small}
\end{table}



\section{Resultados en distintos rangos de energía}
\subsection{Resultados en el rango 0.25 EeV - 0.5 EeV}

En la Tabla \ref{tab:primer_bin_data} se presentan los resultados para este rango de energía en las frecuencias solar y sidérea de Todos Los Disparos. Los mismos  se comparan con resultados con el Disparo Estándar que fueron reportados en \cite{Aab_2020}. En esta tabla se observa que las amplitudes $r$ en frecuencia sidérea no son compatibles dentro de los límites de confianza de cada uno. Los valores de $\sigma$ de Todos los Disparos es la mitad que el valor reportado para el Disparo Estándar,  esto se debe a que el primer conjunto de datos tiene registrados $\sim 5$  veces más eventos que el segundo.

\begin{table}[H]
    \begin{small}
        \begin{center}
            \begin{tabular}[c]{l|c|c||c|}
\cline{2-4}                                       & \multicolumn{2}{c||}{Todos los disparos}    & \multicolumn{1}{c|}{Disparo Estándar}   \\ \hline
\multicolumn{1}{|l|}{Frecuencia:                } & Solar	                & Sidérea	                & Sidérea \cite{Aab_2020}   \\ \hline
\multicolumn{1}{|l|}{Amplitud r [\%]:           } & $0.17^{+0.22}_{-0.07}$	& $0.12^{+0.24}_{-0.03}$ 	& $0.5^{+0.4}_{-0.2}$ \cite{codigo}      \\
\multicolumn{1}{|l|}{$r_{99}$ [\%]:             } & \multicolumn{2}{c||}{0.58}                          & 1.1\cite{codigo}                 \\
\multicolumn{1}{|l|}{$r^{UL}$ [\%]:             } & 0.67 	                & 0.64                      & 1.4\cite{codigo}                 \\ 
\multicolumn{1}{|l|}{$\sigma$[\%]:              } & \multicolumn{2}{c||}{0.19}                          & 0.38\cite{codigo}       \\\hline
\multicolumn{1}{|l|}{Amplitud $d_\perp$[\%]:    } & -	                    & $0.16^{+0.31}_{-0.04}$ 	& $0.6^{+0.5}_{-0.3}$       \\
\multicolumn{1}{|l|}{$d_{99}$ [\%]:             } & - 	                    & 0.73                      & 1.5  \cite{codigo}                \\
\multicolumn{1}{|l|}{$d_{\perp}^{UL}[\%]$       } & -                       & 0.80                      & 1.8                         \\
\multicolumn{1}{|l|}{$\sigma_{x,y}$[\%]:        } & -	                    & 0.24	                    & 0.48       \\\hline
\multicolumn{1}{|l|}{Probabilidad      :        } & 0.66                    & 0.81	                    & 0.45       \\
\multicolumn{1}{|l|}{Fase[$^o$]:                } & 221$\pm$93              & 280$\pm$124                & 225$\pm$64\\ \hline
\multicolumn{1}{|l|}{$\langle\cos\delta \rangle$} & \multicolumn{2}{c||}{0.79}        	                & 0.79 \cite{codigo}        \\        
\multicolumn{1}{|l|}{$\langle\sin\theta \rangle$} & \multicolumn{2}{c||}{0.46}        	                & 0.52 \cite{codigo}        \\ \hline       
            \end{tabular}
            
        \end{center}
    \end{small}
    \caption{Características para las frecuencias solar y sidérea con el método East-West en el primer armónico en rango de energía 0.25 EeV - 0.5 EeV.}
    \label{tab:primer_bin_data}
\end{table}


En la Fig. \ref{fig:primer} se comparan las  fases en frecuencia sidérea obtenida en este trabajo y la reportada en \cite{Aab_2020}, donde la línea punteada marca la dirección del centro galáctico.  En esta figura en la tabla anterior, se observa que la incertidumbre obtenida para la fase de Todos los Disparos es amplia, esto se debe a que la amplitud $r$ es pequeña comparada con el valor de $\sigma$. 


Realizando el barrido de frecuencias con la variable de la Ec.\ref{ra_arb}, se obtiene que en este rango de energía las amplitudes se  distribuyen en frecuencia como se muestra en la Fig.\ref{fig:primer_barrido}. La línea horizontal indica el valor de $r_{99}$ para cada frecuencia, además se observa que ninguna amplitud supera dicho umbral.

\begin{figure}[H]
    \begin{small}
        \begin{center}
            \includegraphics[width=0.75\textwidth]{phase_primer_bin_v2.pdf}
        \end{center}
        \caption{Valores de las fases obtenidos en este trabajo y en el trabajo \cite{Aab_2020} con sus respectivas incertidumbres para la frecuencia sidérea en el  rango 0.25 EeV - 0.5 EeV .}
        \label{fig:primer}
    \end{small}
\end{figure}

\begin{figure}[H]
    \begin{small}
        \begin{center}
            \includegraphics[width=0.75\textwidth]{plot_bin_1_barrido_v3_EW.pdf}
        \end{center}
        \caption{Barrido de frecuencias en el  rango 0.25 EeV - 0.50 EeV .}
        \label{fig:primer_barrido}
    \end{small}
\end{figure}

\subsection{Resultados en el rango 0.5 EeV - 1 EeV}
En la Tabla \ref{tab:primer_bin_data} se presentan los resultados para el rango 0.5 EeV - 1 EeV en las frecuencias solar y sidérea de Todos Los Disparos, además se comparan con los resultados reportados en \cite{Aab_2020}.


\begin{table}[H]
        \begin{small}
            \begin{center}
                \begin{tabular}[c]{l|c|c||c|}
\cline{2-4}                                       & \multicolumn{2}{c||}{Todos los disparos}    & \multicolumn{1}{c|}{Disparo Estándar}   \\ \hline
\multicolumn{1}{|l|}{Frecuencia:                } & Solar	                & Sidérea	                & Sidérea \cite{Aab_2020}   \\ \hline
\multicolumn{1}{|l|}{Amplitud r [\%]:           } & $0.43^{+0.21}_{-0.14}$	& $0.44^{+0.21}_{-0.14}$ 	& $0.38^{+0.20}_{-0.14}$ \cite{codigo}      \\
\multicolumn{1}{|l|}{$r_{99}$ [\%]:             } & \multicolumn{2}{c||}{0.56}                         & 0.64\cite{codigo}                 \\
\multicolumn{1}{|l|}{$r^{UL}$ [\%]:             } & 0.89 	                & 0.90                      & 0.90 \cite{codigo}                 \\ 
\multicolumn{1}{|l|}{$\sigma$[\%]:              } & \multicolumn{2}{c||}{0.18}                         & 0.21 \cite{codigo}      \\\hline
\multicolumn{1}{|l|}{Amplitud $d_\perp$[\%]:    } & -	                    & $0.56^{+0.27}_{-0.18}$ 	& $0.5^{+0.3}_{-0.2}$       \\
\multicolumn{1}{|l|}{$d_{99}$ [\%]:             } & - 	                    & 0.71                      & 0.8   \cite{codigo}                \\
\multicolumn{1}{|l|}{$d_{\perp}^{UL}[\%]$       } & -                       & 1.1                       & 1.1                         \\
\multicolumn{1}{|l|}{$\sigma_{x,y}$[\%]:        } & -	                    & 0.23	                    & 0.21       \\\hline
\multicolumn{1}{|l|}{Probabilidad      :        } & 0.065                   & 0.055	                    & 0.20       \\
\multicolumn{1}{|l|}{Fase[$^o$]:                } & 205$\pm$35              & 258$\pm$34                & 261$\pm$43\\ \hline
\multicolumn{1}{|l|}{$\langle\cos\delta \rangle$} & \multicolumn{2}{c||}{0.79}        	                & 0.79 \cite{codigo}        \\        
\multicolumn{1}{|l|}{$\langle\sin\theta \rangle$} & \multicolumn{2}{c||}{0.50}        	                & 0.54\cite{codigo}        \\ \hline       
                \end{tabular}
            \end{center}
        \end{small}
        \caption{Características para las frecuencias solar y sidérea con el método East-West en el primer armónico en rango de energía 0.5 EeV - 1 EeV}
        \label{tab:segundo_bin_data}
    \end{table}


    En la Fig. \ref{fig:segundo} se comparan las direcciones en las que apuntan la fase en frecuencia sidérea obtenida en este trabajo con la obtenida en \cite{Aab_2020}. En esta figura se observa que resultados similares entre sí en valor e incertidumbre, y apuntan a una dirección cercana al centro galáctico.

    El barrido de frecuencias con la variable de la Ec.\ref{ra_arb} para este rango de energía se observa en la Fig.\ref{fig:segundo_barrido}. La línea horizontal indica el valor de $r_{99}$ para cada frecuencia, además se observa que ninguna frecuencia supera dicho umbral. 
    
    \begin{figure}[H]
        \begin{small}
            \begin{center}
                \includegraphics[width=0.75\textwidth]{phase_segundo_bin_v2.pdf}
            \end{center}
            \caption{Valores de las fases obtenidos en este trabajo y en el trabajo \cite{Aab_2020} con sus respectivas incertidumbres para la frecuencia sidérea en el rango 0.5 EeV - 1.0 EeV .}
            \label{fig:segundo}
        \end{small}
    \end{figure}


    \begin{figure}[H]
        \begin{small}
            \begin{center}
                \includegraphics[width=0.75\textwidth]{plot_bin_2_barrido_v3_EW.pdf}
            \end{center}
            \caption{Barrido de frecuencias en el  rango 0.5 EeV - 1.0 EeV .}
            \label{fig:segundo_barrido}
        \end{small}
    \end{figure}    


\subsection{Resultados en el rango 1 EeV - 2 EeV}

 
En las Tablas \ref{tab:solar_3}  se comparan los resultados de este trabajo  para la frecuencia solar. Las amplitudes están por debajo de $r_{99}$ y con compatibles entre sí.

    \begin{table}[H]
        \begin{small}
            \begin{center}
                \begin{tabular}[c]{l|c|c|}
                    \cline{2-3}         & \multicolumn{2}{c|}{Todos los disparos} \\ \cline{2-3}
                                        & Rayleigh                      & East - West            \\\hline
\multicolumn{1}{|l|}{Frecuencia:}             &\multicolumn{2}{c|}{Solar}        \\
\multicolumn{1}{|l|}{Amplitud $r$[\%]:} & $0.24^{+0.16}_{-0.09}$        & $0.28^{+0.35}_{-0.11}$ \\
\multicolumn{1}{|l|}{$r_{99}$ [\%]:   } & 0.41                          & 0.91       \\
\multicolumn{1}{|l|}{$r_{UL}$ [\%]:   } & 0.58                          & 1.1       \\
\multicolumn{1}{|l|}{$\sigma$:        } & 0.14                          & 0.30          \\\hline
\multicolumn{1}{|l|}{Probabilidad:    } & 0.22                          & 0.65          \\
\multicolumn{1}{|l|}{Fase:            } & 260$\pm$48                    & 279$\pm$90    \\\hline
                \end{tabular}
            \end{center}
        \end{small}
        \caption{Características para la frecuencia solar con los métodos de Rayleigh  e East-West en el primer armónico.}
        \label{tab:solar_3}
    \end{table}
    
    En la Tabla \ref{tab:siderea_3} se comparan los resultados de este trabajo y los obtenidos en el trabajo \cite{Aab_2020} para la frecuencia sidérea. Para Todos los Disparos se comparan los métodos de Rayleigh y East-West, en el primer método se obtiene que la probabilidad que la amplitud obtenida se deba al ruido es de $6
    3\%$ mientras que en segundo método $26\%$. Esta diferencia entre probabilidades no puede deberse a la cantidad de eventos, porque es el mismo conjunto de datos. El método Rayleigh nos indica que en este rango de energía pueden existir efectos sistemáticos que no están siendo corregidos.


    \begin{table}[H]
        \begin{small}
            \begin{center}
                \begin{tabular}[c]{l|c|c||c|}
                    \cline{2-4}               &  \multicolumn{2}{c||}{Todos los Disparos}                  & Disparo Estándar      \\
                    \cline{2-4}               & Rayleigh                      & East - West                 & East - West\cite{Aab_2020}      \\\hline
\multicolumn{1}{|l|}{Frecuencia:             }& \multicolumn{2}{c||}{Sidérea}                               & Sidérea        \\ \hline
\multicolumn{1}{|l|}{Amplitud $r$ [\%]:      }& $0.32^{+0.16}_{-0.10}$ 	      & $0.5^{+0.3}_{-0.2}$         & $0.14^{+0.37}_{-0.02}$\cite{codigo}       \\
\multicolumn{1}{|l|}{$r_{99}$[\%]:           }& 0.41	                      & 0.91                        & 0.84\cite{codigo}        \\
\multicolumn{1}{|l|}{$r^{UL}[\%]$      }      & 0.66                          & 1.3                         & 0.89 \cite{codigo}        \\
\multicolumn{1}{|l|}{$\sigma$[\%]:     }      & 0.14                          & 0.30	                    & 0.28 \cite{codigo}          \\ \hline
\multicolumn{1}{|l|}{Amplitud $d_\perp$ [\%]:}& $0.41^{+0.20}_{-0.13}$        & $0.6^{+0.4}_{-0.3}$         & $0.18^{+0.47}_{-0.02}$       \\ 
\multicolumn{1}{|l|}{$d_{99}$[\%]:           }& 0.53	                      & 1.1                         & 1.1\cite{codigo}        \\
\multicolumn{1}{|l|}{$d_{\perp}^{UL}[\%]$    }& 0.84                          & 1.6                         & 1.1        \\
\multicolumn{1}{|l|}{$\sigma_{x,y}$[\%]:     }& 0.17                          & 0.38	                    & 0.35          \\ \hline
\multicolumn{1}{|l|}{Probabilidad:           }& 0.063	                      & 0.26                        & 0.87          \\
\multicolumn{1}{|l|}{Fase[$^o$]:             }& 357$\pm$35                    & 320$\pm$50                 & 291$\pm$100      \\\hline
\multicolumn{1}{|l|}{$\langle\cos\delta\rangle$} & \multicolumn{2}{c||}{0.78}                              & 0.78       \\        
\multicolumn{1}{|l|}{$\langle\sin\theta\rangle$} & \multicolumn{2}{c||}{0.55}                              & 0.57       \\ \hline       
\end{tabular}
            \end{center}
        \end{small}
        \caption{Características para la frecuencia sidérea con los métodos de Rayleigh  e East-West en el primer armónico.}
        \label{tab:siderea_3}
    \end{table}
   

    En el Fig.\ref{fig:tercer} se observan en un gráfico polar las fases del trabajo \cite{Aab_2020} y este trabajo para la frecuencia sidérea
    \begin{figure}[H]
        \begin{small}
            \begin{center}
                \includegraphics[width=0.75\textwidth]{phase_tercer_bin_v2.pdf}
            \end{center}
        \caption{Valores de las fases obtenidos en este trabajo y en el trabajo \cite{Aab_2020} con sus respectivas incertidumbres para la frecuencia sidérea en el  rango 1.0 EeV - 2.0 EeV .}
        \label{fig:tercer}
        \end{small}
    \end{figure}


    El barrido de frecuencias con la variable de la Ec.\ref{ra_arb} para este rango de energía se observa en la Fig.\ref{fig:tercer_barrido}. La línea horizontal indica el valor de $r_{99}$ para cada frecuencia y se observa que ninguna frecuencia supera dicho umbral. En la frecuencia solar no se observa ningún pico, esto se debe a que el método East - West es robusto con respecto a las modulación del clima. Se observa un pico en sidérea pero el mismo no es significativo con respecto al $r_{99}$.


    \begin{figure}[H]
        \begin{small}
            \begin{center}
                \includegraphics[width=0.75\textwidth]{plot_bin_3_barrido_v3_EW.pdf}
            \end{center}
            \caption{Barrido de frecuencias en el rango 1 EeV - 2 EeV .}
            \label{fig:tercer_barrido}
        \end{small}
    \end{figure}    

    \section{Gráficos}

    % Para poder comparar los resultados de $d_\perp$ entre sí, podríamos graficar los valores de la proyección y de la límite del $99\%$ como se muestra en la Fig.\ref{fig:no_normalizado}. El inconveniente es la cantidad de datos en cada rango de energía entre los conjuntos de datos, Todos los Disparos y Disparo Estándar, son distintos.


    % \begin{figure}[H]
    %     \begin{small}
    %         \begin{center}
    %             \includegraphics[width=0.75\textwidth]{d_perp_no_normalizado_v4.pdf}
    %         \end{center}
    %         \caption{Sin normalizar}
    %         \label{fig:no_normalizado}
    %     \end{small}
    % \end{figure}
    
    % Para compararlos mejor con respecto a $d_{\perp,UL}$, usamos el valor de cada rango y de cada conjunto de datos, para normalizar la amplitud de $d_{\perp,UL}$. Como se muestra en la Fig.\ref{fig:normalizado}, ahora $d_{\perp,UL}=1$ y los otros valores se pueden comparar. 

    % \begin{figure}[H]
    %     \begin{small}
    %         \begin{center}
    %             \includegraphics[width=0.75\textwidth]{d_perp_normalizado.pdf}
    %         \end{center}
    %         \caption{Valores normalizados con $d_{\perp,UL}$}
    %         \label{fig:normalizado}
    %     \end{small}
    % \end{figure}

    Una forma  para poder comparar los resultados de $d_\perp$ calculados de distintos conjuntos de  datos entre sí, es dividir estos valores con  sus respectivos $\sigma_{x,y}$. De esta manera, podemos comparar cuan apartados están con respecto $\sigma_{x,y}$, así se obtiene la Fig.\ref{fig:normalizado_sigma}.

    \begin{figure}[H]
        \begin{small}
            \begin{center}
                \includegraphics[width=0.75\textwidth]{d_perp_normalizado_sigmas_v5.pdf}
            \end{center}
            \caption{Valores normalizados con $d_{\perp,UL}$}
            \label{fig:normalizado_sigma}
        \end{small}
    \end{figure}

Por lo que ahora podemos decir que en los rangos entre 0.5 EeV - 1.0 EeV y 1.0 EeV - 2.0 EeV, la amplitud obtenida en este trabajo está por encima que en el trabajo \cite{Aab_2020} por $\sim 1\sigma_{x,y}$ y $\sim 2 \sigma_{x,y}$ respectivamente.

Para comparar los resultados en el  rango 0.25 EeV - 0.5 EeV, tenemos que tener en cuenta que el Disparo Estándar tiene una sensibilidad menor que el Todos los Disparos. Esto se ve claramente en la Tabla \ref{tab:datasets}, donde el primero tiene 7 veces menos eventos para analizar que el segundo. Por lo tanto, la discrepancia entre en el trabajo \cite{Aab_2020} y los trabajos puede deberse a la  diferencia de eventos a estudiar causada por la sensibilidad del disparo.


Considerando los valores de $\sigma_{x,y}$ y $d_\perp$ obtenidos para cada rango de energía, es posible  comparar las direcciones, valores e incertidumbres en la Fig.\ref{fig:incertidumbre}. Las líneas punteadas están centradas en los valores reportados en el trabajo \cite{Aab_2020} en cada rango de energía y con radio igual a sus $\sigma_{x,y}$ . 

\begin{figure}[H]
    \begin{small}
        \begin{center}
            \includegraphics[width=0.9\textwidth]{comparando_sigmas_v3.pdf}
        \end{center}
        \caption{Amplitudes con incertidumbre, apuntando en la dirección  de la fase. Los círculos punteados los valores del trabajo \cite{Aab_2020}del trabajo \cite{Aab_2020} con sus respectivas incertidumbres y la línea punteada en negro marca la dirección del centro galáctico.}
        \label{fig:incertidumbre}
    \end{small}
\end{figure}


\chapter{Esto va a ir en otro capítulo que todavía no armé. Y los resultados no estan actualizados}
\section{Comparando resultados entre métodos para barridos de frecuencias}

\section{Verificación del código escrito durante la maestría}

Para ver que todo cierre, obtuve los resultados del paper \cite{Aab_2020} con el código del Rayleigh para distintos bines. En el bin  2 EeV - 4 EeV  tuve incongruencias entre mi código y los valores reportados en el paper, pero si comparo los valores obtenidos con el código utilizado para el paper con mis resultados si se corresponden. En los demás bines los resultados entre el código implementado en \cite{Aab_2020}, los resultados publicados y los resultados de mi código se corresponden.


En el bin 2 EeV - 4 EeV, verifiqué sin cambiaba los números considerando los eventos hasta $80^o$, pero los parámetros de Rayleigh eran los mismos que usar $60^o$  como límite en $\theta$.  Cuando no considero los pesos en mi código, obtengo resultados congruentes con los publicados pero eso puedo ser una casualidad.

\begin{table}[H]
    \begin{small}
        \begin{center}
            \begin{tabular}[c]{l|c|c|c|c|}
                                            & \multicolumn{4}{|c|}{2 EeV - 4 EeV}                                                               \\ \hline
                Frecuencia:                 & Sidérea              & Sidérea (Sin pesos)  & Sidérea \cite{codigo}    & Sidérea \cite{Aab_2020}   \\ \hline
                Amplitud r [\%]:            & $0.5^{+0.3}_{-0.2}$ & $0.4^{+0.3}_{-0.2}$ & $0.5^{+0.3}_{-0.2}$     & -                          \\
                $r_{99}$ [\%]:              & 0.8                 & 0.8                 & 0.8                     & -                          \\\hline
                Amplitud $d_\perp$[\%]:     & $0.7^{+0.4}_{-0.2}$ & $0.5^{+0.4}_{-0.2}$ & $0.7^{+0.4}_{-0.2}$ 	  & $0.5^{+0.4}_{-0.2}$                    \\
                $d_{99}$ [\%]:              & 1.0                 & 1.0                 & 1.0                     & -                         \\
                $d_{\perp,UL}$[\%]:         & 1.9                 & 1.7                 & -                       & 1.4                               \\\hline
                $\sigma_{x,y}$[\%]:         & 0.34	              & 0.34	            & 0.34	                  & 0.34                           \\
                Probabilidad      :         & 0.14                & 0.33                & 0.15               	  & 0.34                       \\
                Fase[$^o$]:                 & 355$\pm$29          & 351$\pm$38          & 346$\pm$29              & 349$\pm$55                    \\\hline
            \end{tabular}
        \end{center}
    \end{small}
    \caption{Características para las frecuencias solar y sidérea con el método Rayleigh en el primer armónico en el rango de energía 2 EeV - 4 EeV, obtenidos con el código de este trabajo \cite{Aab_2020} y comparados con los resultados reportados en el último.}
\end{table}


\begin{table}[H]
    \begin{small}
        \begin{center}
            \begin{tabular}[c]{l|c|c||c|c|}
                                            & \multicolumn{2}{c||}{8 EeV - 16 EeV}              & \multicolumn{2}{c|}{16 EeV - 32 EeV}                   \\ \hline
                Frecuencia:                 & Sidérea                    & Sidérea \cite{Aab_2020} & Sidérea                   & Sidérea \cite{Aab_2020}   \\ \hline
                Amplitud r [\%]:            & $4.4^{+1.0}_{-0.8}$ 	    & -                      & $5.8^{+1.8}_{-1.3}$ 	    & -                         \\
                $r_{99}$ [\%]:              & 2.6                       & -                      & 4.9                      & -                          \\\hline
                Amplitud $d_\perp$[\%]:     & $5.6^{+1.2}_{-1.0}$ 	    & $5.6^{+1.2}_{-1.0}$    & $7.5^{+2.3}_{-1.8}$ 	    & $7.5^{+2.3}_{-1.8}$                   \\
                $d_{99}$ [\%]:              & 3.3                       & -                      & 6.3                      & -                         \\
                $d_{\perp,UL}$[\%]:         & 10                        & -                      & 16                       & -                                 \\\hline
                $\sigma_{x,y}$[\%]:         & 1.1	                    & 1.1                    & 2.1	                    & 2.1                           \\
                Probabilidad      :         & $2.3\times10^{-6}$	    & $2.3\times10^{-6}$     & $1.5\times10^{-3}$	    & $1.5\times10^{-3}$              \\
                Fase[$^o$]:                 & 96$\pm$11                 & 97$\pm$12              & 80$\pm$16                & 80$\pm$17                     \\\hline
            \end{tabular}
        \end{center}
    \end{small}
    \caption{Características para las frecuencias solar y sidérea con el método Rayleigh en el primer armónico en distintos rangos de energía, obtenidos con el código de este trabajo \cite{Aab_2020} y comparados con los resultados reportados en el último.}
\end{table}


\section{Comparando amplitud en función de la frecuencia}

En las Figs.\ref{fig:primer_barrido_EW_Ray}, \ref{fig:segundo_barrido_EW_Ray} y \ref{fig:tercer_barrido_EW_Ray} se comparan el barrido en frecuencia con el método East - West y el barrido con Rayleigh considerando los pesos de los hexágonos en distintos rangos de energía.


\begin{figure}[H]
    \begin{small}
        \begin{center}
            \includegraphics[width=0.4955\textwidth]{plot_bin_1_barrido_v3_EW.pdf}
            \includegraphics[width=0.4955\textwidth]{plot_bin_1_barrido_v1_Ray.pdf}
        \end{center}
        \caption{Barrido de frecuencias en el rango 0.25 EeV - 0.5 EeV .}
        \label{fig:primer_barrido_EW_Ray}
    \end{small}
\end{figure}    

\begin{figure}[H]
    \begin{small}
        \begin{center}
            \includegraphics[width=0.4955\textwidth]{plot_bin_2_barrido_v3_EW.pdf}
            \includegraphics[width=0.4955\textwidth]{plot_bin_2_barrido_v1_Ray.pdf}
        \end{center}
        \caption{Barrido de frecuencias en el rango 0.5 EeV - 1 EeV .}
        \label{fig:segundo_barrido_EW_Ray}
    \end{small}
\end{figure}    

\begin{figure}[H]
    \begin{small}
        \begin{center}
            \includegraphics[width=0.4955\textwidth]{plot_bin_3_barrido_v3_EW.pdf}
            \includegraphics[width=0.4955\textwidth]{plot_bin_3_barrido_v1_Ray.pdf}
        \end{center}
        \caption{Barrido de frecuencias en el rango 1 EeV - 2 EeV .}
        \label{fig:tercer_barrido_EW_Ray}
    \end{small}
\end{figure}    



\begin{biblio}
	\bibliography{mibib.bib}
\end{biblio}
    
    \end{document}
