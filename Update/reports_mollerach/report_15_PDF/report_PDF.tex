\input{Preamblev2.sty}

\begin{document}
%%%%%%%%%%%%%%%%%%%%%%%%%%%%%%%%%%Título%%%%%%%%%%%%%%%%%%%%%%%%%%%%%%%%%%%%%%
%%%%%%%%%%%%%%%%%%%%%%%%%%%%%%%%%%%%%%%%%%%%%%%%%%%%%%%%%%%%%%%%%%%%%%%%%%%%%%

\title{Sobre el código de la PDF de la amplitud}
\author{Evelyn~G.~Coronel}

\affiliation{
Tesis de Maestría en Ciencias Físicas\\ Instituto Balseiro\\}

\date[]{\lowercase{\today}} %%lw para lw, [] sin date

%\begin{abstract}

%\end{abstract} 
\maketitle
%%%%%%%%%%%%%%%%%%%%%%%%%%%%%%%%%%%%%%%%%%%%%%%%%%%%%%%%%%%%%%%%%%%%%%%%%%%%%%%%%%%
% Podemos usar cualquiera de los dos comandos: \input o \include para incluir el texto


\section*{Como  es la PDF de la amplitud}

La función de la distribución tiene la siguiente forma:

\begin{equation}
    P(s) = \int_{s}^{0} dr \frac{r}{\sigma}
\end{equation}


\end{document}
