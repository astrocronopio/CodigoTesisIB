\begin{itemize}
\done Para calcular los coeficientes del weather, queremos los mejores eventos, asi que tambien se usan solo los 6T5 (no es tan importante ganar un poquito mas de estadistica arribade 4 EeV para eso).

\done En resumen usa los cortes 6T5 y $\theta<60$ para anisotropias y para weather correction.

\done El corte de quality weather flag solo se usa para seleccionar los eventos para calcular las correcciones del weather.

\done Los resultados nuevos son un poco raros, en siderea desaparecio toda la señal cuando pones pesos 1, y despues crece algo con los pesos. Revisa con los cortes bien puestos.

\done Pone una tabla con amplitud y fase con y sin peso tambien.

\end{itemize}


20/05/2020

\begin{itemize}


\done Una cosa que se me ocurre es que al plot de hexagonos en frecuencia siderea le fitees un coseno. De ahi saca la amplitud del primer armónico de la modulación y la fase en RA donde esta el maximo.

\done Despues hace el analisis de anisotropia en los eventos sin pesos y con pesos, y obtene la amplitud de la modulación y las fases del máximo en ambos casos.

\done De ahi podriamos ver comparando las cantidades vectoriales (no solo la amplitud de la modulacion sino para donde apunta) qué es lo que esta pasando.

\done Haceme un mail con esos resultados cuando los tengas, a ver si entendemos eso.

\end{itemize}